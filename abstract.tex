In this thesis, we consider \emph{Hardware-based Security Enforcement} (HSE)
mechanisms, a class of security enforcement mechanisms such that trusted
software components rely on the underlying hardware architecture to constrain
the execution of untrusted software components with respect to targeted
security properties.

HSE mechanisms are particularly used to keep the software components of modern
software stacks isolated from each other.
%
As a consequence, hardware components play an essential role in securing
computing platforms.
%
Yet during the past decades, several critical vulnerability disclosures have
shown that trust in hardware can be misplaced.
%
Modern computing platforms have grown in complexity, and now involve multiple
hardware components executing sophisticated software stacks.
%
This paves the way for \emph{architectural attacks}, where the attacker is
able to leverage one component to threaten the security enforced by another.

With SpecCert, we have proposed a three-step methodology, supported by a Coq
framework, for specifying and verifying HSE mechanisms against hardware
architecture models.
%
We have written a minimal model for a single core x86-based computing platform,
and used SpecCert to specify and verify a real HSE mechanism to isolate the
code executed while the CPU is in System Management Mode (SMM), a highly
privileged execution mode of x86 microprocessors.

For many years, industrial manufacturers and researchers have aimed to
formally specify and verify computing platform.
%
Defining a comprehensive model in terms of hardware and software components
remains challenging, yet such a model is a prerequisite to verify a computing
platform in terms of architectural attacks.
%
We have proposed FreeSpec, a Coq framework built upon a formalism to specify and verify each component independently, then verify their
\emph{composition}.