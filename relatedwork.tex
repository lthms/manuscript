\documentclass[oneside,a4paper]{memoir}

% ABOUT THIS FILE
% ---------------
%
% The goal of this document is to prepare the next (and final?) version of the
% chapter 3.

\usepackage{geometry}
\usepackage{paralist}
\usepackage{hyperref}

\usepackage[T1]{fontenc}
\usepackage[utf8]{inputenc}
%\usepackage[sc]{mathpazo}    % Palatino with smallcaps
%\usepackage[scaled]{helvet}  % Helvetica, scaled 95%
%\usepackage{eulervm}
\usepackage{inconsolata}

\usepackage[dvipsnames]{xcolor}
\usepackage{xargs}
\usepackage{todonotes}
\newcommandx{\thomasrk}[2][1=]{\todo[linecolor=Plum,backgroundcolor=Plum!25,bordercolor=Plum,#1]{#2}}

\usepackage{acro}
\DeclareAcronym{acpi}{
  short = ACPI,
  long = Advanced Configuration and Power Interface,
  class = abbrev
}
\DeclareAcronym{tcg}{
  short = TCG,
  long = Trusted Computing Group,
  class = abbrev
}
\DeclareAcronym{tpm}{
  short = TPM,
  long = Trusted Platform Module,
  class = abbrev
}
\DeclareAcronym{idt}{
  short = IDT,
  long = Interrupt Descriptor Table,
  class = abbrev
}
\DeclareAcronym{pch}{
  short = PCH,
  long = Platform Controller Hub,
  class = abbrev
}
\DeclareAcronym{mch}{
  short = PCH,
  long = Memory Controller Hub,
  class = abbrev
}
\DeclareAcronym{smrr}{
  short = SMRR,
  long = System Manangement Range Registers,
  class = abbrev
}
\DeclareAcronym{uefi}{
  short = UEFI,
  long = Unified Extensible Firmware Interface,
  class = abbrev
}
\DeclareAcronym{dma}{
  short = DMA,
  long = Direct Memory Access,
  class = abbrev
}
\DeclareAcronym{dram}{
  short = DRAM,
  long = Dynamic Random Access Memory,
  class = abbrev
}
\DeclareAcronym{acsl}{
  short = ACSL,
  long = ANSI/ISO C Specification Language,
  class = abbrev
}
\DeclareAcronym{epc}{
  short = EPC,
  long = Enclave Page Cache,
  class = abbrev
}
\DeclareAcronym{sgx}{
  short = SGX,
  long = Software Guard eXtension,
  class = abbrev
}
\DeclareAcronym{sfi}{
  short = SFI,
  long = Software-based Fault Isolation,
  class = abbrev
}
\DeclareAcronym{nacl}{
  short = NaCl,
  long = Native Client,
  class = abbrev
}
\DeclareAcronym{hse}{
  short = HSE,
  long = Hardware-based Security Enforcement,
  class = abbrev
}
\DeclareAcronym{xom}{
  short = XOM,
  long = eXecute Only Memory,
  class = abbrev
}
\DeclareAcronym{lts}{
  short = LTS,
  long  = Labelled Transition System,
  class = abbrev
}
\DeclareAcronym{cpu}{
  short = CPU,
  long  = Central Processing Unit,
  class = abbrev
}
\DeclareAcronym{mmu}{
  short = MMU,
  long  = Memory Management Unit,
  class = abbrev
}
\DeclareAcronym{bios}{
  short = BIOS,
  short-plural = es,
  long  = Basic Input/Output System,
  class = abbrev
}
\DeclareAcronym{smm}{
  short = SMM,
  long  = System Management Mode,
  class = abbrev
}
\DeclareAcronym{smi}{
  short = SMI,
  long  = System Management Interrupt,
  class = abbrev
}
\DeclareAcronym{apic}{
  short = APIC,
  long  = Advanced Programmable Interrupt Controller,
  class = abbrev
}
\DeclareAcronym{tcb}{
  short = TCB,
  long  = Trusted Computing Base,
  class = abbrev
}


\begin{document}
\chapter{State of the Art of Formal Verification}

% 1. Start with a generic definition of formal verification%
Formal verification consists in proving the correctness of a system,
characterized by a mathematical model, with respect to certain properties,
defined as logic formula.
%
The result of decades of academic researches and industrial investments is a
large collection of complementary
formalisms\,\cite{berthomieu1991petri,mcmillan1989compositional,baeten2005processalgebra,emmi2008interfaceautomata,hoare1969hoare,reynolds2002separation}
to model systems and defined properties of interest, and
tools\,\cite{cuoq2012framac,coq,cimatti2002nusmv,jackson2012alloy} to assist and
automate the verification process.
%
This allowed for applying formal verification to a broad variety of computer
science domains, including cryptographic
protocols\,\cite{goubault2000crypto,meadows2003crypto},
compilers\,\cite{leroy2012compcert}, system software
component\,\cite{klein2009sel4,gu2016certikos} or hardware
designs\,\cite{lie2003xom,kaivola2009formalintel,reid2016armv8}.
%
With regard to \ac{hse} mechanisms, they fall between two domains that are
hardware design verification and low-level software components verification.
%
On the one hand, hardware designs verification often focus on properties
transparent to the executed software components (\emph{e.g.} cache
coherency\,\cite{stern1995cachecoherence}, linearizability of SGX
instructions\,\cite{leslie2015sgx}, hardware-based memory
isolation\,\cite{lie2003xom}).
%
Security gap in interactions of multiple platform components are less regarded
due to their increasing complexity\,\cite{potlapally2011hardwaresecurity}, yet
they are responsible for architectural attacks we want to avoid.
%
On the other hand, low-level software components such as
seL4\,\cite{klein2009sel4} or CertiKOS\,\cite{gu2016certikos} use the features
exposed by these components, and are verified against \emph{ad hoc} hardware
models, whose scope is often limited to necessary hardware features.
%
%For our part, we would rather characterize a set of sufficient requirements over
%low-level software components, such that any software implementation which satisfy these
%requirements and is executed by the hardware architecture proves to be correct
%with respect to a targeted property.

\bibliographystyle{unsrt}%
\bibliography{manuscript}
\end{document}