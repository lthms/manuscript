\chapter{Introduction}

\endquote{``\emph{All problems in computer science can be solved by another
    level of indirection.}''

  \hfill\footnotesize --- David Wheeler}

\vspace{1cm}\noindent To manage complexity, computing platforms are commonly
built as successions of abstraction layers, from the hardware architecture's
components up to high-level software applications.
%
Each layer leverages the interface of its predecessor to expose a higher-level,
more constrained set of functionalities for its successors.
%
This enables separation of concerns ---each layer encapsulates one dimension of
the overall complexity--- and modularity ---two layers which expose the same
interface can be seamlessly interchange---.

\thomasrk[inline]{Describe common layer: HW, FW, OS, Apps}

In this context, each layer is often \emph{more privileged} than, and therefore
implicitly trusted by, its successors.
%
From a security perspective, the former may constrain the latter, with respect
to targeted security properties.
%
For instance, an operating system shares hardware resources among several
applications, and therefore is able to enforce availability ---fair share of CPU
time---, confidentiality and integrity ---exclusive partition of physical
memories--- properties.
%
Constraining one's execution can be achieved in various ways.
%
Concerning the lowest layers of a software stack, the common approach is to rely
on features provided by the hardware architecture.
%
For instance, mainstream operating systems leverage, among others mechanisms, a
\emph{Memory Management Unit} (MMU) to partition the system memory and a
hardware timer to schedule arbitrary applications.

\section{Hardware-based Security Enforcement mechanisms}

We call this class of security enforcement mechanisms, where a trusted software
component configure the underlying hardware architecture to constrain the
execution of untrusted and potentially arbitrary software components,
\emph{Hardware-based Security Enforcement} (HSE).
%
A HSE mechanism enforces its targeted security property when both the hardware
features are correctly implemented and the software components correctly
configure them.

Both remain challenging.

Over the past decades, vendors have added many security features to their
products.
%
Intel, for instance, have notably introduced hardware-based virtualization
(VT-x, VT-d), dynamic root of trust (TXT), or applicative enclaves (SGX).
%
Most of them have been compromised at least once.
%
At the same time, it has been repeatedly shown that vendors were not always
correctly taking advantage of important hardware features at their disposal.
%
These lacks of hardware configuration put the affected computing platforms at
risk.
%
Many security vulnerabilities disclosed over the years have been the result of a
misconfiguration of hardware features by key software components.

In addition, hardware architecture often comprise hundreds of features
implemented by dozens of interconnected devices.
%
This paves the road for a class of security vulnerabilities we call
\emph{architectural attacks}, where each component is working as expected, yet
their composition creates an attack path.
%
This latter class of security vulnerabilities differs from the two other,
because it concerns the specifications of the computing platform rather than its
implementations.

\section{Towards Formal Verification}

In this thesis, our initial objective is to formally specify HSE mechanisms,
which implies defining both the targeted security property and the required
configuration steps trusted software components have to perform.
%
The expected result is an unambiguous, security-focused specification which can
be leveraged by software developers.
%
It would be an advantageous complement to the existing hardware specifications,
as the latter are often made of many independent, large documents (\emph{e.g.}
datasheets, developer manuals).

This initial specification effort shall serve as a premise for further
verification work for both the hardware architecture and the trusted software
components.
%
To adopt a verification approach guided by HSE mechanisms specifications could
help uncover architectural attacks ahead of time.
%
This obliges us to consider the hardware architecture as a whole, which poses
many challenges.

\section{Contributions}

This thesis proceeds as follows. In Chapter\,\ref{chapter:usecase}, we describe
in more detail HSE mechanisms commonly used in computing platforms, several
architectural attacks disclosed over the past decode to motivate our interest.
%
In particular, we introduce the SMRAM Cache Poisoning, to act afterward as a
running example for our contributions.
%
We first introduce a formalism to specify a HSE mechanism against hardware
model.
%
In Chapter\,\ref{chapter:relatedwork}
