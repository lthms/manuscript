%!TEX root = ./main.tex
In this thesis, we consider a class of security enforcement mechanisms we called
\emph{Hardware-based Security Enforcement} (HSE).
%
In such mechanisms, some trusted software components rely on the underlying
hardware architecture to constrain the execution of untrusted software
components with respect to targeted security policies.
%
For instance, an operating system which configures page tables to isolate userland
applications implements a HSE mechanism.\FOO{Ca sonne faux, je mettrais an HSE partout}

For a HSE mechanism to correctly enforce a targeted security policy, it requires
both hardware and trusted software components to play their parts.
%
During the past decades, several vulnerability disclosures have defeated HSE
mechanisms.
%
We focus on the vulnerabilities that are the result of errors at the
specification level, rather than implementation errors.
%
In some critical vulnerabilities, the attacker makes a legitimate use of one
hardware component to circumvent the HSE mechanism provided by another one.
%
For instance, cache poisoning attacks leverage inconsistencies between cache
and DRAM's access control mechanisms.
%
We call this class of attacks, where an attacker leverages inconsistencies in
hardware specifications, \emph{compositional attacks}.

Our goal is to explore approaches to specify and verify HSE mechanisms using
formal methods that would benefit both hardware designers and software
developers.
%
Firstly, a formal specification of HSE mechanisms can be leveraged as a
foundation for a systematic approach to verify hardware specifications, in the
hope of uncovering potential compositional attacks ahead of time.
%
Secondly, it provides unambiguous specifications to software developers, in the
form of a list of requirements.

Our contribution is two-fold:
%
\begin{itemize}
\item We propose a theory of HSE mechanisms against hardware
  architecture models. This theory can be used to specify and verify such
  mechanisms.
  %
  To evaluate our approach, we propose a minimal model for a single core
  x86-based computing platform.
  %
  We use it to specify and verify the HSE mechanism provided by Intel to isolate
  the code executed while the CPU is in System Management Mode (SMM), a highly
  privileged execution mode of x86 microprocessors.
  %
  We have written machine-checked proofs in the Coq proof assistant to that
  end.
\item We propose a novel approach inspired by algebraic effects to enable
  modular verification of complex systems made of interconnected components as a first step towards addressing the challenge posed by the
  scale of the x86 hardware architecture.
  %
  This approach is not specific to hardware models, and could also be leveraged
  to reason about composition of software components as well.
  %
  In addition, we have implemented our approach in the Coq theorem prover, and
  the resulting framework takes advantages of Coq proof automation features to
  provide general-purpose facilities to reason about components interactions.
\end{itemize}

\paragraph{Keywords:}
%
Security $\bullet$ Hardware Verification $\bullet$ Formal Specification
$\bullet$ Formal Methods $\bullet$ Coq
