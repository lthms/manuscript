\documentclass{article}

\usepackage[english]{babel}
\usepackage[T1]{fontenc}
\usepackage[utf8]{inputenc}
\usepackage{hyperref}

\title{Specifying and Verifying Hardware-based Security Enforcement Mechanisms}
\author{Thomas Letan}
\date{}

\begin{document}

\maketitle

\section{Abstract}

In this thesis, we consider a class of security enforcement mechanisms we called
\emph{Hardware-based Security Enforcement} (HSE).
%
In such mechanisms, some trusted software components rely on the underlying
hardware architecture to constrain the execution of untrusted software
components with respect to targeted security properties.
%
For instance, an Operating System that sets up page tables to isolate userland
applications implements a HSE mechanism.

For a HSE mechanism to correctly enforce a targeted security property, it
requires both hardware and trusted software components to play their parts.
%
During the past decades, several vulnerability disclosures have defeated HSE
mechanisms.
%
We focus on the vulnerabilities that are the result of errors at a specification
level, rather than implementation errors.
%
In some critical vulnerabilities, the attacker makes a legitimate use of one
hardware component to circumvent the HSE mechanism provided by another one.
%
For instance, cache poisoning attacks leverages inconsistencies between cache
and DRAM's access control mechanisms.
%
We call this class of security vulnerabilities \emph{architectural attacks}.

Our objective is to explore systemic approaches to specify and verify HSE
mechanisms using formal methods.
%
We believe this would benefit both hardware designers and software developers.
%
Firstly, a formal specification of HSE mechanisms can be leveraged as a
foundation for a systemic approach to verify hardware specifications, in the
hope of uncovering potential architectural attacks ahead of time.
%
Secondly, it provides unambiguous specifications to software developers. We
believe these specifications can be a valuable addition to the existing
documentation.
%
This is in line with an ongoing effort, for many years and by industrial
manufacturers and researchers alike, to formally verify computing platforms.
%
Unfortunately, defining a comprehensive model in terms of hardware and software
components remains challenging, yet such a model is a prerequisite to verify HSE
mechanisms.

Our contribution is two-side:
%
\begin{itemize}
\item We proposed a formal definition of HSE mechanisms against hardware
  architecture models to be used to specify and verify them\,[1].
  %
  To evaluate our approach, we proposed a minimal model for a single core
  x86-based computing platform.
  %
  We used it to specify and verify the HSE mechanism provided by Intel to
  isolate the code executed while the CPU is in System Management Mode (SMM), a
  highly privileged execution mode of x86 microprocessors.
  %
  We have written machine-checked proofs in the Coq proof assistant to that
  end\,[4].
\item We proposed a novel approach to verify compositions of components and to
  enable modular verification of a complex hardware architecture, inspired by
  algebraic effects\,[2].
  %
  This approach is not specific to hardware models, and could also be leveraged
  to reason about software applications as well.
  %
  In addition, we have implemented FreeSpec, a framework for the Coq proof
  assistant\,[5].
  %
  FreeSpec is a complete implementation of our approach, and takes advantages of
  the proof automation features of Coq mechanisms to ease the verification
  process.
\end{itemize}

\paragraph{Keywords}
%
Security $\bullet$ Hardware Verification $\bullet$ Formal Specification
$\bullet$ Formal Methods $\bullet$ Coq

\section{Publications}

\subsection{Peer-reviewed Conferences}

\begin{itemize}
\item[] [1] \textbf{SpecCert: Specifying and Verifying Hardware-based Security
    Enforcement
    Mechanisms.} \\
  Thomas Letan, Pierre Chifflier, Guillaume Hiet, Pierre Néron, Benjamin Morin,
  \emph{21st International Symposium on Formal Methods (FM 2016)}.
\item[] [2] \textbf{Modular Verification of Programs with Effects and Effect
    Handlers in Coq.} \\
  Thomas Letan, Yann Régis-Gianas, Pierre Chifflier, Guillaume Hiet, \emph{22st
    International Symposium on Formal Methods (FM 2018)}.
\end{itemize}

\subsection{Seminar}

\begin{itemize}
\item[] [3] FreeSpec : Modular Verification of Systems using Effects and Effect
  Handlers in Coq, \emph{Analyse et conception de systèmes, IRIF} --
  \url{https://www.irif.fr/seminaires/acs/index}
\end{itemize}

\subsection{Open Source Software}

\begin{itemize}
\item[] [4] \textbf{SpecCert}, a Coq framework for specifying and verifying
  Hardware-based Security Enforcement, distributed under the terms of the
  CeCILL-B --- \url{https://github.com/lthms/speccert}
\item[] [5] \textbf{FreeSpec}, a Coq framework for modularly verifying programs
  with effects and effect handlers, distributed under the terms of the GPL-3 ---
  \url{https://github.com/ANSSI-FR/FreeSpec}
\end{itemize}

\end{document}