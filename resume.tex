%!TEX root = ./main.tex
\chapter{Résumé de la thèse}

\begin{otherlanguage}{french}
  Dans le cadre de cette thèse, nous nous intéressons à une classe particulière
  de mécanismes de sécurité. Ces mécanismes nécessitent qu'un ou plusieurs logiciels de confiance configurent la plate-forme matérielle afin de contraindre l’exécution du reste de la pile logicielle à
  respecter une politique de sécurité donnée.
  %
  Nous qualifierons par la suite de mécanisme HSE (de l’anglais
  \emph{Hardware-based Security Enforcement}) les instances de cette
  classe.
  %
  L’utilisation par le système d’exploitation des mécanismes de pagination pour
  isoler chaque application est sûrement l’exemple le plus courant de la mise en
  pratique d’un mécanisme HSE.

  \subsection*{Contexte}

  Le contournement d’une politique de sécurité normalement assurée par le biais
  d’un mécanisme HSE peut s’expliquer par une erreur dans sa mise en œuvre,
  aussi bien dans l’un des logiciels de confiance chargé de sa configuration que
  dans les mécanismes matériels sur lesquels il se repose.
  %
  Nous nous somme intéressés à ce second cas, et plus spécifiquement encore au
  problème des erreurs affectant les spécifications de la plate-forme.
  %
  Cette dernière est composée de plusieurs dizaines de composants interagissant
  ensemble et la complexité résultante de ces interactions rend en pratique
  difficile la conception d’un mécanisme à visée sécuritaire.
  %
  Pour chaque nouvelle fonctionnalité que le concepteur de la plate-forme désire
  ajouter, il est nécessaire de s’assurer
  %
  \begin{inparaenum}[(1)]
  \item qu’elle n’interfère pas avec les mécanismes HSE existants et
  %
  \item qu'il n'est pas possible de la contourner par le biais d’une
    fonctionnalité antérieure.
  \end{inparaenum}
  %
  Ces risques peuvent être critiques pour la sécurité de la plate-forme comme
  l'illustre l’attaque \emph{\texttt{SENTER} Sandman} présentée en 2015 par Xeno
  Kovah \emph{et al.}\,\cite{kovah2015senter}.
  %
  Les auteurs ont montré qu’il était possible de modifier le contenu de la
  mémoire flash d’un ordinateur en utilisant une extension de l’architecture x86
  nommée Intel TXT\,\cite{intel2015txt}.
  %
  Ils ont en effet remarqué que l’instruction \texttt{SENTER} permettait
  ---~dans sa première version~--- de désactiver une composante essentielle du
  mécanisme de protection en intégrité de la mémoire flash.

  La multiplication des chemins d’attaque potentiels liée au nombre grandissant
  des composants constitue une menace clairement identifiée dans la
  littérature\,\cite{wing2003compositionalattack}.
  %
  Dans le cas particulier des mécanismes HSE, les vulnérabilités successives
  affectant l’architecture
  x86\,\cite{duflot2009smram,wojtczuk2009smram,domas2015sinkhole,kallenberg2015racecondition,kovah2015senter}
  ont été caractérisées par une sévérité très importante, car ces mécanismes
  visent à assurer l’isolation de les couches les plus basses ---~et, par voie de
  conséquence, les plus privilégiées~--- de la pile logicielle.

  \subsection*{Objectifs}

  Dans cette thèse, nous avons cherché à proposer une approche rigoureuse pour
  spécifier et vérifier, par le biais de méthodes formelles, des mécanismes HSE.
  %
  Notre hypothèse de départ est qu’une telle approche bénéficierait à la fois
  aux concepteurs des plate-formes matérielles et aux développeurs de logiciels
  qui s'appuient sur les mécanismes matériels de ces plate-formes.
  %
  Les premiers pourraient vérifier que leurs mécanismes matériels permettent
  effectivement de mettre en œuvres les politiques de sécurité visées.
  %
  Quant aux seconds, ils pourraient profiter de spécifications décrivant sans
  ambiguïtés les exigences auxquelles leurs logiciels doivent se conformer pour
  pouvoir profiter de ces politiques.

  Notre démarche se place à la croisée de deux domaines de vérification.
  %
  La vérification matérielle, d’une part, se concentre généralement sur des
  propriétés qui s’appliquent inconditionnellement à la plate-forme comme à la
  pile logicielle exécutée par cette dernière.
  %
  De nombreux travaux ont ainsi cherché à vérifier des protocoles de cohérence
  de caches\,\cite{stern1995cachecoherence,vijayaraghavan2015modular}, ou la
  correction de implémentation d’un modèle mémoire par un
  processeur\,\cite{choi2017kami}.
  %
  La vérification de logiciels bas niveaux, notamment des systèmes
  d’exploitation ou des hyperviseurs, se repose naturellement sur des modèles de
  la plate-forme matérielle sous-jacente.
  %
  Néanmoins, ces modèles abstraient bien souvent autant que faire se peut la
  complexité de l’architecture matérielle, pour n’en garder que les éléments
  essentiels ---~bien souvent, les mécanismes de pagination et les
  interruptions.
  %
  La conséquence de cet état de fait est que les mécanismes nécessitant une
  configuration logicielle sont moins souvent les sujets de vérification
  formelle.

  Les travaux qui se rapprochent le plus de notre objectif et dont nous avons
  connaissance sont ceux de Jomaa \emph{et al.}\,\cite{jomaa2016mmu}, en lien
  avec le protokernel Pip\,\cite{pipwww}.
  %
  Nous nous inscrivons dans la continuité de cette approche, mais cherchons à
  dégager un formalisme beaucoup plus générique, qui reposerait notamment sur un
  modèle matériel le plus générique possible.
  %
  Cependant, un tel modèle n’est pas sans poser de sérieux défis quant à son
  applicabilité dans un problème de vérification réaliste.
  %
  En effet, la complexité d’un modèle a un impact direct sur la facilité avec
  laquelle on peut l’exploiter.
  %
  Il est donc important de se poser, en amont, les bonnes questions quant à
  l’approche utilisée pour le définir, afin que les efforts nécessaires pour sa
  conception ne soient pas dépensés en vain.
  %
  Plusieurs travaux ont plaidé en faveur d’une approche basée sur un
  raisonnement par composition (\emph{compositional reasoning} en anglais), où
  le système est divisé en un sous-ensemble de composants et la vérification
  axée autour de leurs interactions, pour faire face à ces
  défis\,\cite{garg2010compositional,heyman2012securemodel}.

  \subsection*{Contributions}

  Dans cette thèse, nous présentons deux contributions complémentaires, qui ont
  chacune fait l’objet d’une publication à la conférence \emph{Formal Methods};
  d’abord en 2016\,\cite{letan2016speccert}, puis en
  2018\,\cite{letan2018freespec}.

  \paragraph{Une théorie des mécanismes HSE.}
  %
  Notre première contribution est une théorie des mécanismes HSE, dont
  l’objectif premier est de servir de support à la spécification et à la
  vérification de ces derniers.
  %
  Elle s’articule autour d’une méthodologie divisée en plusieurs étapes.
  %
  L’architecture matérielle est dans un premier temps modélisée sous la forme
  d’un système de transitions étiquetées (\emph{labeled transition system}, en
  anglais, désigné par la suite par l’acronyme LTS).
  %
  Un LTS est traditionnellement caractérisé par un ensemble d’états, un ensemble
  d’étiquettes et une relation de transition.
  %
  Une étiquette est attachée à chaque transition pour permettre de leur donner
  une sémantique particulière.
  %
  Le plus souvent, l’étiquette permet de décrire ce qui a causé la transition.

  Une trace d’un LTS est une séquence potentiellement infinie de transitions et
  décrit un comportement possible du système. Dans le cas qui nous intéresse,
  une trace décrit une exécution d’une pile logicielle par la plate-forme.
  %
  La formalisation de politiques de sécurité pour des systèmes de transitions
  est désormais bien établie.
  %
  Ces politiques peuvent être formellement définies sous la forme de prédicats
  de
  traces\,\cite{lamport1977proving,lamport1985logical,lamport1985logical,alpern1985liveness}
  ou, dans les cas les plus complexes, d’ensemble de
  traces\,\cite{clarkson2010hyperproperties}.

  Dans notre théorie, un modèle matériel se présente sous la forme d’un
  quadruplet \( \langle H, L_S, L_H, \rightarrow \rangle \), où :
  %
  \begin{itemize}
  \item \( H \) est l’ensemble des états que peut prendre le LTS, par exemple la
    valeur des registres des différents composants matériels de la plate-forme
    et le contenu de la DRAM;
  \item \( L_S \) est l’ensemble des étiquettes attachées aux transitions dites
    logicielles, qui sont une conséquence directe et prévisible de l’exécution,
    par la plate-forme matérielle, d’une instruction faisant parti du programme
    d’un logiciel;
  \item \( L_H \) est l’ensemble des étiquettes attachées aux transitions dites
    matérielles;
  \item \( \rightarrow \) est la relation de transition du système.
  \end{itemize}

  En se basant sur ce type de modèle matériel, nous pouvons spécifier un
  mécanisme HSE sous la forme d’un ensemble le logiciels de confiance chargés
  d’implémenter ce mécanisme et d’exigences qu’ils doivent respecter pendant
  leurs exécutions.
  %
  À partir de cette définition, il est possible de dégager un sous-ensemble de
  traces du modèle matériel dans lesquelles les logiciels de confiance ont
  correctement implémenté le mécanisme HSE étudié, en respectant à tout moment
  les deux exigences.
  %
  Par la suite, un mécanisme HSE peut être prouvé correct vis-à-vis de la
  politique de sécurité qu’il cherche à mettre en œuvre, si le sous-ensemble des
  traces du modèle matériel qui le caractérisent satisfont le prédicat
  formalisant la politique.

  Nous avons déroulé notre méthodologie sur un exemple réel, à savoir le
  mécanisme \ac{hse} implémenté par le BIOS des plate-formes x86 pour isoler
  leur exécution de celles des logiciels appartenant au reste de la pile
  logicielle.
  %
  Le BIOS (\emph{Basic Input/Output System}) est un composant logiciel fourni
  par le constructeur de la plate-forme, dont le principal objectif est
  d’initialiser puis de maintenir cette dernière en état de fonctionnement.
  %
  Ce mécanisme \ac{hse} repose sur des fonctionnalités matérielles relativement
  peu connues, notamment un contexte d’exécution particulier des processeurs
  d’Intel ---~au même titre que les \emph{rings}, par exemple~--- nommé le
  \emph{System Management Mode} (SMM).
  %
  Le SMM à ceci d’intéressant qu’il est le contexte d’exécution le plus
  privilégiés de l’architecture x86, si bien qu’une escalade de privilège
  permettant d’exécuter du code malveillant en SMM peut avoir des conséquences
  désastreuses pour la sécurité de la plate-forme.
  %
  Malheureusement, il a été l’objet en dix ans de plusieurs vulnérabilités
  profitant d’incohérences dans les différents composants matériels impliqués
  dans son
  isolation\,\cite{duflot2009smram,wojtczuk2009smram,domas2015sinkhole}.
  %
  Parce que le SMM, ainsi que les autres mécanismes matériels impliqués dans le
  mécanisme HSE qui nous intéresse, ne sont pas pris en compte dans les modèles
  x86 de notre connaissance, nous avons dû en définir un nouveau.
  %
  Cette démarche s’est révélée riche d’enseignement quant aux propriétés qu’un
  modèle générique d’une plate-forme matérielle devrait exhiber pour que ce
  dernier demeure utilisable dans un processus de vérification.
  %
  La modularité, tant de la modélisation que du travail de vérification est, de
  notre point de vue, l’élément clef à privilégier pour pouvoir accompagner au
  mieux un passage à l’échelle.

  \paragraph{Raisonnement par composition pour Coq.}
  %
  Le raisonnement par composition permet de vérifier chaque composant d’un
  système complexe en isolation, en spécifiant des hypothèses sur le comportement
  du reste du système d’une part et en s’assurant que le composant adoptera bien
  un comportement attendu en retour.
  %
  Une fois cette première étape réalisée pour chaque composant, il devient
  possible de raisonner sur la composition de chacun des éléments: si le
  comportement garanti pour un composant \( C_1 \) satisfait les hypothèses de
  raisonnement d’un second composant \( C_2 \), alors il est possible de
  conclure que la composition de \( C_1 \) et de \( C_2 \) garantit le
  comportement de \( C_2 \).

  Notre seconde contribution est une approche permettant la conduite de
  raisonnement par composition sur des composants modélisés grâce à un langage
  purement fonctionnel\,\cite{letan2018freespec} ainsi qu’une implémentation
  nommée FreeSpec de cette approche\,\cite{letan2018freespeccode} dans
  l’assistant de preuve Coq\,\cite{coq}.
  %
  L’originalité de notre contribution est de mettre à profit des paradigmes
  fonctionnels ---~les monades\,\cite{jones2005io}, les effets algébriques et
  les \emph{handlers} d’effets\,\cite{bauer2015effects}~--- pour les appliquer
  au domaine de la vérification de plate-formes matérielles.
  %
  Les modèles écrits dans notre formalisme sont facilement lisibles et soulignent bien les
  connexions entre les différents composants.
  %
  De plus, nous avons implémenté des tactiques dédiées permettant d’automatiser
  en partie l’exploration de ces modèles, ce qui facilite grandement le travail
  de vérification.

  Rien ne cantonne cependant le résultat de ce travail à la vérification de mécanismes HSE.
  Il peut ainsi tout à fait s’appliquer dans le cadre de la
  vérification d’un système purement logiciel, dès lors que ce logiciel puisse être décrit sous la forme de
   plusieurs composants interagissant ensemble.
  %
  La principale limitation de notre approche tient dans les contraintes que nous
  imposons aux interconnexions des composants du système.
  %
  En l’état, FreeSpec ne permet par exemple pas de considérer des graphes qui
  contiennent des cycles ou des \og{} arêtes en avant \fg{} (\emph{forward
    edges} en anglais).
  %
  Cette contrainte est importante, mais dans la pratique, nous avons constaté
  qu’il est possible d’étudier une architecture matérielle à un niveau de détail
  où les composants s’organisent en arbre, selon une hiérarchie par couches
  successives.

  \subsection*{Travaux futures}
  %
  La suite logique de nos travaux seraient d’appliquer le formalisme de FreeSpec
  dans le cadre de la vérification complète d’un mécanisme HSE par le biais de
  notre théorie.
  %
  Par exemple, nous pourrions remplacer le modèle matériel que nous avons
  développé pour spécifier et vérifier le mécanisme HSE implémenté par le BIOS
  et juger l’impact que cela aurait sur le travail de vérification.
  %
  Nous ne doutons par ailleurs pas que les limitations actuelles de l’approche
  que nous proposons avec FreeSpec se heurtera à la
  complexité inhérente aux architectures matérielles.
  %
  Les résoudre nous permettrait d’envisager le passage à l’échelle et donc la
  définition d’un modèle x86 générique, pouvant servir de support à une large
  collection de mécanismes HSE couramment implémentés par les couches basses de
  la pile logicielle.
  %
  Cela poserait très vite la question de la confiance que l’on peut placer dans
  un tel modèle et notamment son adéquation avec ce qui est réellement
  implémenté par le matériel.
  %
  La validation de modèle est un problème de recherche à part entière.
  %
  Dans notre cas, elle est rendue plus complexe par le fait que beaucoup de
  composants matériels sont fortement intégrés à la plate-forme matérielle et
  difficilement accessibles séparément.
  %
  C'est par exemple le cas du \emph{Platform Controller Hub} (PCH), le composant
  matériel qui fait l’intermédiaire entre le processeur et les périphériques les
  moins demandant en vitesse d’accès.
    %
  Ce dernier est maintenant directement intégré directement dans la puce des
  processeurs Intel.
\end{otherlanguage}
