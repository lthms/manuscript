%!TEX root = ../main.tex
\chapter{State of the Art}
\label{chapter:relatedwork}

\PC{tu annonce un état de l'art complet, mais en pratique il est plus limité aux
  transition systems - titre à revoir ?}

\endquote{``\emph{We build our computers the way we build our cities: over time,
    without a plan, on top of ruins.}''

  \hfill\footnotesize --- Ellen Ullman}

\vspace{1cm} \PC{je suis toujours dubitatif sur cette première phrase. Je suis
  surpris que tu ne parle pas de \textbf{modèle} (en introduisant le concept) au
  lieu de système. Le modèle peut correspondre ou non à l'implémentation réelle}
\thomasrk[inline]{Mieux ?}
\GH{Je ne suis pas fan d'utiliser un terme connu par tout le monde dans un sens différent. Donc, ici, je ne suis pas fan d'utiliser "implémentation" dans le sens ou tu l'utilises. Je ne sais pas comment les choses sont présentées dans un (bon) cours sur les méthodes formelles. Mais de ce que je comprends, la vérification formelle à proprement parler est réalisé sur un modèle (mathématique) du système étudié. Ce pose ensuite la question de l'adéquation entre ce modèle et l'implémentation (code exécutable, code exécuté (ce qui suppose que le matériel se comporte comme prévu), circuit logique, VHDL...) Cette adéquation peut-être garantie par extraction automatique du modèle (c'est le cas lorsqu'on dispose d'une sémantique formelle du langage source et que le compilateur a été prouvé) ou au contraire génération automatique de l'implémentation (génération automatique de code dans CoQ), par test (si la spec est exécutable), par insertion de pré/post condition, etc. Quels sont les termes utilisé par ton bouquin en référence?}
%
\noindent
%
The formal verification of a system consists of proving the correctness of an
\emph{implementation} with respect to a
\emph{specification}\,\cite{gupta1992formal}.
%
In this context, the terms ``implementation'' and ``specification'' differ from
their usual meaning in software engineering, and are always subjective to a
given verification problem.
%
An implementation is a description of the system design at any level of
abstraction, \emph{e.g.} a program written in a language whose semantics has
been formally specified, a model used to represent a concrete implementation.
%
A specification is a description of properties targeted by the systems.
%
We distinguish between \emph{functional} specifications, which describe expected
behaviors of their implementations,
% in terms of inputs they handle and outputs they produce
and \emph{non-functional} specifications, which notably include security
policies.
%
The same formal description of a system design can serve both as an
implementation in a first verification problem, and as a functional
specification for a more concrete implementation of the system. \GH{Je ne comprens pas cette phrase mais c'est surement lié à ma première remarque}

In the context of this thesis, we focus on verifying the correctness of
implementations defined in the form of idealized models with respect to
non-functional specifications which characterize security policies. \GH{Cette phrase est longue. Et "idealized model" ne veut pas dire grande chose. Ce qu'il faudrait dire c'est trois choses (donc trois phrases). 1) dans ton cas le modèle (que tu appelles implémentation) est spécifié manuellement à partir de la spécification informelle (papier) fournit par les constructeurs. 2) on s'intéresse à des spécifications non-fonctionnelles. 3) on s'intéresse à la sécurité donc ces spécification forment une politique de sécurité}
%
% In particular, we do not cover \emph{model validation}, that is the
% verification that a model is a correct abstraction of the concrete system it
% supposedly describes.
%
The rest of this Chapter proceeds as follows.
%
First, we introduce in Section~\ref{sec:sota:fm} the key concepts of formal methods which support the formal
verification process by providing tools to construct proofs, and we justify our
choice to use the Coq theorem prover in our experiments.
%
We then explain in Section~\ref{sec:sota:formalisms} how transition systems are used as modeling structures to verify the
correctness of a system with respect to security policies, and where does our
formal definition of \ac{hse} mechanisms stands with respect to previous work.
%
Finaly, we introduce compositional verification approaches, which enable the
``divide and conquer'' strategy to reduce the burden of verifying large and
complex systems, and compare our compositional framework for Coq with existing
solutions (Section~\ref{section:sota:compsec}).

\section{Introduction to Formal Methods}
\label{sec:sota:fm}

In a formal system, formulas (also called propositions) of a formal language
represent statements which can either be true or false.
%
A formal system relies on a set of axioms, that are formulas known to be true,
and inference rules which allow for deriving new formulas.

\subsection{Propositions and Inference Rules.}
%
The statement that a proposition \( P \) is true is commonly denoted by
\( \vdash P \).
%
A proposition \( P \) may also be true on the condition that propositions
\( Q_0, Q_1, ..., \text{ and } Q_n \) are also true.
%
This statement is denoted by \( Q_0, Q_1, ... Q_n \vdash P \).
%
A statement is proven true when it is part of the base of axioms or if we use
inference rules to derive it from already proven statements.
%
By doing so, we construct a \emph{proof}.
%
For example, a well-known inference rule is the \emph{Modus ponens}, which
states that if a proposition \( P \) is true, and the proposition
\( P \Rightarrow Q \) is also true, then the proposition \( Q \) is true, that
is
%
\[
  P\text{, }(P \Rightarrow Q) \vdash Q
\]
%
Using the \emph{Modus ponens}, it becomes possible to construct a proof of
\( \vdash Q \) from a proof of \( \vdash P \) and a proof of
\( \vdash P \Rightarrow Q \).
%
Figure~\ref{fig:sota:inference} gives a list of common inference rules.
%
Conditional statements, once proven, become new inference rules to be used to
construct proofs.

Constructing a proof can quickly become cumbersome as propositions become more
complex.
%
They can be described in the form of carefully worded sentences.
%
Another popular approach is to write them as so-called proof trees.
%
Each node of the tree is a proof step.
%
The children of a given proof obligation are proofs which allows for concluding
about the node judgment, by using an inference rule of the system.
%
By convention, a node and its children are separated by a line labeled with the
name of the inference rule.
%
The leaves of the trees are the statements known to be true, either because they
are part of the system axioms or because they have already been proven before.
%
Therefore, a proof which only consists in applying the \emph{Modus ponen} can be
represented as follows:
%
\begin{prooftree}
  \AxiomC{\( P \)} \AxiomC{\( P \Rightarrow Q \)} \RightLabel{\small {\scshape
      Modus Ponens}} \BinaryInfC{\( Q \)}
\end{prooftree}

\begin{figure}
  \begin{center}
    \begin{tabular}{ll}
      {\scshape Modus ponens} & \( P \Rightarrow Q\text{, }P \vdash Q \) \\
      {\scshape Modus tollens} &
                                 \( P \Rightarrow Q\text{, }\neg Q \vdash \neg P \) \\
      {\scshape Conjunction:} & \( P\text{, }Q \vdash P \wedge Q \) \\
      {\scshape Addition:} & \( P \vdash P \vee Q \) \\
      {\scshape Contradiction:} & \( P, \neg P \vdash Q \)
    \end{tabular}
  \end{center}

  \caption{List of common inference rules}
  \label{fig:sota:inference}
\end{figure}

Constructing a proof and verifying its correctness quickly becomes challenging.
%
Theorem provers such as Coq\,\cite{coq} or
Isabelle/HOL\,\cite{nipkow2002isabelle} provide facilities to write proofs, and
a checker to automatically verify these proofs.
%
Therefore, the quality of a theorem prover is measured by its capacity to reject
ill-formed proofs.
%
Modern theorem provers are built as certified layers around a trusted kernel as
small and simple as possible.
%
The idea is that, as long as the kernel does not contain a bug, advanced
features implemented inside a certified layer cannot introduce inconsistency in
the proof-checking process.
%
This greatly diminishes the risk to have inconsistencies in the proof checker,
although it happened in the
past\,\cite{claret2015falso,griffioen1998comparison}.

Another common approach is to rely on an algorithm whose correctness has been
formally established to handle a well-defined class of problems.
%
This allows for automating the verification process.
%
Once the algorithm has been proven correct, it becomes possible to trust its
output.
%
Abstract interpretation\,\cite{cousot1977absint} and model
checking\,\cite{clarke2018modelc} are two classical instances of this approach.

\subsection{Formal Systems.}
%
Choosing a formal language is a key part of the verification process of a
system.
%
It decides the expressiveness of the statements that can be proven, and the
tools that can be leveraged to construct their proofs.

The most common formal system is the first-order logic\,\cite{smullyan2012fol},
characterized by:
%
\begin{itemize}
\item \emph{Terms}, which represents objects.
\item \emph{Logical operators}, such as conjunction \( \wedge \), disjunction
  \( \vee \), implication \( \Rightarrow \), negation \( \neg \).
\item \emph{Predicates}, that is parameterized formulas that can be true or
  false with respect to the applied terms.
\item \emph{Quantifiers}, such as \( \forall \) (all values) or \( \exists \)
  (there exists a value).
\end{itemize}

First-order logic quantifiers can only be applied on sets of terms (\emph{e.g.}
natural numbers, booleans, states of an airlock system).
%
Higher-order logic\,\cite{leivant1994hol} does not suffer the same limitation,
since they allow quantification over sets of sets, functions and predicates.
%
Hence, it becomes possible to express statements such as \emph{for all sets with
  a total order \( < \), for all pair of values \( \alpha, \beta \), then either
  \( \alpha < \beta \) or \( \beta < \alpha \) or \( \alpha = \beta \)}.
%
Therefore, higher order logic is more expressive than first-order logic, but it
comes at a cost in terms of automation.
%
On the one hand, \emph{interactive} theorem provers (\emph{e.g.}  Coq,
Isabelle/HOL, or more recently Lean\,\cite{de2015lean}) are based on
higher-order logic.
%
On the other hand, \emph{automated} theorem provers (\emph{e.g.}
Vampire\,\cite{riazanov2002vampire}, Z3\,\cite{de2008z3}) are based on
first-order logic.

Finally, modal logic\,\cite{chagrov1997modal} is a formal system which extends
first-order logic with modal operators.
%
Temporal logic forms a family of modal logic systems, including \emph{e.g.}
Linear Temporal Logic (LTL)\,\cite{sistla1985ltl} Computation Logic Tree
(CTL)\,\cite{clarke1981ctl}.
%
Examples of LTL modular operators are \( \square P \) (\( P \) is always true),
\( \Diamond P \) (\( P \) will eventually become true), \( \bigcirc P \)
(\( P \) will be true after the next transition of the system).
%
CTL considers trees of possible futures (by opposition to a \emph{linear}
future).
%
CTL modal operator includes \( \mathbf{A} P \) (\( P \) is true for all possible
futures) and \( \mathbf{E} P \) (there exists at least one path where \( P \)
becomes true).
%
That is, temporal logic operators allow for reasoning about propositions over
time.
%
Temporal logic formulas are commonly verified thanks to model checkers,
\emph{e.g.}  NuSMV\,\cite{cimatti2002nusmv}, SPIN\,\cite{holzmann1997spin} or
TLA+\,\cite{lamport2002tla}.
%
Other approaches can be used to that end ---for instance Gilles Barthe \emph{et
  al.} have encoded \( \square \) and \( \Diamond \) in
Coq\,\cite{barthe2011virtcert1}--- yet model checkers have the benefit of
automation and counter-example generation.

\subsection{Conclusion}

We have implemented our proofs of concepts in the Coq proof assistant.
%
Our choice is motivated by the following reasons:
%
\begin{itemize}
\item {\scshape Gallina}, the specification of Coq, is an expressive formal
  language, that combines both a higher-order logic and a powerful dependent
  type system, while {\scshape Ltac}, the proof tactic language of Coq, enables
  powerful automation features.
  %
  This suits well for specifying security properties to be enforced by HSE
  mechanisms, as they can take many forms.
\item Model checkers and similar automated tools remain subject to the so-called
  \emph{state explosion problem}\,\cite{clarke2012model} despite improvements to
  algorithms.
  %
  State explosion problem imposes modeling compromises, \emph{e.g.}
  verification of cache coherency protocols with a fixed number of cores and
  address spaces limited in size (\emph{e.g.} in \cite{lie2003xom}, the model
  only considers three cache locations).
  %
  Theorem provers allow for modeling \emph{parameterized} systems, so that it
  becomes possible, for example, to construct a proof which holds for an
  arbitrary number of similar components\,\cite{vijayaraghavan2015modular}.
\item The extraction mechanism of Coq allows for transpiling formal
  specifications written in {\scshape Gallina} to OCaml, meaning we can write
  \emph{executable specifications}.
  %
  We could leverage this feature to validate a model against a real
  implementation.
  %
  \TODO{Tu pourrais rajouter une footnote disant que bien qu'il s'agisse d'une
    propriété intéressante qui a notamment motivé ton choix, tu ne l'as pas
    utilisée durant ta thèse}
\end{itemize}

\section{Formal Verification of Transition Systems}
\label{sec:sota:formalisms}

Transition systems have long been used to study the behavior of discrete
systems.
%
Over time, many different definitions of transition system have been proposed,
to address different classes of verification problems.
%
We explain how it is possible to model a system with transition systems
(\ref{subsec:sota:ltsdef}).
%
Then, we focus on characterizing security policies in terms of properties of
transition systems (\ref{subsec:sota:security}).
%
Finally, we describe how the concepts of this section have been used in the
literature in order to verify hardware and software systems
(\ref{subsec:sota:ltsrelated}).
%
We conclude with a summary of our motivations for not reusing an existing x86
model for our experiments (\ref{subsec:sota:ltsconclusion}).

\subsection{Defining Specifications}
\label{subsec:sota:ltsdef}

The formal verification of discrete systems, such as computing systems, commonly
rely on some sort of \emph{transition systems} to describe the behavior of the
subject of study.
%
More precisely, the system is characterized by a set of \emph{states} and by a
set of state transformation, called \emph{transitions}.
%
Transitions occur when the system interacts with its environment (\emph{e.g.} a
hardware circuit receives a clock signal, a hardware controller receives a
message from a bus, an operating system handles a syscall).

Throughout this Chapter, we will use the airlock system in order to act as a
running example to illustrate the introduced definitions, since it is both
simple ---our examples remain of manageable size--- and rich ---our examples do
not feel far-fetched.
%
An airlock system is a device made of two doors, and an intermediary chamber.
%
To get across an airlock system, a user requests the opening of the first door,
enters the chamber, waits for the system to close the first door and open the
second door, and exits the chamber.

\begin{example}[Airlock System]
  \label{example:sota:airlocklts}

  \emph{Labeled} transition systems distinguish between classes of transitions
  \emph{via} the use of labels\,\cite{loiseaux1995lts} (\emph{e.g.} one label
  per syscall).
%
  A labeled transition system is a tuple \( \langle S, L, R \rangle \), such
  that \( S \) is a set of states, \( L \) is a set of labels, and
  \( R \subseteq S \times L \times S \) is the transition relation.
  Figure~\ref{fig:sota:airlock-lts} gives a definition of an idealized airlock
  system, modeled as a labeled transition system.

  \begin{itemize}
  \item A door of the system can be either \( \mathtt{open} \) or
    \( \mathtt{close} \).
    %
    The set of states of the airlock system reflects the Cartesian product of
    the doors states.
  \item A transition is characterized by the opening (\( \mathtt{Open}_i\), with
    \( i \in \{1, 2\} \)) and closing (\( \mathtt{Close}_i \), with
    \( i \in \{1, 2\} \)) of a door of the system.
    %
  \item The model does not allow the simultaneous opening of both doors, as
    stated by the definition of \( R \) which does not contain a transition
    which leads to the state \( (\mathtt{open}, \mathtt{open}) \).
  \end{itemize}
\end{example}

\begin{figure}
  \begin{center}
    % the mathematical definition
    \begin{minipage}[c]{0.55\linewidth}
      \[
        \begin{array}{rcl}
          S & \triangleq & \{ \mathtt{open}, \mathtt{close} \}^2 \\
          L & \triangleq & \{ \mathtt{Open}_1, \mathtt{Close}_1, \mathtt{Open}_2,
                           \mathtt{Close}_2 \} \\
          R & \triangleq & \{ (\mathtt{close}, \mathtt{close}), \mathtt{Open_1},
                           (\mathtt{open}, \mathtt{close}), \\
            & & \ (\mathtt{close}, \mathtt{close}), \mathtt{Open_2},
                (\mathtt{close}, \mathtt{open}), \\
            & & \ (\mathtt{open}, \mathtt{close}), \mathtt{Close_1},
                (\mathtt{close}, \mathtt{close}), \\
            & & \ (\mathtt{close}, \mathtt{open}), \mathtt{Close_2},
                (\mathtt{close}, \mathtt{close}) \}
        \end{array}
      \]
    \end{minipage}
    \hfill
    % a tikzpicture to illustrate the resulting automata
    \begin{minipage}[c]{0.40\linewidth}
      \begin{center}
        \begin{tikzpicture}
          \node [draw, circle split, text width=30pt, text badly centered] (cc)
          {\( \mathtt{close} \) \nodepart{lower} \( \mathtt{close} \)};%
          \node [right=of cc] (x) {};%
          \node [draw, circle split, above=of x, text width=30pt, text badly
          centered] (oc) {\( \mathtt{open} \) \nodepart{lower}
            \( \mathtt{close} \)};%
          \node [draw, circle split, below=of x, text width=30pt, text badly
          centered] (co) {\( \mathtt{close} \) \nodepart{lower}
            \( \mathtt{open} \)};%
          \node [draw, circle split, right=of x, text width=30pt, text badly
          centered] (oo) {\( \mathtt{open} \) \nodepart{lower}
            \( \mathtt{open} \)};%

          \draw [-latex] (cc) edge [bend left] node [xshift=-5pt, left]
          {\( \mathtt{Open}_1 \)} (oc);%
          \draw [-latex] (oc) edge [bend left] node [xshift=5pt, right]
          {\( \mathtt{Close}_1 \)} (cc);%

          \draw [-latex] (cc) edge [bend left] node [xshift=5pt, right]
          {\( \mathtt{Open}_2 \)} (co);%
          \draw [-latex] (co) edge [bend left] node [xshift=-5pt, left]
          {\( \mathtt{Close}_2 \)} (cc);%

        \end{tikzpicture}
      \end{center}
    \end{minipage}

    \caption{A simple airlock system modeled as a labeled transition system}
    \label{fig:sota:airlock-lts}
  \end{center}
\end{figure}

There are many modeling structures for characterizing transition systems, with
each variant better suited to tackle a particular class of systems.
%
For instance, the Kripke structure\,\cite{kripke1971semantical} is a prominent
definition of transition systems used in model checking\,\cite{clarke1999model}.
%
Besides, Petri net\,\cite{peterson1981petri}, process
algebra\,\cite{bergstra1984process} or interface
automata\,\cite{de2001interface} are better suited to model concurrent systems.

\subsection{Specifying Security Policies}
\label{subsec:sota:security}

The verification of a system is \emph{always} relative to certain properties (in
our case, the correct enforcement of a security policy).
%
The definition of a specification of the system to allow for formally reasoning
about its behavior is a key step in the verification process.
%
The formalization of the targeted property is as important, if not more.
%
Indeed, the challenge posed by a potential gap between an implementation and its
specification can be abated by an appropriate correspondence proof.
%
As for the targeted properties, a gap between its characterization using a
particular logic and its semantics should not be understated.
%
For instance, Mathy Vanhoef \emph{et al.} disclosed a critical vulnerability
named KRACK targeting the WPA2 protocol\,\cite{vanhoef2017key}, despite previous
formal verification results (\emph{e.g.} \cite{he2004analysis}).
%
The reason remains simple: KRACK does not violate the security properties proven
in the formal analysis of the protocol, yet these security properties were not
strong enough.

The theory of properties of a transition system is now well understood, with an
intuitive classification of properties, such that:
%
\begin{itemize}
\item \emph{Safety properties}\,\cite{lamport1977proving,lamport1985logical}
  characterize that nothing ``bad'' shall \emph{never} happen.
\item \emph{Liveness properties}\,\cite{lamport1985logical,alpern1985liveness}
  characterize that something ``good'' shall \emph{eventually} happen.
\end{itemize}

We discussed in Subsection~\ref{subsec:usecase:targetedsec} the two classes of
security policies commonly targeted by x86 \ac{hse} mechanisms.
%
An access control policy is a safety property: unauthorized action by a subject
shall never happen.
%
An availability policy is a liveness property: the system shall eventually
satisfy the service

Safety and liveness properties are expressed against the transition systems
\emph{traces}.
%
The most generic definition of a trace of a transition system is a (potentially
infinite) sequence of states generated by successive transitions.
%
Each modeling structure has its own definition of traces, which takes into
account the specificities of the model.
%
For instance, traces labeled transition system will interleave a label between
each state\,\cite{vijayaraghavan2015modular}, to characterize the nature of the
transition which led to the transformation of a state of the sequence with its
successor.
%
Afterwards, we write \( \Sigma(M) \) for the set of traces of a specification
\( M \).

Simplest properties can be defined in terms of predicates on
traces\,\cite{alpern1987recognizing,schneider2000enforceable,basin2013enforceable}.
%
We assume a specification \( M \) whose set of states is \( S \) and whose
traces only contain states.
%
\( M \) is said to be correct with respect to a property modeled as a predicate
on traces \( P \) when
%
\[
  \forall \rho \in \Sigma(M), P(\rho).
\]

On the one hand, safety properties are characterized by an invariant \( \iota \)
on trace elements (\emph{i.e.} on \( S \) in the case of the specification
\( M \)), and
%
\[
  P(\rho) \triangleq \iota(\rho_0) \wedge P(\rho_{[1..]}),
\]
%
where \( \rho_0 \) is the first element of the trace, and \( \rho_{[1..]} \) is
the trace obtained by removing the first element of \( \rho \).
%
On the other hand, liveness properties are characterized by a predicate
\( \eta \) on trace which has to be satisfied for at least a subtrace, that is
%
\[
  P(\rho) \triangleq \exists n > 0, \eta(\rho_{[..n]}) \vee P(\rho_{[1..]}),
\]
%
where \( \rho_{[..n]} \) is the substrace made with the \( n \) first elements
of \( \rho \).

\begin{example}[Airlock Safety and Liveness Properties]
  A typical \emph{safety} property (nothing `bad'' happens) for an airlock
  system is that at least one door shall be close at any time.
  %
  We formalize this property with the invariant \( \iota \), defined as follows:
  %
  \[
    \iota( d_1, d_2) \triangleq d_1 = \mathtt{close} \vee d_2 = \mathtt{close}
  \]
  %
  The specification of the airlock system is defined with a labeled transition
  system.
  %
  Assuming the airlock device is initialized in a correct state (\emph{e.g.}
  both doors are close), we verify this specification is correct with respect to
  the safety property characterized by \( \iota \) by exhibiting a proof that
  \( \iota \) is an invariant with respect to \( R \), that is
  %
  \[
    \forall ((d_1, d_2), l, (d_1', d_2')) \in R, \iota(d_1, d_2) \Rightarrow
    \iota(d_1', d_2')
  \]
  %
  Such a proof is given in Figure~\ref{fig:sota:proofsafety}.

  In addition, we can also prove that both doors of the airlock will
  \emph{eventually} be close.
  %
  We can characterize this liveness property with the predicate \( \eta \) on
  subtraces of one element, such that
  %
  \[
    \eta(d_1, d_2) \triangleq d_1 = \mathtt{close} \wedge d_2 = \mathtt{close}
  \]
  %
  We verify the specification of the airlock system is correct with respect to
  the liveness property characterized by \( \eta \) by exhibiting a proof that
  for each transition of \( R \), one of the states satisfies \( \eta \), that
  is
  %
  \[
    \forall ((d_1, d_2), l, (d_1', d_2')) \in R, \eta(d_1, d_2) \vee \eta(d_1',
    d_2')
  \]

\end{example}

\begin{figure}
  \bigcentering%
  {\footnotesize \AxiomC{} \RightLabel{\footnotesize
      \( \vee,\Rightarrow \){\scshape -Def}} \UnaryInfC{\(
      \begin{array}{l}
        (\mathtt{close} = \mathtt{close} \\
        \vee \mathtt{close} = \mathtt{close}) \\
        \Rightarrow (\mathtt{open} = \mathtt{close} \\
        \qquad \vee \mathtt{close} = \mathtt{close})
      \end{array}
      \)}%
    \RightLabel{\footnotesize \( \iota \){\scshape -Def}} \UnaryInfC{\(
      \begin{array}{l}
        \iota(\mathtt{close}, \mathtt{close}) \\
        \Rightarrow
        \iota(\mathtt{open}, \mathtt{close})
      \end{array} \)}%
    \AxiomC{}
    \RightLabel{\footnotesize \( \vee,\Rightarrow \){\scshape -Def}}
    \UnaryInfC{\(
      \begin{array}{l}
        (\mathtt{close} = \mathtt{close} \\
        \vee \mathtt{close} = \mathtt{close}) \\
        \Rightarrow (\mathtt{close} = \mathtt{close} \\
        \qquad \vee \mathtt{open} = \mathtt{close})
      \end{array}
      \)}%
    \RightLabel{\footnotesize \( \iota \){\scshape -Def}} \UnaryInfC{\(
      \begin{array}{l}
        \iota(\mathtt{close}, \mathtt{close}) \\
        \Rightarrow
        \iota(\mathtt{close}, \mathtt{open})
      \end{array} \)}%
    \AxiomC{}
    \RightLabel{\footnotesize \( \vee,\Rightarrow \){\scshape -Def}}
    \UnaryInfC{\(
      \begin{array}{l}
        (\mathtt{open} = \mathtt{close} \\
        \vee \mathtt{close} = \mathtt{close}) \\
        \Rightarrow (\mathtt{close} = \mathtt{close} \\
        \qquad \vee \mathtt{close} = \mathtt{close})
      \end{array}
      \)}%
    \RightLabel{\footnotesize \( \iota \){\scshape -Def}} \UnaryInfC{\(
      \begin{array}{l}
        \iota(\mathtt{open}, \mathtt{close}) \\
        \Rightarrow
        \iota(\mathtt{close}, \mathtt{close})
      \end{array} \)}%
    \AxiomC{See \( \square \)}
    \RightLabel{\footnotesize \( R \){\scshape -Def}}
    \QuaternaryInfC{\( \forall ((d_1, d_2), l, (d_1', d_2')) \in R, \iota(d_1,
      d_2) \Rightarrow \iota(d_1', d_2') \)}%
    \DisplayProof}

  \vspace{2em}

  {\footnotesize \AxiomC{} \RightLabel{\footnotesize
      \( \vee,\Rightarrow \){\scshape -Def}} \UnaryInfC{\(
      \begin{array}{l}
        (\mathtt{close} = \mathtt{close} \\
        \vee \mathtt{open} = \mathtt{close}) \\
        \Rightarrow (\mathtt{close} = \mathtt{close} \\
        \qquad \vee \mathtt{close} = \mathtt{close})
      \end{array}
      \)}%
    \RightLabel{\footnotesize \( \iota \){\scshape -Def}} \UnaryInfC{\(
      \begin{array}{l}
        \iota(\mathtt{close}, \mathtt{open}) \\
        \Rightarrow
        \iota(\mathtt{close}, \mathtt{close})
      \end{array} \)}%
    \UnaryInfC{\( \square \)}
    \DisplayProof}

  \caption{Airlock system proof of correctness with respect to a safety
    property}
  \label{fig:sota:proofsafety}
\end{figure}

\PC{la figure~\ref{fig:sota:proofsafety} n'est pas très lisible}

Not all security policies can be formalized with predicates on traces.
%
For instance, \emph{noninterference}\,\cite{goguen1982security} is a
confidentiality policy which requires that so-called public inputs handled by a
given system always produce the same output, regardless of concurrent secret
inputs.
%
In this context, considering each trace independently does not make sense.
%
To witness a violation of the security policy requires to compare two traces
together.
%
As a consequence, such policies are characterized by sets of sets of traces;
they are called called \emph{hyperproperties}\,\cite{marr2002hypertheading}.
%
Verifying a system with respect to a hyperproperty is harder in the general
case, but certain hyperproperties, called \( k \)-safety properties, can be
reduced to an invariant.
%
For instance, we assume \( M \) handles inputs which can be either secret or
public.
%
Let \( \equiv \) be the binary relation, such that two states \( s \) and
\( r \) satisfies this relation (\( s \equiv r \)) if and only if they
characterize two instances of the system which have handled the same public
inputs and produced the same visible behavior.
%
Similarly to Gilles Barthe \emph{et al.}\,\cite{barthe2011virtcert1}, we can
prove \( M \) enforces noninterference if for any public input \( l \), then
%
\[
  \forall (s, r) \in S \times S, s \equiv r \Rightarrow (s \xrightarrow{l} s'
  \wedge r \xrightarrow{l} r') \Rightarrow s' \equiv r',
\]
where \( s \xrightarrow{l} s' \) is a transition of the system from a state
\( s \) to a state \( s' \) wherein the input \( l \) is handled.

% \subsection{Adversary Model}
%
% An \emph{adversary model} (or \emph{threat model}) characterizes what
% attackers can and cannot do inside a system.
%%
% Security guarantees shall always be defined with respect to a certain
% adversary model, to define the limits of the system security enforcement
% mechanisms.
%%
% That is, attackers which can leverage capabilities outside of the adversary
% model are likely to defeat the security policies.
%%
% Adversary model is a theoretical tool commonly used in the verification of
% cryptographic protocol, with a well-identified hierarchy of archetypes
% (\emph{e.g.} the Dolev–Yao model allows attackers to intercept, modify,
% generate any messages, but require necessary secret to encrypt and decrypt
% messages\,\thomasrk{ref}).

% \begin{itemize}
% \item Combinatronics of system actions (Pip, VirtCert)
% \item Additional rule to model more powerful attacker (Moat, XOM + bus
%   snooping)
% \end{itemize}

\subsection{Related Work}
\label{subsec:sota:ltsrelated}

Throughout this Section, we have detailed how we can model a target of
verification as a transition system in order to verify its correctness with
respect to security policies specified as predicates on set of model traces.

\paragraph{Model Prerequisites.}
%
Our objective is to apply these concepts in order to verify \ac{hse} mechanisms,
with the hope to uncover compositional attacks.
%
To that end, it is important to consider that
%
\begin{inparaenum}[(1)]
\item \label{needreuse}%
  hardware architectures often allow for implementing several \ac{hse}
  mechanisms, and
  %
\item \label{needreduce}%
  hardware features involved in \ac{hse} mechanisms are not safe by default,
  hence the role played by trusted software components to configure them.
\end{inparaenum}
%
In this thesis, we advocate for defining \ac{hse} mechanisms of a given hardware
architecture against a generic functional specification of this architecture,
rather than relying on ad-hoc models dedicated to specific \ac{hse} mechanisms.
%
Our hypothesis is that a pre-existing model which is comprehensive in terms of
hardware features is a prerequisite to address the threat posed by compositional
attacks.
%
Besides, this means the same model can be reused for several \ac{hse}
mechanisms, reducing the overall verification effort induced by
(\ref{needreuse}).
%
However, (\ref{needreduce}) means models of hardware architectures necessarily
contains traces which are legitimate with respect to the functional
specifications, but violate the security policy.

As a consequence, a formal definition of \ac{hse} mechanisms, like the one we
present in Chapter~\ref{chapter:speccert}, shall allow for identifying the
subset of traces wherein the mechanism is correctly implemented by the trusted
software components.
%
In this context, verifying a \ac{hse} mechanism consists of proving this set of
traces satisfies the predicate which characterizes the targeted security policy.

\paragraph{x86 Models by Intel.}
%
The trustworthiness of a verification result depends to a large extent on the
model.
%
As such, because x86 is a proprietary product, Intel remains in the best
position to produce a detailed, trustworthy model of its architecture.

Intel has integrated formal verification in its processors design process for
more than a decade now.
%
Intel engineers first verified arithmetic operations performed by the
processors\,\cite{harrison2000x86}, then increased the verification scope to
cover the complete execution unit responsible for executing assembly
instruction\,\cite{kaivola2009formalintel}.
%
More recently, the SGX instructions ---which allow system software components to
create and manage enclaves\,\cite{costan2016sgxexplained}--- have been verified
with respect to security properties, rather than functional
specification\,\cite{leslie2015linsgx}.
%
Each project focused on one aspect of the x86 architecture and led to uncover
logic errors and inconsistencies in processor designs.
%
However, models and tools are rarely publicly available, making it difficult to
formally specify and verify \ac{hse} mechanisms.
%
Besides, to the extend of our knowledge, Intel has not advertised about a
``global'' model of its architecture, even though Nachiketh Potlapally, who was
working at Intel at the time, discussed the benefits of such a model in
2011\,\cite{potlapally2011hardwaresecurity}.

\paragraph{x86 ISA Models.}
%
Additionally, several x86 model have been proposed by academic researchers over
the years for purposes of formally reasoning about machine-code programs.
%
To that end, they have modeled the x86 instructions set semantics, often
referred to as x86 ISA.

Probably the most mature projects include RockSalt by Morissett \emph{et
  al.}\,\cite{morrisett2012rocksalt}, Goel \emph{et al.}
framework\,\cite{goel2014x86}, and CompCert x86 assembly
model\,\cite{leroy2012compcert}.
%
These models have in common to focus on the execution of one particular software
component, remain focused on the semantics of the x86 instructions set, and
abstract away as many hardware details as possible to increase the applicability
of the model, \emph{e.g.} are limited to the \ac{dram}.
%
This approach works well when software components use a limited amount of
hardware features, as applications typically do, and to reason about correction
with respect to functional specifications.
%
It has been shown in the past that it is possible to extend them, when their
abstraction is too strong and hides crucial of concrete code execution.
%
For instance, Chen \emph{et al.} have extended the hardware model of CompCertX,
a variant of CompCert used in the development of formally verified
kernels\,\cite{gu2016certikos}, with a notion of input devices and
interrupts\,\cite{chen2018interrupt}.

However, this requires an important level of expertise and understanding of
fairly complicated formal developments, and the primarily focus of these
projects is to reason about software components, not the underlying hardware
architecture.

\paragraph{Ad-hoc x86 Models}
%
Finally, a last category of existing x86 models is ad-hoc models, specially
developed to verify a dedicated system software component.
%
As such, they focus on hardware features used by the target of verification, in
particular the \ac{mmu} and the interrupt handlers, and trust the hardware will
behave as expected.
%
We now give a quick overview of available, open-source ad-hoc models we could
reuse.

The seL4 microkernel is, to date, the most advanced and mature verified
implementation of a microkernel.
%
A C implementation of the kernel is proven correct with respect to a functional
implementation modeled in the Isabelle/HOL theorem prover, and they have show
that this model correctly enforces security policies including information flow
control, integrity and confidentiality\,\cite{klein2009sel4}.
%
In practice, the hardware model focus on \ac{mmu}, cache and interrupt handling,
and in large parts the exact behavior of hardware devices is left
non-deterministic, in order to reduces the assumptions made by the model about
the hardware.
%
In 2016, Jomaa \emph{et al.} have proposed a formal machine-checked proof (in
Coq) of guests isolation by an idealized protokernel based on a
\ac{mmu}\,\cite{jomaa2016mmu}.
%
Similarly to seL4, their hardware model focus on \ac{mmu} and interrupt
handling.
%
In both cases, there is a separation between the software component model and
the hardware model, and it may be possible to extract the hardware model and
reuse it.
%
However, their scope does not include the hardware features involved in the
\ac{hse} mechanisms implemented by the \ac{bios} and described in
Section~\ref{subsec:usecase:hse:smm}.

Between 2011 and 2014, Gilles Barthe \emph{et al.} have worked on an idealized
model of a
hypervisor\,\cite{barthe2011virtcert1,barthe2012virtcert2,barthe2014virtcert3}.
%
This model is defined in terms of states, actions and the semantics of actions
as state trans\-formers.
%
The state mixes information about both hardware components (\ac{cpu} execution
mode, registers, memory content, etc.) and software components (list of guests,
the current active guest, memory mapping for the hypervisor and the guests,
etc.).
%
The set of actions describes the events which can trigger a transformation of
the model states.
%
For instance, it includes various tasks that the hypervisor must carry out, such
as scheduling the guests OS, hypercalls handling, or memory management.
%
Certain actions also witness the execution of guests, for instance when the
currently running OS reads from or writes to memory.
%
The resulting project, called VirtCert and implemented in the Coq theorem
prover, is fairly large, with over 50\,000 lines of code.
%
The verification results focus on various isolation properties, from the most
natural and straightforward (\emph{i.e.} an OS guest cannot write to or read
from a page it does not own) up to non-interference variants including
protection against cache-timing attacks, notoriously harder to reason with.
%
However, the model combines hardware and software components behavior and we
would have to make significant changes prior to using it.

%
% First, they showed that the ideal hypervisor
%%
% \begin{inparaenum}[(1)]
% \item ensures strong isolation between guests (two safety properties and one
%   2-safety property\PC{c'est quoi une ``2-safety property''?}), and
%%
% \item eventually processes every request performed by the guests (one liveness
%   property)\,\cite{barthe2011virtcert1}.
% \end{inparaenum}
%%
% Then, they incorporated the \ac{cpu} cache to their hardware model, and a
% countermeasure for preventing cache-timing attack to their hypervisor model.
%%
% They proved the effectiveness of the countermeasure.
%%
% The authors have shown their ideal hypervisor could prevent such attack, at
% the cost of flushing the cache before each context switch.
%%
% Finally, they have taken their approach a step further, by focusing on
% constant-time cryptography as an effective countermeasure against cache-timing
% attack\,\cite{barthe2014virtcert3}.
%%
% This last work also introduced a static analysis for C program, notably
% implemented as an extension for CompCert, a certified C
% compiler\,\cite{leroy2012compcert}.
%%
% \PC{trop générique/vague: ``une analyse statique ?'' laquelle ? pour faire
% quoi}


\paragraph{Other Hardware Architecture Models.}
%
The x86 architecture is not the only hardware architecture which has been the
object of formal verification projects.
%
We give an overview of the approaches implemented to model and verify them.

The two latest versions of the ARM processors have been formally
specified\,\cite{fox2010armv7,reid2016armv8}.
%
It is important to emphasize that the specification of the ARMv8
architectures\footnote{There are three variants of the ARM Architecture, for as
  many use cases: the A-class architecture provides the necessary features to
  allow an operating system to manage applications, the R-class processors are
  dedicated to real-time systems, and the M-class are used in microcontrollers.}
is the result of an important 5 year effort by ARM Ltd to integrate formal
specification definition to their regular specification process.
%
The formal specification of ARMv8 architectures is written in a dedicated
language called ARM Specification Language, and have been intensively validated
against the ARM internal conformance testsuit.
%
Once the level of trustworthiness of the ASL specifications have been asserted,
they have been able to leverage them to formally verify properties of the
hardware architectures, \emph{e.g.} by compiling them to a subset of Verilog
accepted by commercial Verilog model checkers\,\cite{reid2016end}.
%
The result of these efforts is being made available on the ARM website, in
addition to regular informal specifications\,\cite{arm2018aspec}.
%
This represent an exciting opportunity for research targeting ARM architectures,
and we can only hope that it eventually becomes a standard in the industry.
%
However, it would not make sense to use the ASL language ourselves to specify
the x86 architecture.
%
Indeed, ASL is used to \emph{describe} the architecture in an unambiguous
fashion, but it cannot be used as is to reason about the correctness of this
architecture.

The \ac{xom} microprocessor architecture maintains separate so-called
\emph{compartments} for applications\,\cite{lie2000architectural}.
%
A \ac{xom} \ac{cpu} keeps track of each memory location owner, thanks to a
tagging mechanism, and supposedly prevents an application from accessing a
memory location it does not own.
%
In 2003, David Lie \emph{et al.} have verified the \ac{xom}
architecture\,\cite{lie2003xom} using the Mur$\varphi$ model
checker\,\cite{murphi}.
%
The verification objective of the authors was to prove that the \ac{xom}
architecture fulfills its promise to be tamper-resistant, by forbidding an
attacker to modify the memory location owned by a given application.
%
The verification proceeds as follows:
%
\begin{enumerate}
\item A first specification of the \ac{xom} architecture, called the ``actual
  model'', is defined.
  %
  States of this first model contain different hardware components of a \ac{xom}
  microprocessor, \emph{i.e.} registers, cache, volatile memory, and the
  internal machinery of \ac{xom} to track ownership of memory locations.
  %
  Transitions can be divided into two categories: the normal execution of an
  application by the microprocessor, and active tampering from an adversary
  part, leading the actual model to embed an adversary model.
  %
\item A second specification, called the ``idealized model'', abstracts away the
  memory hierarchy formed by the cache and the volatile memory, and models the
  execution of a single application, without an adversary.
  %
  From this perspective, it encodes the security property.
  %
\item To let Mur\( \varphi \) explore both models simultaneously, the authors
  have manually defined a third model.
  %
  Transitions which describe the execution of an application in the actual model
  also update the idealized model, whereas transitions which describe actions by
  the attacker only affect the actual model.
  %
\item The authors have defined a function which checks if an ``actual state'' is
  equivalent to an ``idealized state'', and let Mur\( \varphi \) verify that the
  state equivalence is an invariant of the third model.
\end{enumerate}
%
In the process of verifying \ac{xom}, the authors have been able to show with
their model that the \ac{xom} architecture was subject to several replay
attacks, and that their countermeasures were effective.

We believe the necessity to manually maintain a merge of two transition systems
reduces its applicability in the long run, as new architecture versions are
released frequently.
%
This is why our formal definition of \ac{hse} mechanism is characterized by a
subset of traces of a model, rather than an additional, idealized model.
%
Besides, the state explosion problem forces the authors to simplify their model,
in order to reduce the state combinatory.
%
From a security perspective, simplifying the model comes with the risk to hide
an attack path.
%
Theorem proving has been shown as a viable alternative to model checking to
overcome the state explosion problem to verify hardware designs, with modern
theorem proving providing facilities to compensate the cost in terms of
automation, \emph{e.g.} Coq provides {\scshape Ltac}, a dedicated language to
automate proof constructions to some extend.
%
For instance, in our proofs in Chapter~\ref{chapter:speccert2}, the cache,
\ac{dram} and VGA memories are of arbitrary sizes.

\subsection{Conclusion}
%
\label{subsec:sota:ltsconclusion}

Transition systems are a popular formalism to reason about the correctness of
computing system, and have been successfully applied to model and reason about
(part of) hardware architectures.
%
In order to specify and verify \ac{hse} mechanisms, we need a comprehensive
model of the related hardware architecture.
%
On the one hand, hardware design verification often focus on properties
transparent to the executed software components (\emph{e.g.} cache
coherency\,\cite{stern1995cachecoherence}, linearizability of SGX
instructions\,\cite{leslie2015linsgx}, hardware-based memory
isolation\,\cite{lie2003xom}).
%
Security gap in interactions of multiple platform components are less subject to
formal verification, due to their increasing
complexity\,\cite{potlapally2011hardwaresecurity}, and we are not aware of any
x86 hardware model comprehensive in terms of hardware components and hardware
features exposed by these components
%
On the other hand, low-level software components such as
seL4\,\cite{klein2009sel4} or CertiKOS\,\cite{gu2016certikos} use the features
exposed by these components, and are verified against \emph{ad hoc} hardware
models, whose scope is driven by their target of verification.
%
As a consequence, we would need to make significant changes prior to using them.

Thus, we face a 'chicken-and-egg' problem: we need a detailed hardware model to
reason about \ac{hse} mechanisms, yet such a model does not exist and requires a
tremendous amount of efforts to be developed, reviewed and validated; which
means we need to be reasonably certain of our modeling choices for those efforts
not going to waste.
%
We decided develop an ad-hoc, simplified x86 hardware model which includes the
necessary hardware components abstracted away in related work.
%
Using this model, we demonstrate in Chapter~\ref{chapter:speccert2} how it is
possible to leverage our formal definition of \ac{hse} mechanisms presented in
Chapter~\ref{chapter:speccert} in order to specify and verify a \ac{hse}
mechanism implemented by the \ac{bios}.
%
This experiment allowed us to learn a lot with regard to desirable properties a
detailed hardware model could have.

\section{Compositional Verification}
\label{section:sota:compsec}

Traditional transition systems, such as the ones presented in the previous
section, impose to model the complete system all at once.
%
This raises important challenges for complex systems.
%
Theorem provers allow for leveraging parameterized models to generalize similar
components (\emph{e.g.} a memory of arbitrary size, an arbitrary number of
core), but this approach does not address the challenges posed by the
composition of many different ---and complex--- components.
%
A comprehensive hardware model in terms of hardware components would undoubtedly
make the construction of the proof unbearable for any non-trivial properties,
yet is a requirement to uncover compositional attacks by the means of formal
methods.
%
% \PC{il manque aussi l'argument de la difficulté à maintenir / faire évoluer le
% modèle si un composant change - or c'est le cas en pratique: de nombreuses
% archis sont similaires, mais avec quelques différences dans les composants}

Compositional verification is a promising (family of) approach(es) to overcome
the challenges posed by the scale of complex and large systems, as it enables to
locally abstract a subpart of the problem \emph{via} a specification.
%
Traditional compositional approaches in the programming language communities
---that the Coq community is part of--- include type
systems%\,\cite{knowles2008compositional}
(specifications for values manipulated by a program), and Hoare logic systems
(specifications for the state updates provoked by the execution of a program).
%
The model checking communities take a different approach, and rather breakdown a
complex system in several stateful and interacting components.
%
The verification process proceeds as follow: components are verified in
isolation, then their composition is verified in order to ensure that local
properties remain true in the complete system.

We believe component-based verification is a particularly well-suited solution
to address the scale of hardware architectures verification, notably because it
mimic their concrete implementations.
%
The rest of the section proceeds as follwos.
%
We first give an overview of compositional verification approaches which can be
leveraged to enable ``divide and conquer'' strategies to reason about a
component-based system (\ref{subsec:sota:compverif}).
%
Then, we describe what modeling structures have been proposed to model
components and their interactions (\ref{subsec:sota:compverif}).
%
We discuss the already existing approaches which have been proposed to enable
component-based verification in theorem prover (\ref{subsec:sota:comprelated}),
and summarize our motivations to propose our own framework for the Coq proof
assistant (\ref{subsec:sota:compconclu}).

\subsection{Compositional Verification Approaches}
\label{subsec:sota:compverif}

Historically, compositional verification\,\cite{peng1998survey} is the answer of
model checking\,\cite{mcmillan1989compositional} and concurrent programs
verification\,\cite{jones1983tentative} communities to the state explosion
problem.
%
It can be divided into two complementary approaches: compositional minimization
and compositional reasoning.

\paragraph{Compositional Minimization.}
%
Components of a larger system are primarily identified by the interface they
expose to their pairs, that is the set of operations they can carry out for the
rest of the system.
%
They also ``hide'' the components they are connected to, and use in order to
handles their input.
%
In this context, two components which expose the same interface can be freely
interchange on the condition that they exhibit an equivalent behavior with
respect to their interface, where the meaning ``equivalent behavior'' is really
specific to each verification problem.
%
For instance, in the context of process algebra, trace equivalence or
bisimulation are two possible definitions of equivalent
behavior\,\cite{fokkink2013pa}.

Using a behavior equivalence approach, it becomes possible to reason about one
component by abstracting away the part of the systems it interacts with
functional specifications.
%
This is referred to as compositional minimization in model checking.
%
The goal is to reduce the complexity of a model, with smaller yet provably
equivalent components, as pictured in Figure~\ref{fig:sota:minicomp}.
%


\begin{figure}
  \begin{minipage}{0.48\linewidth}
    \begin{center}
      \begin{tikzpicture}
        \node [draw, inner sep=10pt, minimum height=40pt] (p) {\( P \)};%
        \node [draw, inner sep=10pt, right=of p, minimum height=40pt] (q)
        {\( Q \)};%

        \draw ([yshift=5pt]p.east) -- ([yshift=5pt]q.west);%
        \draw ([yshift=-5pt]p.east) -- ([yshift=-5pt]q.west);%

        \draw ([yshift=10pt, xshift=-25pt]p.west) -- ([yshift=10pt]p.west);%
        \draw ([xshift=-25pt]p.west) -- (p.west);%
        \draw ([yshift=-10pt, xshift=-25pt]p.west) -- ([yshift=-10pt]p.west);%

        \draw ([yshift=15pt, xshift=25pt]q.east) -- ([yshift=15pt]q.east);%
        \draw ([yshift=5pt, xshift=25pt]q.east) -- ([yshift=5pt]q.east);%
        \draw ([yshift=-5pt, xshift=25pt]q.east) -- ([yshift=-5pt]q.east);%
        \draw ([yshift=-15pt, xshift=25pt]q.east) -- ([yshift=-15pt]q.east);%
      \end{tikzpicture}
    \end{center}
  \end{minipage}
  %
  \begin{minipage}{0.48\linewidth}
    \begin{center}
      \begin{tikzpicture}
        \node [draw, inner sep=10pt, minimum height=40pt] (p) {\( P \)};%
        \node [draw, inner sep=10pt, right=of p] (q) {\( Q' \)};%

        \draw ([yshift=5pt]p.east) -- ([yshift=5pt]q.west);%
        \draw ([yshift=-5pt]p.east) -- ([yshift=-5pt]q.west);%

        \draw ([yshift=10pt, xshift=-25pt]p.west) -- ([yshift=10pt]p.west);%
        \draw ([xshift=-25pt]p.west) -- (p.west);%
        \draw ([yshift=-10pt, xshift=-25pt]p.west) -- ([yshift=-10pt]p.west);%
      \end{tikzpicture}
    \end{center}
  \end{minipage}

  \caption{Illustration of the compositional minimization paradigm}
  \label{fig:sota:minicomp}
\end{figure}

\paragraph{Compositional Reasoning.}
%
Assume-guarantee\,\cite{pnueli1985ag} and
rely-guarantee\,\cite{jones1983tentative} reasoning are about proving a
component \( C \) guarantees the property \( G \) under the assumptions that a
property \( A \) is satisfied, then proving the rest of the system \( S' \)
satisfies \( A \).
%
This allows for concluding about the correctness of the system as a whole, that
is the composition of \( C \) with \( S' \) (denoted by \( C || S' \))
guarantees \( G \).

First assume-guarantee systems have been developed for model checkers, where
assumptions and guarantees were defined in temporal logic formulas.
%
The main obstacle to the adoption of the assume-guarantee paradigm is the
necessity for the user to provide additional inputs to the checker, in addition
to the model and its properties to verify.
%
The \( L* \) learning algorithm, originally proposed by
Augluin\,\cite{angluin1987lstart}, addresses this issue by automatically
generating appropriate assumptions.
%
The \( L* \) algorithm has latterly been adapted to other modeling structure,
for instance for interface automata\,\cite{emmi2008assume}.

\subsection{Compositional Modeling Structures}
\label{subsec:sota:compmod}

We now detail the popular formalisms which have been used to model components in
isolation for the purpose of compositional verification.
%
Because our long-term objective is to apply a component-based modeling approach
to a complex hardware architecture, we focus on their applicability.
%
A component model should be easy to write, read and exploit in the context of a
verification problem.

\paragraph{Concurrent Automata.}
%
I/O automata\,\cite{lynch1988ioautomata} are labeled transition systems which
distinguish between three classes of transitions, modeled with three disjoints
sets of labels:
%
\begin{inparaenum}[(1)]
\item input actions (denoted by \( \mathrm{in}(S) \) for an automaton \( S \)),
  %
\item output actions (denoted by \( \mathrm{out}(S) \)), and
  %
\item internal actions (denoted by \( \mathrm{int}(S) \)).
  %
\end{inparaenum}
%
They form the signature of a given automaton (denoted by \( \mathrm{act}(S) \)).
%
Its transition relation \( R(S) \) is a subset of
\( \mathrm{state}(S) \times \mathrm{act}(S) \times \mathrm{state}(S) \), where
\( \mathrm{state}(S) \) is the set of states of \( S \).
%
I/O automata are expected to be \emph{input enabled}, that is it cannot delay
the treatment of its inputs.
%
This translates as follows: for every state \( p \) and input actions \( \pi \),
it exists a state \( q \) such that \( (p, \pi, q) \in R(S) \).
%
Composition of I/O automaton is achieved \emph{via} input and output actions.
%
More precisely, when one automaton performs an output action \( \pi \) during a
transition, all automata having \( \pi \) as input action perform \( \pi \)
simultaneously.

Although by definition, I/O automata are required to be input enabled, in
practice the component they model will correctly behave (for a certain
definition of ``correct'') only if certain requirements are met regarding the
inputs they have to handle.
%
Interface automata\,\cite{de2001interfaceautomata} are similar to I/O automata,
except they do not require to be input enabled.
%
As a consequence, the definition of an interface automaton captures its
requirements over the rest of the system.

Another popular modeling structure is the Moore Machine, as they play a key role
in compositional model checking\,\cite{mcmillan1989compositional}, because they
can be translated into Kripke structures.

\begin{example}[Airlock System as Interface Automata]
  \label{example:sota:airlockinterface}

  In order to illustrate how a system can be broken down into small components,
  we take once again the example of the airlock system.
  %
  In this context, the most obvious component is the door.
  %
  A door has two states: it can be either open or close.
  %
  It takes two input actions: \( \mathtt{Open} \) (the action to open the door)
  and \( \mathtt{Close} \) (the action to close the door).
  %
  It does not have any output action, which means a door does not interact
  actively with the rest of the system.
  %
  One possible specification for a door can be the following interface
  automaton:

  \begin{center}
    \begin{tikzpicture}
      \node [draw, circle] (o) {1};%
      \node [draw, circle, right=of o] (c) {2};%

      \draw [-latex] (o) edge [bend left] node [above] {\( \mathtt{Open}? \)}
      (c);%
      \draw [-latex] (c) edge [bend left] node [below] {\( \mathtt{Close}? \)}
      (o);%
      \draw [-latex] (c) edge [loop right] node {\( \mathtt{Open}? \)} (c);%
      \draw [-latex] (o) edge [loop left] node {\( \mathtt{Close}? \)} (o);%

      \node [draw, fit=(o) (c), inner sep=20pt, text width=130pt, text badly
      centered] (d1) {};%

      \node [yshift=10pt, left=35pt of d1.west] (open) {};%
      \draw [-latex] (open) to node [above] {\( \mathtt{Open} \)}
      ([yshift=10pt]d1.west);%
      \node [yshift=-10pt, left=35pt of d1] (close) {};%
      \draw [-latex] (close) to node [above] {\( \mathtt{Close} \)}
      ([yshift=-10pt]d1.west);%
    \end{tikzpicture}
  \end{center}

  In addition to two doors, an airlock system needs a controller, whose purpose
  is to handle users' requests and open and close doors in consequence.
  %
  We consider a slightly different situation than the specification given in
  Example~\ref{example:sota:airlocklts}.
  %
  Here, there are only two commands, modeled with two input actions:
  \( \mathtt{Req}_1 \) (one user wants the first door to be opened) and
  \( \mathtt{Req}_2 \) (one user wants the second door to be opened).
  %
  The controller does not embed the states of the doors, but has four output
  actions, two per doors (\( \mathtt{Open}_i \) and \( \mathtt{Close}_i\), for
  \( i \in \{1, 2\}\)).
  %
  We propose the following interface automaton:

  \begin{center}
    \begin{tikzpicture}
      \node [draw, circle] (s1) {2};%
      \node [draw, circle, right=50pt of s1] (s2) {3};%
      \node [draw, circle, right=50pt of s2] (s3) {4};%

      \draw [-latex] (s1) to node [above] {\( \mathtt{Close2}! \)} (s2);%
      \draw [-latex] (s2) to node [above] {\( \mathtt{Open1}! \)} (s3);%

      \node [draw, circle, below=60pt of s3] (s4) {5};%
      \node [draw, circle, left=50pt of s4] (s5) {6};%
      \node [draw, circle, left=50pt of s5] (s6) {7};%

      \draw [-latex] (s4) to node [below] {\( \mathtt{Close1}! \)} (s5);%
      \draw [-latex] (s5) to node [below] {\( \mathtt{Open2}! \)} (s6);%

      \node [draw, circle, below=21pt of s2] (s7) {1};%

      \draw [-latex] (s7) to node [left,yshift=-5pt] {\( \mathtt{Req1}? \)}
      (s1);%
      \draw [-latex] (s7) to node [right,yshift=5pt] {\( \mathtt{Req2}? \)}
      (s4);%

      \draw [-latex] (s3) edge [bend left] node [right] {\( \mathtt{Req2}? \)}
      (s4);%
      \draw [-latex] (s6) edge [bend left] node [left] {\( \mathtt{Req1}? \)}
      (s1);%

      \node [draw, fit=(s1) (s4), inner sep=20pt, text width=220pt] (ctrl) {};%

      \node [yshift=20pt, left=35pt of ctrl] (in_open1) {};%
      \draw [-latex] (in_open1) to node [above] {\( \mathtt{Req}_1 \)}
      ([yshift=20pt]ctrl.west);%
      \node [yshift=-20pt, left=35pt of ctrl] (in_open2) {};%
      \draw [-latex] (in_open2) to node [above] {\( \mathtt{Req}_2 \)}
      ([yshift=-20pt]ctrl.west);%

      \node [yshift=45pt, right=35pt of ctrl] (out_open1) {};%
      \draw [latex-] (out_open1) to node [above] {\( \mathtt{Open}_1 \)}
      ([yshift=45pt]ctrl.east);%
      \node [yshift=15pt, right=35pt of ctrl] (out_close1) {};%
      \draw [latex-] (out_close1) to node [above] {\( \mathtt{Close}_1 \)}
      ([yshift=15pt]ctrl.east);%

      \node [yshift=-15pt, right=35pt of ctrl] (out_open2) {};%
      \draw [latex-] (out_open2) to node [above] {\( \mathtt{Open}_2 \)}
      ([yshift=-15pt]ctrl.east);%
      \node [yshift=-45pt, right=35pt of ctrl] (out_close2) {};%
      \draw [latex-] (out_close2) to node [above] {\( \mathtt{Close}_2 \)}
      ([yshift=-45pt]ctrl.east);%

      \draw (s3) edge [loop right] node {\( \mathtt{Req1}? \)} (s3);%
      \draw (s6) edge [loop left] node {\( \mathtt{Req2}? \)} (s6);%
    \end{tikzpicture}
  \end{center}
\end{example}

Labels of interface automata allows for composing together several transition
systems by synchronizing their shared transitions.
%
However, the manual definition of these components to specify sequences of input
and output transitions can quickly become cumbersome, as we our experiences with
the definition of the airlock system controller in
Example~\ref{example:sota:airlockinterface}.
%
Besides, interactions between components are often motivated by the need to
exchange data, \emph{e.g.} the processor interacts with the \ac{pch} to read
from or write to peripherals memories.

\paragraph{Process Algebra.}
%
Another popular family of approaches which enable compositional modeling of
complex systems is process algebra.
%
Process algebra has been developed to reason about programs executed in
parallel.
%
It is a formal system to describe interacting systems, together with a proof
system for verifying them.
%
In process algebra such as Calculus of Communication
Systems\,\cite{milner1980ccs} or the Communicating Sequential
Processes\,\cite{hoare1978csp}, concurrent threads run in parallel, and
synchronization is achieved by sending and waiting for messages, as specified by
a dedicated language.

\begin{example}[Airlock System in \( \pi \)-calculus]
  \label{example:sota:airlockprocess}

  We now try to give a specification of our airlock system using a process
  algebra called \( \pi \)-calculus.
  %
  Once again, we consider three components: two doors and a controller.
  %
  Our objective is to write a specification equivalent to our interface automata
  (although we do not provide a proof of that equivalence).

  We have used the following \( \pi \)-calculus construction to specify the
  airlock system:
  %
  \begin{itemize}
  \item \( c(x). P\) means receiving a value from the channel \( c \), bounding
    this value to the name \( x \), then executing the process \( P \).
  \item \( \bar{c} \langle x \rangle . P \) means sending the value bounded to
    name \( x \) through the channel \( c \), then executing the process
    \( P \).
  \item \( [x = \mathrm{OPEN}] . P \) is a guard, that is \( P \) is executed if
    \( x \) is equal to the value \( \mathrm{OPEN} \).
  \item \( P + Q \) is the nondeterministic choice operator, we use it here in
    conjunction with guards to implement a \texttt{if-then-else} construct. That
    is, considering the process
    %
    \[
      c(x) . ([x = 1] . P + [x = 2] . Q + R)
    \]
    %
    If the value received from \( c \) is \( 1 \), \( P \) is executed.
    %
    If it is \( 2 \), then \( Q \) is executed. Otherwise \( R \) is executed.
  \item \( \nu c. P \) means a new channel \( c \) is created, and available for
    \( P \) to use it.
  \item \( P || Q \) is the parallel execution of \( P \) and \( Q \).
  \end{itemize}

  \[
    \begin{array}{rcl}
      \mathrm{CloseDoor}(c)
      & \triangleq
      & c(x) . \ ([x = \mathrm{OPEN}] . \mathrm{OpenDoor}(c) \\
      &
      & \qquad + \mathrm{CloseDoor}(c)) \\
      & & \\
      \mathrm{OpenDoor}(c)
      & \triangleq
      & c(x) . \ ([x = \mathrm{CLOSE}] . \mathrm{CloseDoor}(c) \\
      &
      & \qquad + \mathrm{OpenDoor}(c)) \\
      & & \\
      \mathrm{Controller}(c, d_1, d_2)
      & \triangleq
      & c(x). \ \ [x = \mathrm{OPEN}_1] . \bar{d_2} \langle
        \mathrm{CLOSE} \rangle . \bar{d_1} \langle \mathrm{OPEN} \rangle . \mathrm{Controller}(c, d_1, d_2) \\
      &
      & \qquad + [x = \mathrm{OPEN}_2] . \bar{d_1} \langle
        \mathrm{CLOSE} \rangle . \bar{d_2} \langle \mathrm{OPEN} \rangle. \mathrm{Controller}(c, d_1, d_2) \\
      & & \\
      \mathrm{System}
      & \triangleq
      & \nu c. \nu d_1. \nu d_2. (\mathrm{Controller}(c, d_1, d_2) \\
      &
      & \quad\qquad\qquad ||\ \mathrm{CloseDoor}(d_1) \\
      &
      & \quad\qquad\qquad ||\ \mathrm{CloseDoor}(d_2))
    \end{array}
  \]

  The system, modeled with the process \( \mathrm{System} \) creates the
  channels used by its components to communicate, then starts their concurrent
  executions.
  %
  A door is either open or close, and we model this with two mutually recursive
  processes \( \mathrm{CloseDoor} \) and \( \mathrm{OpenDoor} \).
  %
  They take one channel \( c \) as an argument, then wait for new inputs coming
  from \( c \).
  %
  A controller is a recursive process which takes three channels \( c \),
  \( d_1 \) and \( d_2 \) as arguments.
  %
  It waits for new requests coming from \( c \).
  %
  When it receives a new request to open the first (resp. second) door, it first
  close the second (resp. first) door, using the channel \( d_2 \) (resp.
  \( d_1 \)).
  %
  Then, it opens the first (resp. second) door, using the channel \( d_1 \)
  (resp. \( d_2 \)).
\end{example}

Process algebras such as \( \pi \)-calculus are well suited formalisms for
describing component interactions.
%
However, their applicability is limited by the low-level description of the used
languages and their minimalistic semantic.

\paragraph{Programs with Effects.}
%
The relation between one component and the rest of the system it is connected to
is reminiscent of the programming language to model and verify large programs
with side effects realized by an outer environment (\emph{e.g.} an operating
system).
%
Modeling side effects in pure programming language, such as Haskell or {\scshape
  Gallina}, is usually achieved thanks to
Monads\,\cite{wadler1990comprehending,jones2005io}, \emph{e.g.} the State Monad
to seamlessly write computations which manipulate a global state, or the List
Monad to abstract non-determinism,
%
However, Monads have a reputation not to compose very
well\,\cite{hyland2006combining}, despite construction such as Monad
Transformers\,\cite{liang1995mtl}.
%
Algebraic effects and effect handlers are a generic approach overcoming this
challenge, with many different implementations (\emph{e.g.}
Eff\,\cite{bauer2015effects}, {\scshape Idris}\,\cite{brady2013idris}, or
Haskell\,\cite{kiselyov2013extensible}).
%
In this context, \emph{effects} are characterized by sets of operations,
\emph{e.g.} \texttt{get} and \texttt{set} for global state manipulation.
%
These operations are expected to produce a value whose type is known by
definition.
%
For instance, \texttt{get} produces a value of the same type as the global
state.

On the one hand, programs with effects are pure computations which compose
effects results together.
%
On the other hand, handlers model the environment which realizes these effects,
computes their results, and manipulates programs control flows.
%
One possible representation of the relation between programs with effects and
effect handlers is by the means of coroutines\,\cite{kiselyov2013extensible}.
%
When a program with effects requires the result of a given effect, it sends a
request to the handler, and waits for the result.

\subsection{Related Work}
\label{subsec:sota:comprelated}

In Chapter~\ref{chapter:freespec}, we show how it is possible to model a
hierarchy of components which interacts \emph{via} their interfaces by means of
programs with effects and effect handlers.
%
The main idea is that a component can be modeled as a handler for the interface
it exposes, in terms of program with effects of the interface it uses.
%
To our surprise, we did not find any generic project to write and verify
programs with effects and effect handlers written for {\scshape Gallina}, but
several related approaches exist.

\paragraph{{\scshape Kami}.}
%
In 2017, Joonwon Choi \emph{et al.} have released {\scshape
  Kami}\,\cite{choi2017kami}, a framework for Coq to model and verify hardware
components.
%
The resulting hardware models can also be converted into
BlueSpec\,\cite{nikhil2004bluespec} programs, which can be compiled to Verilog,
a hardware description language.
%
This means it is possible to turn a {\scshape Kami} model into a concrete
implementation.

In in {\scshape Kami}, a model \( M \) is a particular labeled transition
system, whose transitions are of the form
%
\[
  p \xrightarrow[a]{i} (v, q)
\]
%
where \( p \) is the state of the component before an operation \( i \) which is
called by another component, \( a \) is a sequence of actions performed by the
component in order to compute the result of \( i \), \( v \) is the result of
\( i \), for the caller to use, and \( q \) is the modified state of the
component.
%
Actions, in this context, are either local manipulation of the component's state
or calls of operations.

To reason about \( M \) with respect to a specification \( M_S \), {\scshape
  Kami} introduces a refinement relation \( \sqsubseteq \).
%
A module \( M \) refines a module \( M_S \) (\emph{i.e.} \( M \) is an
implementation of \( M_S \)) if any traces of \( M \) can also be produced by
\( M_S \).
%
A component \( M \) can be composed with another component \( M' \) to form a
larger component \( M + M' \), for instance when the operations exposed by
\( M' \) are used by \( M \).
%
The authors introduced another ``modular'' refinement property, whose simplest
expression could be
%
\[
  M \sqsubseteq N \wedge R \sqsubseteq S \Rightarrow M + R \sqsubseteq N + S
\]
%
This approach is related to the compositional minimization, where one component
is reduced to a simpler, more abstract model, to reduce the complexity of the
proofs.

\paragraph{\texttt{Coq.io}.}
%
The \texttt{Coq.io} framework, developed and released by Guillaume Claret
\emph{et al.}\,\cite{claret2015coqiowww}, have been originally proposed in order
to allow the definition and verification of interactive programs in Coq by the
means of use cases\,\cite{claret2015coqio}.
%
Such programs are defined in a dedicated monad, with side effects (\emph{e.g.}
system calls to read from or write to a file) axiomatized as monadic operations
of this monad.
%
The proofs rely on scenarios which determine how an environment (\emph{e.g.} an
operating system) would react to the program requests.
%
The verification goal is to verify that, under the hypothesis that the
environment is correct with respect to a scenario, then a program with effects
is also correct with respect to the trace of side effects operations it
produces.

The approach used by \texttt{Coq.io} paves the road towards a compositional
approach reasoning.
%
A program with effects can be verified, but the verification of the composition
of this program with its environment is out of the scope of the project.

\paragraph{Interface Specifications.}
%
Logic of Program, but to the best of our knowledge, no implementation.
%
\thomasrk[inline]{TODO}

\subsection{Conclusion}
\label{subsec:sota:compconclu}

Our formal definition of \ac{hse} mechanism assumes there exists a formal model
of the hardware architecture.
%
The development of our proof of concept, implemented in
Coq\,\cite{letan2016speccertcode}, convinced us that such a model cannot be
monolithic.
%
Otherwise, it would make the proof burden hard to bear.
%
Compositional verification is a very important domain research for the
verification of large, realistic model.
%
However, it is a discipline mostly dominated by automated verification, like
model checking for instance.

With FreeSpec\,\cite{letan2018freespeccode}, we aim to provide a generic
compositional reasoning framework for Coq.
%
As detailed in Chapter~\ref{chapter:freespec}, FreeSpec leverages programs with
effects and effect handlers to model component-based system, and introduces
so-called abstract specification in order to enable a verification methodology
inspired by the assume-guarantee paradigm.
%
In doing so, FreeSpec is at the intersection of several research domains.
%
Our work is in the continuity of \texttt{Coq.io}\,\cite{claret2015coqio}, but
our abstract specifications are more generic and expressive than their
scenarios.
%
We have shown how programs with effects and effects handlers can be used to
modularly verify a complex system made of interconnected components, \emph{via}
the notion of interface as the rendezvous point.
%
The principle of our approach is similar to what {\scshape Kami}
achieves\,\cite{choi2017kami}, but is not hardware specific.
%
In addition, our abstract specifications allow for compositional verification,
while {\scshape Kami} focuses on refinement proofs.
%
To the best of our knowledge, the closest related research is the work of Deepak
Garg \emph{et al.}\,\cite{garg2010compositional}.
%
Probably the main difference between our two approaches is that they rely on a
dedicated programming language, while our programs with effects and effect
handlers are regular {\scshape Gallina} development.
%
As a consequence, we can use the full interactive proof development environment
of Coq in order to write and validate our proofs.

\section{Conclusion}
\label{sec:sota:conclusion}
