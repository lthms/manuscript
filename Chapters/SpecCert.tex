\chapter{SpecCert}

Modern hardware architectures have grown in complexity. They now are made of
numerous devices which expose multiple programmable functions. In this article,
we identify a class of security enforcement mechanisms we call Hardware-based
Security Enforcement (HSE) such that a set of software components configures the
hardware in a way which prevents the other software components to break a
security policy. For instance, when an operating system uses the ring levels and
memory paging features of x86 microprocessors to isolate the userland
applications, it implements a HSE mechanism. A HSE mechanism is sound when it
succeeds in enforcing a security policy. It requires (1) the hardware functions
to provide the expected properties and (2) the software components to make a
correct use of these hardware functions. In practice, both requirements are hard
to meet.

First, hardware architectures comprise multiple interconnected devices which
interact together. From a security perspective, it implies considering the
devices both individually and as a whole. Hardware functions are not immune to
security vulnerabilities. For instance, early versions of the \texttt{sinit}
instruction implementation of the Intel TXT technology\,\cite{intel2015txt}
allowed an attacker to perform a privilege
escalation\,\cite{wojtczuk2011txtbug}. The legitimate use of a hardware
mechanism can also break the security promised by another. For instance, until
2008, the x86 cache allowed to circumvent an access control mechanism exposed by
the memory controller\,\cite{wojtczuk2009smram,duflot2009smram}.  Secondly,
hardware architectures have grown in complexity and, as a consequence, HSE
mechanisms too. There are many examples of security vulnerabilities which are
the consequence of an incorrect HSE mechanism
implementation\,\cite{kallenberg2014failure,bulygin2014bios,intel2014chipsec}.

In this paper, we introduce SpecCert, a framework for specifying and verifying
HSE mechanisms against hardware architecture models. SpecCert relies on a
three-steps methodology. First, we model the hardware architecture
specifications. Then we specify the software requirements that must be satisfied
by the trusted software components which implement the HSE mechanism. Finally,
we prove that the HSE mechanism is sound under the assumption that the software
components complies to the specified requirements. This implies the hardware
involved in the HSE mechanism indeed provides the security properties they
promise. We believe this approach to be beneficial to both hardware designers
and software developers.  The former can verify their hardware mechanism
assumptions and the latter can get a formal specification to implement the HSE
mechanism.

In Section \ref{sec:speccert:framework}, we give a formal definition of the
SpecCert formalism. In Section \ref{sec:speccert:hardware}, we define a model of
x86-based hardware architectures to verify HSE mechanisms targeting software
isolation policies using publicly available Intel specifications. In Section
\ref{sec:speccert:smm}, we verify the soundness of the HSE mechanism implemented in many
x86 computer firmware codes to isolate the code executed while the CPU is in
System Management Mode (SMM), a highly privileged execution mode of x86
microprocessors. Our model and proofs have been implemented using Coq, a proof
assistant system and have been released as an open source
software\,\footnote{Which can can be found at:
\url{https://github.com/lethom/speccert}}. We discuss our results in Section
\ref{sec:speccert:discuss}.
% TODO
%, some related works in Section \ref{sec:related} and conclude
%in Section \ref{sec:conclu}.

\section{SpecCert Formalism} \label{sec:speccert:framework}

In SpecCert, we model the hardware architecture and its features with a set of
states $\set{H}$, a set of events $\set{E}$ and a Computing Platform $\Sigma$
which defines a semantics of events as state-transformers. Hence, the execution
of a set of software components by a hardware architecture is a sequence of
state-transformations (denoted $\transitionLTS{Ez}{h}{ev}{h'}$) in this model.
In this paper, we consider exclusively Execution Monitoring (EM) enforceable
security policies\,\cite{schneider2000enforceable,basin2013enforceable} that are
security policies which can be enforced by monitoring the software execution. As
a consequence, we model a security policy with a predicate $P$ on sequences of
state-transformations.  Finally, we model a HSE mechanism $\Delta$ with a set of
requirements on states to characterize safe hardware configurations and a set of
requirements on state-transformations for trusted software components to
preserve the state requirements through software execution. A HSE mechanism is
sound when every sequence of state transformations which satisfies these
requirements also satisfies the security policy predicate.

\subsection{Computing Platforms} \label{subsec:speccert:computing}

We now dive more deeply into the SpecCert formalism and give a formal definition
of the Computing Platform. We model a hardware architecture which executes
several software components using states, events and a semantics of events as
states-transformers.

The state of a hardware architecture models the configuration of its devices at
a given time. This configuration may change over time with respect to the
hardware specifications and comprises any relevant data such as registers
values, inner memory contents, etc. A hardware architecture state update is
triggered by some events. We distinguish two classes of events: the software
events which are direct and forseeable side-effects of the execution of an
instruction and the hardware events which are not. The execution of an
instruction can be broken down into a sequence of software events.

For instance, to execute the x86 instruction\,\footnote{Written in AT\&T syntax
here.} \texttt{mov (\%ecx),\%eax}, a x86 CPU:
  \begin{compactitem}
  \item reads the content of the register \texttt{ecx} as an address
  \item reads the main memory at this address
  \item writes this content into the register \text{eax}
  \item updates the register \texttt{eip} with the address of the next
    instruction to execute
  \end{compactitem}

We model this sequence of actions as four software events which trigger four
state updates. Note that if the content of the \texttt{ecx} register is not a
valid address, the scenario is different. In such a case, the read access to the
main memory fails and an interrupt is raised. This second scenario is modeled
with another sequence of events which involved a hardware event \emph{i.e.} the
interrupt.

The semantics of events as state-transformers is specified using preconditions
and postconditions. Preconditions specify the state requirements which are
necessary for an event to be observed. Postconditions specify the consequences
of an event on the hardware architecture state.

\begin{definition}[Computing System]
  Given $\set{H}$ a set of hardware architecture states and $\set{E}$ a set of
  events, a Computing Platform $\shortLTS{Ez}$ is a pair of $(pre\-condition,
  post\-condition)$ where $pre\-condition$ is a predicate on $\set{H} \times
  \set{E}$ and $post\-condition$ is a predicate on $\set{H} \times \set{E} \times
  \set{H}$. $\shortLTS{Ez}$ defines a semantics of events as state-transformers
  such as
  \begin{prooftree}
    \AxiomC{$precondition(h,ev)$}
    \AxiomC{$postcondition(h,ev,h')$}
    \BinaryInfC{\transitionLTS{Ez}{h}{ev}{h'}}
  \end{prooftree}

  \transitionLTS{Ez}{h}{ev}{h'} is called a state-transformation of\,
  $\shortLTS{Ez}$.
\end{definition}

%\begin{definition}[Software Stack]
%  Given $\set{S}$ a set of Software Components, $\set{H}$ a Hardware
%  Architecture and $\set{P}$ the Processor Unit of $\set{H}$, $\zeta : \set{P}
%  \rightarrow \set{S}$ is a Software Stack which dedicates states of $\set{P}$
%  to each Software Component of $\set{S}$.
%\end{definition}

\subsection{Security Policies}

Given $\set{H}$ a set of states of a hardware architecture, $\set{E}$ a set of
events, $\shortLTS{Ez}$ a Computing Platform and $\set{S}$ a set of software
components being executed by the hardware architecture, a particular execution
of a set of software components is modeled with a sequence of
state-transformations we call a run of $\shortLTS{Ez}$.

\begin{definition}[Run]
  A run of the Computing Platform $\shortLTS{Ez}$ is a sequence of
  state-transformations of\, $\shortLTS{Ez}$ such that for two consecutive
  transformations, the resulting state of the first is the initial state of the
  next. We denote $\pathesLTS{Ez}$ the set of runs of the Computing Platform
  $\shortLTS{Ez}$ and $init(\rho)$ the initial state of a run $\rho$.
\end{definition}

We consider EM-enforceable security policies\,\cite{schneider,schneider2}
specified with predicates on runs. A run is said to be secure according to a
security policy when it satisfies the predicate specifying this policy.

In this paper, we focus on a class of security policies we call software
execution isolation policies. Such a policy prevents a set of untrusted software
components to tamper with the execution of another set of so-called trusted
software components. We consider that a software component tampers with the
execution of another when it is able to make the latter execute an instruction
of its choice.

In practice, a subset of states of the hardware architecture is dedicated to
each software component. For instance, the x86 CPU has a feature called
protection rings where each ring can be seen as an execution mode dedicated to a
software component. Hence, the ring 0 is dedicated to the operating system
whereas the userland applications are executed when the CPU is in ring 3. In
SpecCert, we take advantage of this CPU state sharing to infer which software
component is currently executed from a hardware architecture state. For the
following definitions, we assume the hardware architecture contains only one
CPU.

\begin{definition}[Hardware-Software Mapping]
  \label{def:hardsoftmap}
  A hardware-software mapping $con\-text : \set{H} \rightarrow \set{S}$ is a
  function which takes a hardware state and returns the software component
  currently executed.
\end{definition}

Dealing with multi-core architectures would require additional efforts and
notations. One possible solution could be to define an identifier per core and
to use this identifier in addition to the current hardware state to deduce the
software component currently executed by the corresponding core. However, this
is out of the scope of this article.

% okay so know we need to explain the event-software mapping. it is not that
% hard, but the issue is that it is more or less related to the memory
% ownership, so maybe i need to introduced it here.
We now introduce the concept of \textit{memory location ownership}. A memory
location within a hardware architecture is a container which is able to store
data used by a software component \emph{e.g.}\,a general-purpose register of a
CPU, a DRAM memory cell, etc. We say that a Computing Platform tracks the memory
location ownership if the hardware architecture states maps each memory location
with a software component called its \emph{owner}, and the Computing Platform
semantics updates this mapping through state-transformations. A software
component becomes the new owner of a memory location when it overrides its
content during a state-transformation. By extension, we say a software component
owns some data when it owns the memory location in which these data are stored.

With this mapping, it becomes possible to determine the owner of an instruction
fetched by the CPU in order to be decoded and executed.

\begin{definition}[Event-Software Mapping]
  \label{def:evsoft}
  An event-software mapping $fet\-ched: \set{H} \times \set{E} \rightarrow
  \mathcal{P}(\set{S})$ is a function which takes an initial hardware state and
  an event and returns the set of the fetched instructions owners during this
  state-transformation.
\end{definition}

Hence, $s \in fetched(h, ev)$ means that an instruction owned by $s$ was fetched
during a state-transformation triggered by an event $ev$ from a state $h$. With
a hardware-software mapping and an event-software mapping, we give a formal
definition of a \textit{software execution tampering}.

\begin{definition}[Software Execution Tampering]
  \label{def:codeinjection}
  Given $h$ the initial state of a state-transformation triggered by an event
  $ev$, $context$ a hardware-software mapping, $fetched$ an event-software
  mapping and $x, y \in \set{S}$ two software components, the software component
  $y$ tampers with the execution of another software component $x$ if the CPU
  fetches an instruction owned by $y$ in a state dedicated to $x$.
  \[ \begin{array}{l}
      software\_tampering(context, fetched, h, ev, x, y)
    \triangleq \\
    \qquad\qquad\qquad\qquad context(h) = x\,\wedge\,y \in fetched(h,ev)
  \end{array}
\]
\end{definition}

Given $\set{T} \subseteq \set{S}$ a set of trusted software components, the
software execution isolation policy prevents the untrusted components from
tampering with the execution of the trusted components. Such a policy is
enforced during a run if no untrusted component is able to tamper with the
execution of a trusted component.

\begin{definition}[Software Execution Isolation]
  \label{def:softwareisolation}
  Given $context$ a hardware-software mapping, $fetched$ an event-software
  mapping and $\rho$ a run of $\shortLTS{Ez}$,
  \[ \begin{array}{l}
  software\_execution\_isolation(context, fetched, \rho, \set{T}) \triangleq \\
  \qquad\forall \transitionLTS{Ez}{h}{ev}{h'} \in \rho, \forall t \in \set{T}, \forall
  u \not\in \set{T}, \\
  \qquad\qquad \neg software\_tampering(context, fetched, h, ev, t, u)
  \end{array} \]
\end{definition}

In this definition, $t$ is a trusted software component and $u$ is an untrusted
---~potentially malicious or hijacked~--- one.

\subsection{Hardware-based Security Enforcement Mechanism}

A HSE mechanism is a set of requirements on states to characterize safe
hardware configurations and a set of requirements on state-transformations to
preserve the state requirements through software execution. The software
components which implement a HSE mechanism form the Trusted Computing Base
(TCB).

\begin{definition}[HSE Mechanism]
  \label{def:hse}
  Given $\set{H} $ a set of states of a hardware architecture, $\set{E}$ a set
  of events and $\shortLTS{Ez}$ a Computing Platform, we model a HSE mechanism
  $\Delta$ with a tuple $(\fun{inv}, \fun{behavior}, \set{T}, context)$ such
  as
  \begin{compactitem}
  \item $inv$ is a predicate on $\set{H}$ to distinguish between safe hardware
    configurations and potentially vulnerable ones
  \item $behavior$ is a predicate on $\set{H} \times \set{E}_{Soft}$ to
    distinguish between safe software state-transformations and potentially
    harmful ones
  \item $\set{T} \subseteq{S}$ is the set of software components which form the
    TCB of the HSE mechanism
  \item $context$ is a hardware-software mapping to determine when the TCB is
    executed
  \end{compactitem}
\end{definition}

For instance, in x86-based hardware architectures, the SPI Flash contents (the
code and configuration of the firmware) is protected as follows:

\begin{compactenum}
  \item By default, the SPI Flash is locked and its content cannot be overriden
    until it has been unlocked
  \item Some software components can unlock the SPI Flash
  \item When they do so, the CPU is forced to start the execution of a
    special-purpose software component
  \item This software component has to lock the SPI Flash before the end of
    its execution
\end{compactenum}
In this example, the special-purpose software component is the TCB. A safe
hardware state (modeled with $inv$) is either a state wherein the
special-purpose software component is executed or a state wherein the SPI Flash
is locked. This requirement on hardware architecture states is preserved by
preventing the special-purpose software component to end its execution before it
has locked the SPI Flash (modeled with $behavior$).

For a HSE mechanism to be correctly defined, it must obey a few axioms, together
called the HSE Laws. The first law says that the state requirements specified by
$inv$ are preserved through state-transformations if the software
transformations which do not satisfy $behavior$ are discarded. The second law
says that the $behavior$ predicate specifies state-transformations restrictions
for the TCB only. The software components which are not part of the TCB are
considered untrusted and we make no assumption on their behavior.

\begin{definition}[HSE Laws]
  \label{def:laws}
  A HSE mechanism $\Delta = (inv, behavior, \set{T}, context)$ has to satisfy
  the following properties:
  \begin{compactenum}
  \item $behavior$ preserves $inv$: $\forall \transitionLTS{Ez}{h}{ev}{h'}$,
      \[ \begin{array}{l}
            inv(h) \Rightarrow (ev \in
            \set{E}_{Soft} \Rightarrow behavior(h,ev)) \Rightarrow inv(h')
        \end{array}
      \]
    \item $behavior$ only restricts the TCB: $\forall x \not\in \set{T}, \forall
      h \in \set{H}, \forall ev \in \set{E}_{Soft}$,
      \[
        \begin{array}{l}
          context(h) = x \Rightarrow behavior(h, ev)
        \end{array}
      \]
  \end{compactenum}
\end{definition}

A run complies to a HSE mechanism definition if its initial state satisfies the
state requirements and each state-transformation of the run satisfies the
state-transformations requirements. The set of the runs which comply with
$\Delta$ is denoted by $\mathcal{C}(\Delta)$.

\begin{definition}[Compliant Runs]
  Given $\rho \in \pathesLTS{Ez}$, \[ \rho \in \mathcal{C}(\Delta) \triangleq
    inv(init(\rho))\,\wedge\,\forall \transitionLTS{Ez}{h}{ev}{h'}, ev \in
  \set{E}_{Soft} \Rightarrow behavior(h,ev) \]
\end{definition}

Eventually, we aim to prove that a HSE mechanism is sound \mbox{---it} succeeds
to enforce a security policy--- under the assumption that software components of
the TCB always behave according to the specification given in the HSE mechanism
definition.

\begin{definition}[Sound HSE Mechanism]
  \label{def:sound}
A HSE mechanism $\Delta$ succeeds in enforcing a security policy $P$ when each
compliant run of $\Delta$ is secure. In such a case, $\Delta$ is said to be
sound.
\[ sound(\Delta, P) \triangleq \forall \rho \in \set{C}(\Delta), P(\rho)
\]
\end{definition}

\begin{table}[h]
  \begin{tabular}{clc}
    \hline
    \bf Notation  & \multicolumn{1}{c}{\bf Description} \\
    \hline
    $\set{S}$ & Set of software components \\
    \hline
    $\set{H}$ & Set of states of the hardware architecture \\
    \hline
    $\set{E}$ & Set of states of events of the hardware architecture \\
    \hline
    $\shortLTS{Ez}$ & Semantic of events as state-transformers (Computing
    Platform) \\
    \hline
    $\transitionLTS{Ez}{h}{ev}{h'}$ & State-transformation according to
    $\shortLTS{Ez}$ \\
    \hline
    $\pathesLTS{Ez}$ & Set of sequences of $\shortLTS{Ez}$ state-transformations
    (Runs) \\
    \hline
    $P$ & Predicate on run to model a EM-enforceable security policy \\
    \hline
    $\Delta$ & Requirements on states and state-transformation (HSE mechanism)
    \\
    \hline
  \end{tabular}
  \caption{SpecCert CheatSheet}
\end{table}

\section{\textsc{MinX86}} \label{sec:speccert:hardware}

The SpecCert formalism is the foundation of the SpecCert framework. It comprises
a set of high-level definitions to specify a HSE mechanism against a hardware
architecture model. In its current state, the SpecCert framework contains a
model of x86 called $\formatLTS{Minx86}$. $\formatLTS{Minx86}$ is intended to be
a minimal model for single core x86-based machines and we have used publicly
available Intel documents\,\cite{intel2013celeron,intel2009mch,intel2014manual} to define
it.

\subsection{Model Scope}

The hardware architecture we are modeling with $\formatLTS{Minx86}$ contains a
CPU, a cache, a memory controller, a DRAM controller and a VGA
controller\,\footnote{A VGA controller is a hardware device which on we can
connect a screen. It exposes some memory to the CPU for communication purposes.}
which both expose some memory to the CPU.

$\formatLTS{Minx86}$ is meant to be a proof of concept of the SpecCert formalism
and thus is not exhaustive. In its current state of implementation, its scope
focuses on the System Management Mode (SMM) feature of x86 microprocessors.

\paragraph{Hardware Specifications}
We consider the CPU can be either in System Management Mode (SMM) or in an
unprivileged mode. The SMM is "a special-purpose operating mode provided for
handling system-wide functions like power management, system hardware control,
or proprietary OEM-designed code"\,\cite{intel2014manual}. It is the most privileged
execution mode of x86 processors.  When a CPU receives a special hardware
interrupt called System Management Interrupt (SMI), it halts its current
execution and reconfigures itself to a specified state from which it executes
the code stored in memory at the address $SMBASE + \texttt{0x8000}$. In
practice, the SMBASE value points to the base of a memory region called the
SMRAM. Leaving the SMM is done by executing a special purpose instruction called
\texttt{rsm} (for \emph{resume}).

The CPU relies on a cache to reduce the Input/Output (\IO, that is a read or
write access to the memory) latency. We model one level of cache which stores
both data and instructions and we consider two cache strategies: uncacheable
(UC) and writeback (WB). With the UC cache strategy, the cache is not used and
all \IOs are forwarded to the memory controller, whereas with the WB strategy,
the cache is used as much as possible\,\footnote{These cache strategies are
explained in \cite{intel2014manual}, Volume 3A, Chapter 11, Section 11.3 (page 2316 --
2317)}. To determine which cache strategy to use, the CPU relies on several
configuration registers and mechanisms. One of them is a pair of registers
called the System Management Range Registers (SMRR) which can only be configured
when the CPU is in SMM. They are used to tell the CPU where the SMRAM is and
which cache strategy to use for \IO targeting the SMRAM when the CPU is in SMM.
When it is not in SMM, the CPU always uses the UC strategy for \IO targeting the
SMRAM. SMRR have been introduced as a countermeasure of the SMRAM cache
poisoning attack\,\cite{wojtczuk2009smram,duflot2009smram} which allowed an
untrusted code to tamper with the copy of the SMRAM stored in the cache.  The
memory controller\,\cite{intel2009mch} receives all the CPU \IOs which are not
handled by the cache and dispatches them to the DRAM controller or to the VGA
controller. It exposes a unified view (the memory map) of the system memory to
the CPU. The CPU manipulates this memory map with a set of addresses called the
physical addresses. The memory controller dedicates a special range of physical
addresses to form the SMRAM. The SMRAM is dedicated to store the code intended
to be executed when the CPU is in SMM.

\paragraph{Tracking the Memory Ownership} The \formatLTS{Minx86} definition is
parameterized with an hardware-software mapping (see
Definition~\ref{def:hardsoftmap}). The memory locations of \formatLTS{Minx86}
Computing Platforms are either cache lines or memory cells exposed by the DRAM
controller or the VGA controller. The memory ownership is updated through
state-transformations according to three rules:
\begin{compactenum}
  \item When a cache line gets a copy of a DRAM or VGA cell content, the owner
    of this cell becomes the new owner of this cache line.
  \item When the content of this cache line is written back to a memory cell,
    the new owner of this memory cell is the owner of this cache line.
  \item When a state-transformation implies the content of a memory location to
    be overriden with a new value, the software currently executed becomes its
    new owner.
\end{compactenum}

Given $\set{S}$ a set of software components, the set of states of
\formatLTS{Minx86} Computing Platform hardware architecture is denoted by
$\texttt{Archi}_{\set{S}}$ and the set of \formatLTS{Minx86} Computing Platform
events is denoted by $\texttt{Event}$. Given $context$ a hardware-software
mapping, we denote the Computing Platform $\formatLTS{Minx86}$ parameterized
with $context$\,\footnote{The related definitions and explanations are given on
page \pageref{page:minx86def}.} such that \[ \formatLTS{Minx86}(context)
\triangleq (minx86\_pre, minx86\_post(context)) \]

\subsection{Hardware Architecture State}

$\texttt{Archi}_{\set{S}}$ is defined as the Cartesian product of the set of
states of the CPU, the CPU's cache, the memory controller and the hardware
memories exposed by both the DRAM controller and the VGA controller. Each of
these sets is defined in order to model the hardware features we have previously
described. We define $\texttt{PhysAddr} \triangleq \setdef{\val{pa}_i}{i \leq
\val{max\_addr}}$ the set of physical addresses the CPU uses to perform \IO. The
maximal address offset (denoted by $\val{max\_addr}$ here) is specific to the
CPU and may vary in time according to its addressing mode (real mode, long mode,
etc.), therefore we left its value as a parameter of our model.  An in-depth
definition of $\texttt{Archi}_{\set{S}}$ is given in appendix~\ref{app:archis}.
% TODO

We model the projection of the SMRAM in the memory map such that
$\texttt{pSmram} \triangleq \setdef{\val{pa}_i}{\val{smram\_base} \leq i \le
\val{smram\_end}}$.  The values of $\val{smram\_base}$ and $\val{smram\_end}$
are specified in the memory controller specifications. It is the software
responsability to set the SMRR accordingly. We assume $\val{smram\_end} -
\val{smram\_base} > \val{0x8000}$. This way, when the SMBASE contains the
address of the beginning of the SMRAM, the SMM entry point (that is $SMBASE +
\val{0x8000}$) is in SMRAM.

The hardware architecture states are implemented in the
\emph{SpecCert.x86.Archi\-tecture} module (about 1\,500 lines of code).

\subsection{Events as State-Transformers}

The set of events which trigger the state-transformations is denoted by
$\texttt{Event}$. As we said in Section \ref{subsec:speccert:computing}, we
distinguish hardware events denoted by $\texttt{Event}_{Hard}$ and software
events denoted by $\texttt{Event}_{Soft}$.

\begin{table}
  \bigcentering
  \begin{tabular}{lp{3cm}p{6cm}}
    \hline
    \textbf{Event} & \textbf{Paramters} & \textbf{Description} \\
    \hline
    $Write$ & $pa \in \texttt{PhysAddr}$ & A CPU \IO to write at physical address
    $pa$ \\
    \hline
    $Read$ & $pa \in \texttt{PhysAddr}$ & A CPU \IO to read at physical address $pa \in
    \texttt{PhysAddr}$ \\
    \hline
    $SetCacheStrat$ & $pa \in \texttt{PhysAddr}$ \newline $strat \in
    \setshortdef{\val{UC}, \val{WB}}$ & Change the cache strategy for $pa$ to
    $strat$ ($\val{WB}$ means write-back and $\val{UC}$ means uncacheable) \\
    \hline
    $UpdateSmrr$ & $smrr \in \texttt{Smrr}$ & Update the SMRR content with the
    new value $smrr$ \\
    \hline
    $Rsm$ & \centering --- & The CPU leaves SMM \\
    \hline
    $OpenBitFlip$ & \centering --- & Flip the $d\_open$ bit \\
    \hline
    $LockSmramc$ & \centering --- & Set the $d\_lock$ bit \\
    \hline
    $NextInstruction\quad$ & $pa \in \texttt{PhysAddr}$ & The program counter register
    of the CPU is set to $pa$ \\
    \hline
  \end{tabular}
  \caption{List of software events}
  \label{tab:softev}
\end{table}

Table\,\ref{tab:softev} lists the software events we consider in the
$\formatLTS{Minx86}$ Computing Platforms. We model the CPU \IOs with $Read(pa)$
and $Write(pa)$, the configuration of the memory controller with $OpenBitFlip$
and $LockSmramc$, the configuration of the cache strategy with
$SetCacheStrat(pa,strat)$, the configuration of the SMRR with $UpdateSmrr(smrr)$
the exit of the SMM with $Rsm$ and the update of the CPU program counter
register with $NextInstruction(pa)$.

\begin{table}
  \bigcentering
  \begin{tabular}{lp{9cm}}
    \hline
    \textbf{Event} & \textbf{Description} \\
    \hline
    $Fetch$ & A CPU \IO to fetch the instruction stored at the physical address
    contained in the program counter register \\
    \hline
    $ReceiveSmi\quad$ & A SMI is raised and the CPU handles it \\
    \hline
  \end{tabular}
  \caption{List of hardware events}
  \label{tab:hardev}
\end{table}

The other causes of state-transformations are modeled using hardware events.
Table \ref{tab:hardev} lists the hardware events we consider in the
$\formatLTS{Minx86}$ Computing Platforms. $Fetch$ models the \IO to fetch the
instruction pointed by the program counter register. $ReceiveSmi$ models a
System Management Interrupt being risen and handled by the CPU.

We define $minx86\_fetched$ an event-software mapping for $\formatLTS{Minx86}$
Computing Platforms (see Definition \ref{def:evsoft}). The $minx86\_fetched$ function
maps a state-transformation to the set of software components which own an
instruction fetched during this state-transformation. In the case of
\formatLTS{Minx86}, there is only one event which implies fetching instructions:
$Fetch$. Let $o$ be the owner of the instruction pointed by the program counter
register in the formula
\[ minx86\_fetched(h, ev) \triangleq \begin{cases}
    \setshortdef{o} & \text{if }ev = Fetch \\
    \emptyset & \text{otherwise} \\
\end{cases} \]
We can determine $o$ because $\formatLTS{Minx86}$ tracks the memory location
ownership.

\label{page:minx86def} Given $context$ a hardware-software mapping (see
Definition \ref{def:hardsoftmap}), the precondition predicate of the Computing
Platform $\formatLTS{Minx86}(context)$ is named $min\-x86\_pre$ and the
postcondition predicate is named $min\-x86\_post(con\-text)$. We give an
informal description of the $min\-x86\_pre$ and $minx\-86\_post(context)$ for
each event. These definitions have been implemented in Coq in the module
\emph{Spec\-Cert.x86.Transi\-tion}.

We first give the semantics of software events as state-transformers. A software
component can always read and write at any physical address. As a consequence,
the precondition for $Read(pa)$ and $Write(pa)$ always holds true. The
postcondition for $Read(pa)$ and $Write(pa)$ requires the memory ownership to be
updated according to the memories and cache state updates. The memory controller
enforces a simple access control to protect the SMRAM content in the DRAM memory
by forwarding the related \IO to the VGA controller when the CPU is not in SMM.
To determine the owner of the memory location which sees its content overriden
during a state transformation, the postcondition uses the hardware-software
mapping used to define the Computing Platform.

A software component can always update the cache strategy used for an \IO. The
postcondition for $SetCacheStrat(pa,strat)$ requires only the cache strategy
setting for this physical address $pa$ to change. The precondition for
$UpdateSmrr$ requires the CPU to be in SMM. The postcondition requires the SMRR
of the CPU to be updated with the correct value, the rest of the hardware
architecture state being left unchanged.

A software component can jump to any physical address, hence the postcondition
for $NextInstru\-ction(pa)$ always holds true. The postcondition for
$NextInstru\-ction(pa)$ requires the program counter register to be updated with
$pa$. The $OpenBitFlip$ precondition requires the SMRAMC register to be
unlocked. The postcondition requires the \texttt{d\_open} bit to be updated. The
$Lock\-Smramc$ precondition requires the $\texttt{d\_lock}$ bit to be unset. The
postcondition requires the $\texttt{d\_open}$ bit to be unset and the $d\_lock$
bit to be unset.

We now describe the semantics of hardware events as state-transformers. $Fetch$
models the fetching of an instruction by the CPU. As a consequence, the
definition of its precondition and postcondition are the same as $Read(pa)$ with
$pa$ being the program register value.  $ReceiveSmi$ precondition requires the
CPU not to be in SMM because SMM is non-reentrant.  The postcondition of
$ReceiveSmi$ requires the program counter to be set with the $smbase +
\val{0x8000}$ (where $smbase$ is the value of the SMBASE register of the CPU)
and the CPU is in SMM.

\section{System Management Mode HSE} \label{sec:speccert:smm}

In \cite{intel2014manual}, Intel states "the main benefit of SMM is that it
offers a distinct and easily isolated processor environment that operates
transparently to the operating system or executive and software
applications". For the SMM processor environment to be isolated, the code
executed when the CPU is in SMM needs to implement a HSE mechanism. In this
section, we formalize and verify this mechanism against the model we have
previously introduced.

\subsection{Computing Platform and Security Policy}

We consider three software components: the boot sequence code, the SMM code and
the OS code. During the boot sequence, only the boot sequence code is executed
and it loads both the OS code and the SMM code into memory. At the end of the
boot sequence, the OS kernel is executed. This OS kernel will schedule different
applications. Because applications are less privileged than the OS kernel, we
will not distinguish them from the kernel code. Thus, in the following, OS code
refers to both OS kernel and application codes.

At runtime, both the OS code and the SMM code can be executed. Our objective is
to evaluate the security provided by the hardware to isolate SMM code from OS
code. Thus, we define \[ \set{S} \triangleq \setshortdef{\val{smm}, \val{os}} \]

We assume the SMM is dedicated to the SMM code.  Let $cpu\_in\_smm :
\texttt{Archi}_{\set{S}} \rightarrow \setshortdef{\val{true}, \val{false}}$ be
the function which returns $\val{true}$ if the CPU is in SMM and $\val{false}$
otherwise.  We define $smm\_context$ a hardware-software mapping such that
\[ smm\_context(h) \triangleq \begin{cases}
  \val{smm} & \text{if } cpu\_in\_smm(h) = \val{true} \\
  \val{os} & \text{otherwise}
\end{cases} \]
Let $\formatLTS{Smmx86}$ be the Computing Platform such as \[ \formatLTS{Smmx86}
\triangleq \formatLTS{Minx86}(smm\_context) \]

We assume that both the OS code and the SMM code have been loaded in distinct
memory regions. In particular, all the SMM code has been loaded in SMRAM. Our
objective is to enforce a security policy which prevents the OS code to tamper
with the SMM code execution. This way, the SMM (which is the most privileged
execution mode of the CPU) cannot be used to perform an escalation privilege. We
define $smm\_security$ a predicate to model this security policy such as given
$\rho \in \formatLTS{Smmx86}$, \[ \begin{array}{l} smm\_security(\rho)
\triangleq \\ \qquad software\_execution\_isolation(smm\_context,
minx86\_execute, \rho, \setshortdef{\val{smm}}) \end{array} \]

\subsection{HSE Definition}

We define $\Delta_{Smm}$ to model the HSE mechanism applied by the SMM code such
that $\Delta_{Smm} = (inv_{Smm}, behavior_{Smm}, \setshortdef{\val{smm}},
smm\_context)$ (see Definition \ref{def:hse}).

In order to enforce the SMM security policy, we have identified six requirements
on states.
\begin{compactitem}
\item When the CPU executes the SMM code, the program counter register value
needs to be an address in SMRAM.
\item The SMBASE register was correctly set during the boot sequence to point to
  the base of the SMRAM.
\item The SMRAM contains only SMM code.
\item For a physical address in SMRAM, in case of cache hit, the related cache
  line content must be owned by the SMM code.
\item In order to protect the content of the SMRAM inside the DRAM memory, the
  boot sequence code has locked the SMRAMC controller. This ensures that an OS
  cannot set the \texttt{d\_open} bit any longer and only a CPU in SMM can
  modify the content of the SMRAM.
\item The range of memory declared with the SMRR needs to overlap with the
  SMRAM.
\end{compactitem}

The Appendix~\ref{app:invsmm} gives the formal definitions of each requirements
and of $inv_{Smm}$.  We now define $\fun{behavior_{Smm}}$. We only define two
restrictions. First, we force the SMM code execution to remain confined within
the SMRAM. The reason is simple: the OS code can tamper with the memory outside
the SMRAM. As a consequence, jumping outside the SMRAM is the best way to fail
the security policy. Secondly, we prevent the SMM code to update the SMRR
registers as it is the responsability of the boot sequence code to correctly set
them.

\[
  \fun{behavior_{Smm}}(h, ev) \triangleq \begin{array}{l}
    \fun{smm\_context}(h) = \val{smm} \\
                                   \quad \Rightarrow ((e = NextInstruction(pa)
                                   \Rightarrow pa \in \texttt{pSmram}) \\
                                   \qquad \wedge\,(e \neq UpdateSmrr(smrr))) \\
                                 \end{array}
  \]

For $\Delta_{Smm}$ to be a HSE mechanism, we need to prove the two HSE Laws (see
Definition \ref{def:laws}).  The first law states the state requirements modeled
with $inv_{Smm}$ are preserved through state-transformations if the
transformations which do not satisfy $behavior_{Smm}$ are discarded. We prove
this by enumeration of $ev \in \texttt{Event}$ and $h \in \texttt{Archi}_{Smm}$,
we check that each requirement described previously is preserved by
$\Delta_{Smm}$. We use those intermediary results to conclude. The second law
states that the $behavior_{Smm}$ predicate specifies state-trans\-formation
requirements for the TCB only. In this use case, it means $behavior_{Smm}$
should always hold true when the OS code is executed by the hardware
architecture.  By definition of $behavior_{Smm}$, $smm\_context(h) = \val{smm}$
is an antecedent of the conditional.

Let $smm\_secure\_transformation$ be a predicate which holds true when a
state-transformation does not imply the OS code to tamper with the execution of
SMM code.
\[ \begin{array}{l}
    smm\_secure\_transformation(h, ev) \triangleq \\
    \qquad \neg software\_tampering(smm\_context, minx86\_execute, h, ev,
                  \val{os}, \val{smm})
\end{array} \]

We prove that this predicate holds true for a state-transformation with respect
to the HSE mechanism. With this result, we can prove the HSE mechanism is sound
(see Definition \ref{def:sound}).

\begin{lemma}[Invariants Enforce Security]
$\forall \transitionLTS{Smmx86}{h}{ev}{h'}$,
\[ \begin{array}{l}
    inv_{Smm}(h) \\
    \qquad \Rightarrow (ev \in \texttt{Event}_{Soft} \Rightarrow
    behavior_{Smm}(h,ev) \\
  \qquad\qquad \Rightarrow smm\_secure\_transformation(h, ev)
\end{array} \]
\end{lemma}

\begin{proof}
  By enumeration of $ev \in \texttt{Event}$ and $h \in
  \texttt{Archi}_{\set{S}}$.
\end{proof}

\begin{theorem}[$\Delta_{Smm}$ is Sound]
\[ sound(\Delta_{Smm}, smm\_security) \]
\end{theorem}

\begin{proof}
  The "Invariants Enforce Security" lemma applies for one transition and the
  first HSE law allows to reason by induction on runs.
\end{proof}

\section{Discussion} \label{sec:speccert:discuss}

Our effort has been originally motivated by the disclosure of several
vulnerabilities targeting multiple x86 HSE mechanisms for the past few
years\,\cite{wojtczuk2009smram,duflot2009smram,rutkowska2008remap,domas2015sinkhole,kallenberg2015racecondition}. These
attacks do not benefit from a software implementation error but rather from a
flaw in the hardware specifications themselves. The result of our work is a
three-steps methodology for formally specifying and verifying HSE mechanisms
against a hardware architecture model. We believe each aspect is important.

First, the hardware architecture model can be used as a formal specification.
The main benefit of a formal specification is to avoid any ambiguity such as the
one we have found in \cite{intel2009mch}. One can read at Section 3.8.3.8,
page 102 that “the OPEN bit must be reset before the LOCK bit is set”.  At the
same page, in the description of the LOCK bit, one can also read that “when
[LOCK] is set to 1 then [OPEN] is reset to 0”. We had modeled the second
statement as the behavior of the memory controller is not specified if the
first statement is true\,\footnote{If we had to actually implement the HSE
mechanism, we would have to assume the first was the correct one.}
\formatLTS{Minx86} as a formal specification does not suffer from the same flaw.

Secondly, a formal specification of a HSE mechanism will help software
developers when the time comes to implement it. For instance, the Chapter 34,
Volume 3C of \cite{intel2014manual} about SMM is about 30 pages long, it gives
many details on how the SMM actually works, yet no section is actually dedicated
to security. On the contrary, our HSE mechanism definition gathers six
requirements on hardware configurations and two requirements on software
executions to enforce a well-defined security property. Even if the proofs only
apply to an abstract model, we believe it is a valuable improvement.

Lastly, the verification process of a HSE mechanism specification against a
hardware architecture model may help to highlight hidden flaws in the hardware
specifications assumptions. We take the example of the SMRAM cache poisonning
attack\,\cite{wojtczuk2009smram,duflot2009smram}, which has motivated the
introduction of the SMRR. If an attacker can set the proper cache strategy (WB)
for the SMRAM physical addresses, then the code inside the SMRAM is loaded into
the cache as soon as the CPU in SMM is executing it. From this point forward
---because the access control is enforced at the memory controller level---
nothing prevents the attacker to tamper with it. The next time the CPU enters in
SMM, it executes the code stored in the cache. With a SMRR-less version of
\formatLTS{Minx86}, we were not able to conclude our HSE mechanism was sound:
such a scenario draws attention of the SpecCert user who is forced to
investigate.

From our point of view, the clear separation between the hardware model, the
security properties and the HSE mechanisms to enforce those properties are the
main advantage of our approach, as two different use cases can be studied
against the same hardware model. However, the scalability of SpecCert remains to
be proven.