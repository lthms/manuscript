%!TEX root = ../main.tex
\chapter{Conclusion and Perspectives}
\label{chapter:conclusion}

\endquote{``\emph{ I suppose it is tempting, if the only tool you have is a
    hammer, to treat everything as if it were a nail.}''

  \hfill\footnotesize --- Abraham Maslow}

\vspace{1cm}%
\noindent
%
The starting point of this thesis is the SMRAM cache poisoning
attack\,\cite{duflot2009smram,wojtczuk2009smram}, and more generally various
compositional attacks\,\cite{wing2003compositionalattack} against the x86
hardware
architecture\,\cite{kovah2015senter,domas2015sinkhole,kallenberg2015racecondition}.
%
These attacks have in common to leverage inconsistencies in the hardware
specifications to defeat security policies supposedly enforced by the hardware
architecture.

In this thesis, we have identified a class of security enforcement mechanisms
called Hardwase-based Security Enforcement (HSE) mechanisms, which consist in
the configuration by a trusted software component of the underlying hardware
architecture in order to constrain the execution of untrusted software
components with respect to a targeted security policy.
%
We have investigated the use of formal methods to formally specify and verify
\ac{hse} mechanisms as a potential solution to uncover compositional attacks.
%
We steer a middle course between two domains: hardware verification and system
software verification.
%
Generally, hardware verification focuses on properties which are transparent to
the executed software and system software verification relies on hardware models
which abstract the architecture complexity as much as possible .
%
On the contrary, when it comes to \ac{hse} mechanisms it is important to
consider that
%
\begin{inparaenum}[(1)]
\item \label{needreuse}%
  hardware architectures often allow for implementing several \ac{hse}
  mechanisms, and
  %
\item \label{needreduce}%
  hardware features involved in \ac{hse} mechanisms are not safe by default.
  % hence the role played by trusted software components to configure them.
\end{inparaenum}

In the following, we first give a brief summary of our two contributions and
then suggest some possible directions for future work.

\section{Summary of the Contributions}
\label{sec:conclusion:summary}

Our contribution is twofold.
%
As a first step, we have proposed a theory of HSE mechanisms.
%
Furthermore, we have proposed a compositional reasoning framework for Coq, based
on our experience in implementing a proof of concept for our theory.

\paragraph{A Theory of HSE Mechanisms.}
%
We have proposed in Chapter~\ref{chapter:speccert} a theory of \ac{hse}
mechanisms, such that a mechanism is primarily characterized by a set of trusted
software components, a set of requirements over states and a set of requirements
over software transitions.
%
We have evaluated our approach in Chapter~\ref{chapter:speccert2}.
%
We have introduced {\scshape Minx86}, a minimal model for a single core
x86-based computing platform, and we have used it to specify and verify the HSE
mechanism implemented by the \ac{bios} to remain isolated from the rest of the
software stack at runtime.
%
We have written machine-checked proofs in Coq to increase our confidence in our
result.
%
Proofs related to the Chapter~\ref{chapter:speccert} have been commented in
Appendix~\ref{appendix:speccert} and those related to the
Chapter~\ref{chapter:speccert2} have been released as free
software\,\cite{letan2016speccertcode}.

\paragraph{Compositional Reasoning for Coq.}
%
We have proposed in Chapter~\ref{chapter:freespec} a novel approach which
enables modular verification of complex systems made of interconnected
components.
%
This approach is the result of various lessons learned during the development of
our first proof of concept, and we believe it represents a first step towards
addressing the challenge posed by the scale of the x86 hardware architecture.
%
Components of a system are primarily identified by the interface they expose,
and secondarily by their current state and the interfaces they uses.
%
We have introduced so-called abstract specifications to allow for reasoning
about components in isolation \emph{and} about the expected properties of their
composition.
%
Besides, the resulting Coq framework, called FreeSpec and also made available as
free software\,\cite{letan2018freespeccode}, is not specific to hardware models,
and could also be leveraged to reason about composition of software components
as well.

\section{Perspectives}
\label{sec:conclusion:perspectives}

This thesis introduces a theory for specifying and verifying \ac{hse} mechanism as
restrictions on hardware models, and a formalism to define and reason about
these models in terms of interconnected components.

\subsection*{Using FreeSpec to Specify and Verifying HSE Mechanisms}
%
Our theory of \ac{hse} mechanisms and our compositional reasoning framework for
Coq remain two separated projects.
%
The most natural continuation of our work would be to connect them, \emph{e.g.}
by substituting {\scshape Minx86} with a model developed thanks to FreeSpec.
%
Going further, we are convinced a general-purpose model for the x86 architecture
would worth the time and effort spent to its construction.
%
However, we have no doubt that the limitations of FreeSpec ---in terms of
components interaction patterns--- still reduce its applicability for the most
complex parts of the x86 hardware architecture.

\subsection*{Extending FreeSpec With a Model Validation Framework}
%
As a complement, the trustworthiness of a general-purpose hardware model is of
key importance, as emphasized by Reid \emph{et al.} for their formal
specification of ARM\,\cite{reid2016armv8}.
%
Extending FreeSpec with a validation model framework would be an important step
in that direction.
%
We have taken great care for FreeSpec to be compatible with the code extraction
feature of Coq, which means we can turn components model into executable
programs.
%
This feature opens interesting opportunities, but remains only a first step
towards a practical model validation framework.
%
In addition to traditional challenges, validating a model of a \ac{pch}, for
instance, promises to be challenging because of its tight integration inside
Intel chips.
%
Focusing our efforts on open source processors, such as
Leon4\,\cite{gaisler2007leon}, could allow us to investigate further this avenue
of research.
