%!TEX root = ../main.tex
\chapter{Intel x86 Architecture and BIOS Background}
\label{chapter:usecase}

\endquote{``\emph{You’re building your own maze, in a way, and you might just
    get lost in it.}''

  \hfill\footnotesize --- Marijn Haverbeke}

\vspace{1cm}%
\noindent
%
From a market share perspective, the x86 hardware architecture is widely used
for laptops, desktops and servers.
%
Intel has introduced several security features to support \ac{hse} mechanism
over the past decades: hardware-based virtualization (VT-x, VT-d)\,\cite[Volume
  3, Chapter 23]{intel2014manual}, dynamic root of trust
(TXT)\,\cite{intel2015txt}, or applicative enclaves (SGX)\,\cite[Volume 3,
  Chapter 36]{intel2014manual}\cite{costan2016sgxexplained}.
%
As for the \ac{bios}, it is the most privileged piece of software executed by
the hardware architecture, and implements several \ac{hse} mechanisms to remain
isolated from the rest of the software stack.
%
The correctness of these mechanisms is therefore of key importance; yet they
have been the object of several compositional
attacks\,\cite{duflot2009smram,wojtczuk2009smram,domas2015sinkhole,kallenberg2015racecondition,kovah2015senter}.
%
These attacks have motivated our effort to provide a formal framework for
reasoning about \ac{hse} mechanisms.
%
However, it is important to emphasize that other mainstream architectures
(\emph{e.g.}  ARM) work in a similar basis and potentially suffer similar issues.
%
Our contributions are thus intended to be applicable to other hardware
architectures.

The rest of this Chapter proceeds as follows.
%
We provide the necessary details on how a typical x86 hardware architecture
works (Section~\ref{sec:usecase:architecture}) and we explain why and how the
\ac{bios} remains isolated from the rest of the software stack at runtime
(Section~\ref{sec:usecase:firmware}).
%
This background allows us to describe the critical compositional attacks we
already mentioned (Section~\ref{sec:usecase:hse}).

\section{Introduction to x86 Architecture}
\label{sec:usecase:architecture}

Describing hardware architectures in depth is challenging, because they tend to
comprise an increasing number of interconnected components of various natures.
%
The x86 hardware architecture is a perfect illustration of this, which is
probably best demonstrated by the scale of its documentation.
%
At the time of writing this thesis\,\footnote{Spring 2018.}, the \emph{Intel 64
  and IA-32 Architectures Software Developer’s Manual} is 4,842 pages long.
%
A typical computing platform is made of dozens of hardware components, including
\emph{e.g.} hard drives, a keyboard, a trackpad, an audio controller, and a
graphic card.
%
Several come with their own documentation, often in the form of large datasheet.

Figure~\ref{fig:usecase:x86arch} pictures how the most important hardware
components are interconnected in a modern x86 computing platform (generation
\emph{Nehalem}\,\cite{thomadakis2011nehalem} and newer).
%
The two main components are the processor ---responsible for executing the
software stack--- directly connected to high-speed peripherals (\emph{e.g.}
DRAM, a display controller), and a companion chipset called the Platform
Controller which handles interactions with the rest of the peripherals
(\emph{e.g.} USB devices, hard drives).
%
The x86 architecture has seen many iterations over the years, and it is quite
common to find in the same computer park different versions of this
architecture.
%
For instance, the processor was previously connected to a \emph{northbrige}
(low-latency), which was itself connected to a \emph{southbridge} (slower
peripherals).

This section proceeds as follows.
%
First, we describe the internals of a x86 processor (\ref{subsec:usecase:proc}).
%
We detail the mechanisms which play a part in interactions between the processor
and the memories scattered inside the hardware architecture
(\ref{subsec:usecase:mem}), then focus on the caches embedded inside the
processor to reduce the latency induced by these interactions
(\ref{subsec:usecase:cachemem}).
%
Finally, we explain how peripherals of the hardware platform actively
communicate with the processor (\ref{subsec:usecase:periphio}).

\begin{figure}
  \centering
  \begin{tikzpicture}
    \node [draw, inner sep=25pt] (MCH) {Processor};%

    \node [draw, inner sep=8pt, right=15pt of MCH, yshift=-20pt, text badly
    centered, text width=65pt] (DRAM) {DRAM Controller};%
    \node [draw, inner sep=8pt, right=15pt of MCH, yshift=20pt, text badly
    centered, text width=65pt] (Display) {Display Controller};%

    \draw (MCH.east) |- (Display.west);%
    \draw (MCH.east) |- (DRAM.west);%

    \node [draw, inner sep=8pt, left=15pt of MCH, text width=50pt, text badly
    centered] (PCIe) {PCI Express Controller};%

    \draw (PCIe) -- (MCH);%

    \node [draw, below=30pt of MCH, inner sep=40pt, text badly centered] (PCH)
    {PCH};%

    \draw (MCH) -- (PCH);%

    \node [draw, inner sep=5pt, below=10pt of PCH, text width=55pt,
    xshift=-55pt, text badly centered] (Flash) {Flash Memory};%
    \draw ([xshift=-10pt]PCH.south) |- (Flash.east);%

    \node [draw, inner sep=10pt, below=10pt of PCH, text width=25pt,
    xshift=55pt, text badly centered] (TPM) {TPM};%
    \draw ([xshift=10pt]PCH.south) |- (TPM.west);%

    \node [draw, inner sep=10pt, yshift=25pt, left=20pt of PCH, text width=65pt,
    text badly centered] (USB) {USB Controller};%

    \node [draw, inner sep=10pt, yshift=-25pt, left=20pt of PCH, text
    width=65pt, text badly centered] (PCI) {PCI Controller};%

    \draw (PCH.west) |- (USB.east);%
    \draw (PCH.west) |- (PCI.east);%

    \node [draw, inner sep=10pt, right=20pt of PCH, text badly centered, text
    width=50pt] (HD) {Hard Drive Controller};%

    \draw (PCH) -- (HD);%
  \end{tikzpicture}

  \caption{High-level view of the x86 hardware architecture}
  \label{fig:usecase:x86arch}
\end{figure}

\subsection{Processor, Architecture and Microarchitecture.}
\label{subsec:usecase:proc}

The main component of the architecture is the \emph{processor}.
%
It embeds several execution units called \emph{cores}, which are responsible for
executing assembly instructions of software component programs.
%
It also integrates several additional hardware modules to connect the cores to
the rest of the hardware architecture, \emph{e.g.} a memory controller to manage
interactions with the DRAM.
%
The concrete hardware implementation of the processor is often referred as the
Intel microarchitecture, in opposition to the Intel architecture which describes
the expected behavior and properties of a x86 system as seen by software
developers.
%
While Intel often modifies the microarchitecture, the architecture has remained
backward compatible for decades\,\cite{turley2014introx86}.
%
The microarchitecture implements many optimizations, such as
multithreading\,\cite{marr2002hypertheading}, instruction
pipelining\,\cite{fog2012microarchitecture}, out-of-order
execution\,\cite[Section 2]{fog2012microarchitecture} or predictive
branching\,\cite{milenkovic2002branchprediction}\cite[Section
3]{fog2012microarchitecture}.
%
% These optimizations should not violate properties of the architecture.
% %
% For instance, the predictive branching implemented by x86 \acp{cpu} increases
% performance, \emph{e.g.} inside a loop.
% %
% In presence of a conditional jump, a core will eagerly choose one branch
% ---based on the past execution--- before it has executed all the computations
% required to determine the correct execution path.
% %
% Most of the time, it chooses the correct branch.
% %
% In case of \emph{branch misprediction}, the erroneous computations performed
% by the core shall be canceled like if they have never happened.
% %
% \PC{quel est l'objectif de l'explication du \textit{branch misprediction} et
% l'exemple qui suivent ?  Cela me semble un peu hors contexte, on se demande
% pourquoi ce point est détaillé (et pas les autres)
%
% A supprimer ?}  We illustrate branch misprediction with the following program
% snippet, written with x86 assembly instructions (left) and its C equivalent
% (right)\,\footnote{Provided no optimization from the compiler nor the
% processor.}:
% %
% \begin{center}
%   \begin{minipage}[t]{0.35\linewidth}
%     \inputminted{asm}{Listings/predict.S}
%   \end{minipage}
%   \begin{minipage}[t]{0.35\linewidth}
%     \inputminted{c}{Listings/predict.c}
%   \end{minipage}
% \end{center}
% %
% During the first 10 steps of the loop, when the core executes the \texttt{jle}
% instruction (l. 5), it takes the same branch and jumps to the \texttt{loop}
% label.
% %
% At the end of the last loop step, the register \texttt{rax} contains the value
% \( 11 \).
% %
% According to the semantics of \texttt{jle}, the core should execute the
% \texttt{movq~\$0,~\%rax} (l.6) next.
% %
% From a microarchitectural point of view, the branch prediction technology can
% decide to optimize the execution of the program, by eagerly jumping to the
% \texttt{loop} label and continuing the execution of the program \emph{before}
% \texttt{\%rax} has been updated to \( 11 \).
%
% TODO: un meilleur snippet de code serait d’incrémenter ebx en plus de eax
%
% TODO : avant, limiter au temps d'accès cache. depuis le début de l'année 2018
% et meltdown/spectre, d'autres mécanismes ont pu être détournés de leur
% objectif de performance (branch prediction, tlb)

Intel microarchitecture blurs the frontier between hardware and software.
%
Indeed, an important part of the microarchitecture is not implemented as
hardware circuit, but rather under the form of \emph{microcode}
programs\,\cite[Section 2.14]{costan2016sgxexplained}.
%
That is, the processor is a programmable device, whose behavior ---including the
semantics of several x86 instructions it
implements\,\cite{johnson2015patentsgx}--- is partly determined by the microcode
it has loaded.
%
In practice, x86 processors only load microcode updates which have been signed
by Intel.
%
To the best of our knowledge, x86 microcode has never been successfully used as
an attack vector.

% GUILLAUME : Je dirais directement un CPU est aujourd'hui composé de plusieurs
% unités d'exécution indépendantes qu'on appelle des coeurs. Ces composants sont
% eux même constitués de différents blocs qui forme un pipeline d'exécution
% (fetch décode ALU, etc.) Ces coeurs peuvent être physiquement indépendants
% (i.e des unités qui possèdent leurs propres pipeline indépendant) ou bien
% virtuel. Dans ce cas (hyperthreading). Dans ce cas la micro-architecture
% partage les blocs d'un même coeur et offre l'illusion aux composants logiciels
% qu'ils peuvent s'exécuter sur différents coeurs indépendants. Chaque coeur
% virtuel possède un état qui lui est propre (compteur ordinal, etc.) mais
% partage les blocs d'exécution avec les autres coeurs virtuels d'un même coeur
% physique). Mettre quelques schéma pour illustrer ces concepts.
%
% A core basically repeats the following tasks:
%%
% \begin{enumerate}
% \item Read the content of its (internal) \emph{program counter} register, and
%   interpret it as the address of the next instruction.
%%
% \item Fetch the content of this address.
%%
% \item Decode the instruction, that is identified the desired operation to
%   perform
%%
% \item Act accordingly, by modifying its internal state, and interacting with
%   memories and features exposed by other hardware components
%%
% \item Update the content of its \emph{program counter} register according to
%   the semantics of the instruction it has executed. Most of the time, it
%   increases it to fetch the next instruction, but it can also set it to an
%   arbitrary value (\emph{e.g.} with jump instructions).
% \end{enumerate}

\subsection{Memories and Cores \IOs}
\label{subsec:usecase:mem}
%
Besides cores, \emph{memories} are the most important components of a computing
platform.
%
Cores interact with these memories during so-called \IOs, for Input/Output: a
core receives data during an input and it sends data during an output.
%
The x86 architecture integrates several sources of memories, along with several
mechanisms that can be used by the cores to interact with these memories.

The main target of cores \IOs is the \ac{dram}.
%
\ac{dram} contains the instructions executed by the cores, \emph{and} the data
they manipulate.
%
A core decides the semantics of a given memory cell depending on the context,
\emph{e.g.}  the same binary sequence can be decoded as an instruction or
interpreted as an operand of an arithmetic operation.
%
In addition, other hardware components also provide additional memory regions
that cores can read from or write to.
%
Contrary to \ac{dram}, \IOs targeting these memories often carry a semantics
specific to each peripheral.
%
For instance, an x86 processor integrates a display controller which exposes a
frame buffer to the cores.
%
By writing to the frame buffer, a core changes the pictures displayed on the
computer screen.

\paragraph{Address Spaces.}
%
Cores interact with memories \emph{via} two distinct memory address spaces,
characterized by a set of addresses and a set of instructions.
%
The most important address space is the \emph{system memory}, and most of the
x86 instructions (\emph{e.g.} \texttt{mov} variants, arithmetic operations
such as \texttt{sub} and \texttt{add}) are designed to manipulate it.
%
Addresses of the system memory are referred to as physical addresses and the
majority of system memory \IOs are dispatched to the \ac{dram}.
%
Besides, cores use another address space characterized by two dedicated
instructions ---\texttt{in} and \texttt{out}--- to target the memories exposed
by other hardware components.
%
\IOs issued by \texttt{in} and \texttt{out} instructions are referred to as
Port-Mapped \IOs (PMIO), and the addresses of this address space are called
\emph{ports}.

Although they are historically used to target different memory regions, nowadays
these two address spaces overlap, as peripheral memories and registers can be
exposed to the cores \emph{via} the system memory thanks to \emph{memory-mapped}
\IOs.
%
That is, \emph{it is possible to read from or write to the same memory location
  using two different address spaces}.
%
% PC: “a memory mapping” plutôt ? La “memory map” c'est l'ensemble de tous les
% mappings.
%
% Thomas: justement, c’est bien dans ce sens que je l'utilise. J'ai mis les
% termes au pluriel pour que ce soit plus clair, c'est mieux ?
%
The mapping between addresses of the system memory and their concrete memory
locations within the hardware architecture is called the \emph{memory map}.
%
This memory map is configurable, that is it can be changed dynamically
\emph{via} configuration registers exposed by the processor and the \ac{pch}.

For instance, the Peripheral Component Interconnect (PCI) standard ---whose aim
is to propose a standard local bus and communication protocol to connect
hardware components to a computing platform--- introduces a so-called
configuration space per PCI device, which is a dedicated memory region with a
specific semantics summarized in Figure~\ref{fig:usecase:pciconfig}.
%
Registers of the PCI configuration space, such as device and vendor IDs, can be
accessed \emph{via} PMIO.
%
Indeed, the \ac{pch} exposes two ports to that end.
%
First, software components modify the content of the
\texttt{PCI\_CONFIG\_ADDRESS} port, to tell the \ac{pch} which PCI configuration
register they want to interact with.
%
Then, they read from and write to the \texttt{PCI\_CONFIG\_DATA} port, and the
\ac{pch} dispatches these \IO to the targeted register.
%
The PCI specification states that the offset \texttt{0x10} of the PCI
configuration spaces is dedicated to so-called \texttt{BAR}s (Base Address
Registers).
%
The purpose of the \texttt{BAR}s is to configure a memory-mapped mechanism, so
that it becomes possible to interact with the PCI configuration space of a given
peripheral \emph{via} the system memory.
%
As a consequence, a core which reads from or write to the system memory in a
range specified by a BAR sees its \IO dispatched to the configuration space of
the related PCI device.

\begin{figure}
  \begin{center}
    \def\svgwidth{0.8\textwidth} \resizebox{0.6\textwidth}{!}{%
      \input{Figures/PCI-config-space.pdf_tex}}
  \end{center}
  \caption{Standard registers of PCI Type 0 (Non-Bridge) Configuration Space
    Header}
  \label{fig:usecase:pciconfig}
\end{figure}

Finally, software components often do not manipulate addresses of the system
memory directly.
%
Indeed, cores have their own address translation mechanisms, namely
segmentation\,\cite[Volume 3, Section 2.4]{intel2014manual} and
pagination\,\cite[Volume 3, Chapter 4]{intel2014manual} (potentially extended
with its virtualization technology\,\cite[Volume 3, Section
28.2]{intel2014manual}), which are configurable by the software components.

As such, determining which hardware component will handle a core \IO targeting a
given virtual address requires to have a complete knowledge of the x86 remapping
and virtual memory mechanisms and of their exact configurations at a given time.

\paragraph{Access Control for Software Components.}
%
The x86 hardware architecture provides a rich collection of hardware
features\,\cite[Volume 3]{intel2014manual} to implement fine-grained access
control policies about software components \IOs.
%
Subjects of these policies include the software and hardware components of the
system.
%
Objects ultimately come down to the memory locations of various natures
scattered within the hardware architecture.
%
Actions comprise reading from and writing to a memory location. From a core
perspective, it is also common to distinguish between reading data and reading
instructions.

For instance, the \ac{mmu} is probably the most well-known hardware feature to
implement an access control policy.
%
Thanks to the \ac{mmu}, an operating system can attribute ranges of \ac{dram} to
user applications it manages and isolate its code and data from these
applications.
%
The \ac{mmu} alone is not sufficient, because its scope does not cover its own
configuration.
%
That is, it is not possible to configure the \ac{mmu} in order to prevent a
software component to modify the \ac{mmu} configuration.
%
As a consequence, an additional hardware feature has to be used: the protection
rings\,\cite[Volume 3, Section 5.5]{intel2014manual}.
%
The x86 cores can operate in 4 different so-called rings, from 0 to 3, where
ring~0 is the most privileged and ring~3 the least.
%
Ring~3 imposes several restrictions on software components, including the
capability to modify the \texttt{CR3} register which identifies the base of the
page table hierarchy used by the \ac{mmu}.
%
This is why ring~3 is commonly dedicated to the execution of applications.

%
% GUILLAUME: Ce qu'il faudrait dire, c'est que le CPU et la RAM sont des
% ressources partagées par les différents composants logiciels et matériels. Du
% coup, d'un point de vue de la sécurité, comme ces composants ne se font pas
% nécessairement confiance (parsqu'ils viennent de fournisseur différent,
% parqsu'il peuvent être malveillant ou vulnérables, etc.) il est nécessaire de
% les isoler les un les autres. La plupart des mécanismes de sécurité de la
% plateforme fournisse don c un mécanisme d'isolation}
%

The complexity of the x86 \IO resolution mechanism requires to take into account
the numerous redirection features exposed by the architecture.
%
Memory locations can have an arbitrary number of aliases, in several layered
address spaces: the \ac{dram} controller assigns an address to each memory cell
the \ac{dram} contains; the processor maps physical addresses to \ac{dram}
addresses; the \ac{mmu} maps virtual addresses to physical addresses.
%
As a consequence, modifying the content of a memory cell may not be the only way
at the disposal of attackers to defeat a given access control policy.
%
For instance, if the access control policy refers to virtual addresses \( v \),
modifying the \ac{mmu} configuration results in modifying the content associated
with \( v \).

\subsection{Cache Memory}
\label{subsec:usecase:cachemem}
%
Interacting with the \ac{dram} remains slow, in regard to the speed of cores.
%
To improve performance, Intel processors come with several levels of caches,
from the smaller and quicker, to the bigger and slower.
%
For instance, Intel Core i7, i5 and i3 processors have three levels of caches.
%
Each core is assigned two levels of cache called L1 ---which has the
particularity of being divided into a cache of instructions (used when the core
read from memory instructions to execute) and a cache of data --- and L2, while
they all share a so-called L3 cache divided into interconnected slices.
%
The Figure~\ref{fig:usecase:l1-l3caches} summarizes this organization.

Caches are divided into several \emph{cache blocks} addressed by an index.
%
Cache blocks are divided into several \emph{cache lines}, which are tagged with
a memory address and contains a copy of the data stored at this address.
%
A cache is characterized by the number of cache lines per cache block, the
nature of the address associated with cache block indexes, and the nature of the
address used to tag cache line.
%
For instance, Intel L2 and L3 caches are \emph{physically-indexed,
  physically-tagged}, meaning they use physical address to compute both the
index and the tag.

Caches are mostly transparent to the software components.
%
For instance, the processor alone enforces the cache coherence, with the notable
exception of multiprocessor (not multicore) systems, for which ``maintenance of
cache consistency may, in rare circumstances, require intervention by system
software.''
%
Intel has developed a dedicated protocol called MESIF to that
end\,\cite{thomadakis2011nehalem}.

\begin{figure}
  \begin{center}
    \begin{tikzpicture}
      \node [draw, inner sep=10pt, rectangle, text width=100pt, text badly
      centered] (l3-1) {L3 slice};%
      \node [draw, inner sep=10pt, rectangle, text width=100pt, text badly
      centered, right=25pt of l3-1] (l3-2) {L3 slice};%
      \node [draw, inner sep=10pt, rectangle, text width=100pt, text badly
      centered, below=10pt of l3-2] (l3-3) {L3 slice};%
      \node [draw, inner sep=10pt, rectangle, text width=100pt, text badly
      centered, left=25pt of l3-3] (l3-4) {L3 slice};%
      \node [draw, dashed, inner sep=10pt, fit=(l3-1) (l3-2) (l3-3) (l3-4)] (l3)
      {};%

      \draw (l3-1) -- (l3-2);%
      \draw (l3-2) -- (l3-3);%
      \draw (l3-3) -- (l3-4);%
      \draw (l3-4) -- (l3-1);%

      % core 1
      \node [draw, text width=100pt, inner sep=10pt, text badly centered,
      above=of l3-1] (l2-1) {Cache L2};%
      \draw (l3-1) -- (l2-1);%

      \node [above=20pt of l2-1] (l1-1) {};%
      \node [draw, right=0pt of l1-1.center, text badly centered, text
      width=50pt, inner sep=5pt] (l1-1-i) {Cache L1 {\small (Instruction)}};%
      \node [draw, left=0pt of l1-1.center, text badly centered, text
      width=50pt, inner sep=5pt] (l1-1-d) {Cache L1 {\small (Data)}};%
      \node [draw, text width=100pt, text badly centered, inner sep=10pt,
      minimum height=40pt, above=20pt of l1-1] (unit-1) {Execution Unit};%

      \draw (unit-1.south) -| (l1-1-d.north);%
      \draw (unit-1.south) -| (l1-1-i.north);%
      \draw (l1-1-i.south) |- (l2-1.north);%
      \draw (l1-1-d.south) |- (l2-1.north);%

      \node [draw, dashed, inner sep=10pt, fit=(unit-1) (l2-1)] (c-1) {};%

      % core 2
      \node [draw, text width=100pt, inner sep=10pt, text badly centered,
      above=of l3-2] (l2-2) {Cache L2};%
      \draw (l3-2) -- (l2-2);%

      \node [above=20pt of l2-2] (l1-2) {};%
      \node [draw, right=0pt of l1-2.center, text badly centered, text
      width=50pt, inner sep=5pt] (l1-2-i) {Cache L1 {\small (Instruction)}};%
      \node [draw, left=0pt of l1-2.center, text badly centered, text
      width=50pt, inner sep=5pt] (l1-2-d) {Cache L1 {\small (Data)}};%
      \node [draw, text width=100pt, text badly centered, inner sep=10pt,
      minimum height=40pt, above=20pt of l1-2] (unit-2) {Execution Unit};%

      \draw (unit-2.south) -| (l1-2-d.north);%
      \draw (unit-2.south) -| (l1-2-i.north);%
      \draw (l1-2-i.south) |- (l2-2.north);%
      \draw (l1-2-d.south) |- (l2-2.north);%

      \node [draw, dashed, inner sep=10pt, fit=(unit-2) (l2-2)] (c-2) {};%

      % core 3
      \node [draw, text width=100pt, inner sep=10pt, text badly centered,
      below=of l3-3] (l2-3) {Cache L2};%
      \draw (l3-3) -- (l2-3);%

      \node [below=20pt of l2-3] (l1-3) {};%
      \node [draw, right=0pt of l1-3.center, text badly centered, text
      width=50pt, inner sep=5pt] (l1-3-i) {Cache L1 {\small (Instruction)}};%
      \node [draw, left=0pt of l1-3.center, text badly centered, text
      width=50pt, inner sep=5pt] (l1-3-d) {Cache L1 {\small (Data)}};%
      \node [draw, text width=100pt, text badly centered, inner sep=10pt,
      minimum height=40pt, below=20pt of l1-3] (unit-3) {Execution Unit};%

      \draw (unit-3.north) -| (l1-3-d.south);%
      \draw (unit-3.north) -| (l1-3-i.south);%
      \draw (l1-3-i.north) |- (l2-3.south);%
      \draw (l1-3-d.north) |- (l2-3.south);%

      \node [draw, dashed, inner sep=10pt, fit=(unit-3) (l2-3)] (c-3) {};%

      % core 3
      \node [draw, text width=100pt, inner sep=10pt, text badly centered,
      below=of l3-4] (l2-4) {Cache L2};%
      \draw (l3-4) -- (l2-4);%

      \node [below=20pt of l2-4] (l1-4) {};%
      \node [draw, right=0pt of l1-4.center, text badly centered, text
      width=50pt, inner sep=5pt] (l1-4-i) {Cache L1 {\small (Instruction)}};%
      \node [draw, left=0pt of l1-4.center, text badly centered, text
      width=50pt, inner sep=5pt] (l1-4-d) {Cache L1 {\small (Data)}};%
      \node [draw, text width=100pt, text badly centered, inner sep=10pt,
      minimum height=40pt, below=20pt of l1-4] (unit-4) {Execution Unit};%

      \draw (unit-4.north) -| (l1-4-d.south);%
      \draw (unit-4.north) -| (l1-4-i.south);%
      \draw (l1-4-i.north) |- (l2-4.south);%
      \draw (l1-4-d.north) |- (l2-4.south);%

      \node [draw, dashed, inner sep=10pt, fit=(unit-4) (l2-4)] (c-4) {};%
    \end{tikzpicture}
  \end{center}
  \caption{Typical caches organization of a x86 processor}
  \label{fig:usecase:l1-l3caches}
\end{figure}

When a core successfully reads the memory at a given address, it keeps a copy of
the result in its caches.
%
Therefore, the next time it needs to read some data at this address, these data
are retrieved from the cache.
%
Regarding write accesses, Intel x86 processors provide five different caching
strategies (uncacheable, write combining, write-through, write-back and
write-protected)\,\cite[Volume 3, Chapter~11]{intel2014manual}.
%
Fine-grained cache strategy configuration is achieved through several hardware
mechanisms, including (but not limited to):
%
\begin{itemize}
\item The \texttt{CR0} register has a flag called \texttt{CD}, which enables
  caching once set.
%
\item The processor has several registers called Memory Type Range Registers
  (MTRR), to specify a cache strategy for pre-defined memory regions.
%
\item The Page Table Attribute (PAT) allows for configuring a cache strategy at
  a memory page granularity.
\end{itemize}
%
The write-back strategy is the most commonly used and is summarized in
Figure~\ref{fig:usecase:writeback}.
%
The purpose of this strategy is to reduce the number of \IO forwarded to the
\ac{dram}.
%
To that end, each cache line has a ``dirty bit'' which is set by the cache when
its value is updated by a write access.
%
A \emph{cache eviction} occurs when the cache frees a given cache line
previously used for a given address in order to use it for another one.
%
When it happens, the cache verifies the value of the ``dirty bit'', so it can
update the underlying memory cell if necessary.
%
Therefore, as long as the cache line is not evicted, the processor does not
issue write access to the underlying memory.
%
The write-back strategy is not suitable for memories of other hardware
components mapped into the system memory.
%
For instance, caching \IOs targeting a framebuffer does not make sense, because
the screen wouldn't be updated.
%
This is why Intel provides four complementary strategies.
%
The uncacheable strategy disables the cache for the given address, while write
combining, write-through, write-back and write-protected are similar strategies
such that copies of the underlying memory are stored in the cache to accelerate
read accesses, but write accesses are directly forwarded.

% For cache-friendly programs, the gain in performance can be
% huge. \TODO{utilité de ce dernier paragraphe? en outre, il n'est pas vraiment
% raccord avec ce qui précède.}
%%
% For instance, for a Pentium M Intel stated that access to L1 cache takes 3
% \ac{cpu} cycles, access to the L2 cache takes 12 cycles and access to the DRAM
% takes 240 cycles\,\cite{drepper2007memory}.

\begin{figure}
  \centering
  \begin{tikzpicture}
    \node [draw, circle] (EP) {};%

    \node [draw, below=of EP] (Sel) {Select a cache line};%
    \draw [-latex] (EP) -- (Sel);%

    \node [draw, signal, signal to=east and west, below=of Sel] (CH) {Cache
      hit?};%
    \draw [-latex] (Sel) -- (CH);

    \node [below=of CH] (CHbranch) {};%
    \draw (CH) to node [right] {[Yes]} (CHbranch.center);%
    \node [left=70pt of CH] (nCHbranch) {};%
    \draw [below] (CH) to node {[No]} (nCHbranch.center);

    \node [draw, signal, signal to=east and west, below=of nCHbranch.center]
    (Dirty) {Cache line dirty?};%
    \draw [-latex] (nCHbranch.center) -- (Dirty);%

    \node [draw, below=of Dirty, text width=70pt, text badly centered] (WB)
    {Write the content of the cache line back to the DRAM};%
    \draw [-latex] (Dirty) to node [right] {[Yes]} (WB);%

    \node [draw, below=of WB, text width=80pt, text badly centered] (Rf) {Read
      data from lower memory and fill the cache line};%
    \draw [-latex] (WB) -- (Rf);%

    \node [draw, below=of WB, text width=80pt, text badly centered] (Rf) {Read
      data from lower memory and fill the cache line};%
    \draw [-latex] (WB) -- (Rf);%

    \node [draw, below=of Rf, text width=80pt, text badly centered] (NotDirty)
    {Mark the cache line as ``not dirty''};%
    \draw [-latex] (Rf) -- (NotDirty);%

    \draw (Dirty) to node [below] {[No]} ([xshift=-90pt]Dirty.center);%
    \draw ([xshift=-90pt]Dirty.center) -- ([xshift=-90pt]Rf.center);%
    \draw [-latex] ([xshift=-90pt]Rf.center) -- (Rf);%
    \draw [-latex] (Rf) -- (NotDirty);%

    \node [below=260pt of CHbranch.center] (Join) {};%
    \node [draw, signal, signal to=east and west, below=5pt of Join] (RxW) {Read
      or Write?};%
    \draw [-latex] (CHbranch.center) -- (RxW);%
    \draw (NotDirty) -| (Join.center);%

    \node [draw, text badly centered, text width=70pt, right=50pt of RxW] (R)
    {Return data};%
    \draw [-latex] (RxW) to node [below] {[Read]} (R);%
    \node [draw, text badly centered, text width=70pt, left=50pt of RxW] (W)
    {Write new value in cache line};%
    \draw [-latex] (RxW) to node [below] {[Write]} (W);%

    \node [draw, text badly centered, text width=70pt, below=of W] (MakeDirty)
    {Mark cache line as dirty};%
    \draw [-latex] (W) -- (MakeDirty);

    \node [draw, fill=black, below=100pt of RxW, circle] (End) {}; \draw
    [-latex] (MakeDirty.south) |- (End.west); \draw [-latex] (R.south) |-
    (End.east);
  \end{tikzpicture}

  \caption{The Write-Back cache strategy}
  \label{fig:usecase:writeback}
\end{figure}

\subsection{Peripherals \IOs}
\label{subsec:usecase:periphio}

Cores are not the only active hardware components present inside a typical x86
hardware architecture.
%
For instance, several hardware components can also read from or write to the
\ac{dram} using a technology called \ac{dma}.
%
Hardware components can also interact with the processor by sending hardware
interrupts of various natures.
%
When a user presses a key of its keyboard, the latter sends an interrupt
request.
%
Interrupt handlers, that is programs executed by the core when it receives
interrupts, are configurable \emph{via} a so-called \ac{idt}\,\cite[Volume 3,
Chapter 6]{intel2014manual}.
%
Each line of the \ac{idt} corresponds to a given interrupt whose semantics is
specified by Intel, as summarized in Table~\ref{tab:usecase:idt}.
%
When a core handles an interrupt, it saves its current context inside the
\ac{dram}, then starts executing the corresponding interrupt handler.
%
Not all x86 interrupts come from a hardware component.
%
Cores use several of them, for instance to recover from errors.
%
For example, if a core is not able to translate a virtual address into a
physical address, it raises a so-called page fault.

\begin{table}
  \centering
  \begin{tabular}{cl}
    {\scshape \#IRQ}
    & \multicolumn{1}{c}{\scshape Semantics} \\
    \hline
    \texttt{0x00}
    & Division by zero \\
    \texttt{0x01}
    & Single-step interrupt \\
    \texttt{0x02}
    & Non-Maskable Interrupt (NMI) \\
    \texttt{0x03}
    & Breaking point (used by debuggers) \\
    \texttt{0x04}
    & Stack overflow \\
    \texttt{0x05}
    & Bounds \\
    \texttt{0x06}
    & Invalid instruction opcode \\
    \texttt{0x07}
    & Coprocessor not available \\
    \texttt{0x08}
    & Double fault \\
    \texttt{0x09}
    & Coprocessor segment overrun \\
    \texttt{0x0A}
    & Invalid task state segment \\
    \texttt{0x0B}
    & Segment not present \\
    \texttt{0x0C}
    & Stack fault \\
    \texttt{0x0D}
    & General protection fault \\
    \texttt{0x0E}
    & Page fault \\
    \texttt{0x0F}
    & Reserved by Intel \\
    \texttt{0x10}
    & Math fault \\
    \texttt{0x11}
    & Alignment check \\
    \texttt{0x12}
    & Machine check \\
    \texttt{0x13}
    & SIMD floating-point exception \\
    \texttt{0x14}
    & Control protection exception \\
  \end{tabular}
  \caption{x86 Interrupt Descriptor Table semantics}
  \label{tab:usecase:idt}
\end{table}

\paragraph{Access Control for Hardware Components.}
%
Hardware components come from various places and can be of various qualities.
%
Once integrated together, an attacker can potentially leverage any of them to
threaten the security of the system.
%
In this context, the principle of least
privilege\,\cite{saltzer1975leastprivilege} applies: a given component should
only be able to leverage capabilities it needs to work according to its purpose,
where a ``capability'' refers to the right to perform a given \IO.
%
To impose an access control policy to hardware components, a system software
component uses the so-called VT-d feature, which implements an \IO-\ac{mmu} for
x86 computing platform\,\cite{abramson2006vtd}.

In practice, the correct enforcement of an access control policy for hardware
components remains challenging, for various reasons.  Firstly, security checks
can have an important impact on performance.
%
Partly for this reason, the hardware components have long been assumed
trustworthy.
%
A good demonstration of this fact is the Address Translation Services mechanism
introduced by the PCI standard, whose purpose is to allow PCI devices to bypass
security mechanisms designed to reduce their
privileges\,\cite{daubignard2017protip}.
%
Other examples showed blind trust in foreign hardware components (\emph{e.g.}
USB devices) is not without consequences from a security
perspective\,\cite{nohl2014badusb,hudson2015thunderstrike,chifflier2013uefi}.
%
Secondly, the ``least privilege'' may vary from one execution to another,
\emph{e.g.} depending on the software stack executed.
%
To handle the numerous use cases of the x86 architecture, its default
configuration is very permissive, until software components such as the
\ac{bios} or an operating system modify it to fit their needs, by implementing
\ac{hse} mechanisms.

% \paragraph{Autonomous Subsystem.}
% %
% Intel has implemented several out-of-band management technologies for its
% products.
% %
% They allow for administering a computer \emph{via} the network, without the
% need for a physical access.
% %
% In practice, the software components responsible for implementing the
% out-of-band management features are not executed by the main processor: some
% network cards implement the ASF technology\,\cite{duflot2010network}, and
% since 2008, the x86 hardware architecture comprises a so-called Management
% Engine to that end\,\cite{ruan2014me,skochinsky2014intel}.
% %
% In 2018, the Management Engine is located inside the \ac{pch}.
% %
% ASF-capable network cards and the Management Engine require important
% capabilities to implement their features.
% %
% For instance, the Management Engine can download from the Internet an
% executable image, provoke the reboot of the computing platform and force the
% processor to execute the software components within the image it has
% downloaded\,\cite{kumar2009active}.

\subsection{Conclusion}

This introduction to the x86 hardware architecture provides the necessary
background to understand the challenges related to the implementation of a
\ac{hse} mechanism in general.
%
In the next section, we explain why the \ac{bios} requires such an isolated
environment to operate, and we detailed the hardware features it leveraged to
that end.

\section{BIOS Overview}
\label{sec:usecase:firmware}

The \ac{bios} plays a significant role in Intel x86 computing platform.
%
It is the first piece of software executed by the processor, which initializes
the hardware components and initiates the execution of the software stack during
the boot sequence (\ref{subsec:usecase:firm:boot}).
%
At runtime, it remains active to perform various tasks, including and not
limited to platform-specific events, device emulation, or \ac{bios} updates
management (\ref{subsec:usecase:firm:runtime}).
%
As such, it can only operate properly if certain security requirements are met
and implements several \ac{hse} mechanisms to that end
(\ref{subsec:usecase:firm:sec}).

\subsection{During the boot sequence}
\label{subsec:usecase:firm:boot}

The \ac{bios} program is stored inside a small flash memory connected to the
\ac{pch} through the Serial Peripheral Interface (SPI) bus on modern x86
computing platform.
%
When the computing platform is powered up, the processor starts executing the
code stored at a hard-coded address within the flash memory.
%
The first task of the \ac{bios} is to initialize the hardware
architecture\,\cite{salihun2006bios}.
%
Then, the \ac{bios} searches for a system software component to load into
memory.
%
Historically, ``legacy'' \acp{bios} were looking for a Master Boot Record (MBR)
at the beginning of mass storage devices (\emph{e.g.} hard drive, USB stick).
%
The MBR, whose size is limited to 512 bytes, contains a small program to
initiate a loader for a system software component.
%
Modern \acp{bios} implement the \ac{uefi}\,\cite{zimmer2007uefi,uefi2017specs}
standard, whose purpose is to standardize the boot sequence process in order to
favor interoperability of \ac{bios} implementations.
%
The boot sequence is divided into several phases, and the \ac{bios} is packaged
into several software components accordingly.
%
In particular, \ac{uefi}-compliant \ac{bios} can load so-called \ac{uefi}
applications of arbitrary size, leading modern hypervisors and operating systems
to be packaged as \ac{uefi} applications\,\cite{2011efistub}.

Because the \ac{bios} is the first software component executed by the hardware
architecture, and is responsible for initiating the execution of following
software components (\emph{e.g.} an operating system), it is commonly designated
as the root of trust\,\cite{rutkowska2015intel} for the software stack.
%
As such, the integrity of the \ac{bios} code is critical, and several strategies
have been proposed to detect \ac{bios} code corruption during the boot sequence,
with the two most predominant being Secure Boot\,\cite{rosenbaum2012secboot} and
Trusted Boot\,\cite{trustedboot}.
%
Secure Boot and Trusted Boot can uncover certain \ac{bios} corruptions prior to
the execution of the illegitimate code.
%
However, they both rely on a so-called \emph{root of trust}, which is the
initial code of the \ac{bios}, whose integrity cannot be guaranteed for certain
because it is not stored on a read-only memory.
%
Recent efforts have been expended to overcome this limitation.
%
For instance, in 2013 HP has introduced a security mechanism called
SureStart\,\cite{hp2016surestart} whose purpose is to move the root of trust
within another hardware component, leaving most attackers unable to modify its
code.
%
More recently, the NIST has published the Special Communication 800-193
---\emph{Platform Firmware Resiliency
  Guidelines}\,\cite{regenscheid2018nist800193}--- which specifically tackles
the challenge posed by illegitimate firmware modification.
%
These approaches have in common to aim at enforcing the correct initialization
of the platform by the \ac{bios}.

% \TODO{En quoi Secure Boot et Trusted Boot permettent de détecter des
% corruptions du BIOS?}  \TODO{Cette sous-section n'est pas censé parler de
% sécurité (si j'ai bien compris le plan çà vient en 2.2.3) mais finalement tu
% t'attardes beaucoup à décrire des méanismes de sécurité. Ce n'est peut-être
% pas le bon endroit }
% %
% The \ac{uefi} standard defines a security mechanism called Secure
% Boot\,\cite{rosenbaum2012secboot}.
% %
% When Secure Boot is enabled, \ac{uefi}-compliant \acp{bios} should only
% execute applications which provide valid cryptographic signature, with respect
% to a key hierarchy. \TODO{Il faudrait développer un peu en expliquant que le
% BIOS contient les clés publiques des fournisseurs de code supposé de
% confiance. D'ailleurs, peut-on ajouter des clés? Cela permet de s'assurer que
% seules des applications UEFI provenant de ces fournisseurs seront chargé puis
% exécuté par le BIOS. Si je ne me trompe pas, la verification à lieu lors du
% chargement (donc c'est le binaire qui est vérifié). }
% %
% The \ac{tcg} has standardized a hardware component called the \ac{tpm}, which
% is the foundation of the Trusted Boot. \TODO{Il faudrait mieux marquer la
% séparation avec le Secure Boot. On a l'impression que tu continues d'expliquer
% le Secure Boot.}
% %
% Each component of the boot sequence should measure any software component it
% initiates prior to starting its execution, and to entrust these measurements
% to the \ac{tpm}.
% %
% If one of the software components executed during the boot sequence has been
% corrupted, the difference is captured by the \ac{tpm}, because its
% measurements are different than expected. \TODO{Ca veut donc dire qu'il existe
% une référence. Comment cette référence est-elle injectée dans le TPM? Qui peut
% le faire? Il faudrait commencer par expliquer çà.}
% %
% The measurements entrusted to the \ac{tpm} can be leveraged at the end of the
% boot sequence in at least two ways.
% %
% The \ac{tpm} can cryptographically sign the measurements during a protocol
% called remote attestation\,\cite{coker2011remoteattestation} which allows a
% third party to verify that a given computing platform executes the code it is
% supposed to.
% %
% Another popular approach is to entrust to the \ac{tpm} the encryption key
% which protects sensitive information that a corrupted \ac{bios} should not be
% allowed to access.
% %
% In this scenario, the access to the encryption key is correlated to the
% measurements received by the \ac{tpm}.

\subsection{At runtime}
\label{subsec:usecase:firm:runtime}

The boot sequence ends once a system software component has been selected and
loaded into memory by the \ac{bios}.

\paragraph{Software Interfaces.}
At runtime, the \ac{bios} provides various software interfaces to the system
software component.
%
For instance, the \ac{acpi} tables\,\cite{uefi2017acpi,duflot2010acpi} is a
standardized interface to configure various vendor-specific aspects of the
hardware platform, such as power management or thermal management.
%
Similarly, legacy \acp{bios} expose facilities to system software components, in
the form of so-called \ac{bios} Interrupt.
%
For instance, the interrupt \texttt{0x10} is dedicated to video services
(\emph{e.g.} setting the video mode, setting the cursor shape and position,
etc.).
%
Nowadays, \ac{uefi}-capable \acp{bios} expose so-called \emph{Runtime Services}
to system software component\,\cite[Chapter 5]{zimmer2017uefi} under the form of
a table of function pointers.

In either case, these interfaces act as an intermediary layer between a system
software component and the hardware architecture.
%
In doing so, they reduce the coupling between the software and hardware
components.
%
Sometimes, their use is optional, and \acp{bios} only provide them as a
facility.
%
Other are mandatory gates towards certain computing platform features, because
they are related to critical mechanisms of the platform and the hardware vendors
do not want to rely on a (potentially vulnerable or malicious) system software
component.
%
For instance, the \ac{bios} takes care of its own software updates, in order to
verify submitted versions prior to applying them, \emph{e.g.} by verifying
cryptographic signature or preventing the installation of older, outdated
versions.

\paragraph{Proactive Features.}
%
In addition to supporting the execution of the rest of the software stack
through its interfaces, the \ac{bios} carries out several hardware-specific
tasks which are not publicly documented.
%
This includes and is not limited to handling hardware errors, checking thermal
zones, adjusting cores speed, configuring hardware workarounds, and emulate
complete hardware devices to the system software
component\,\cite{yao2009system}.

The execution of the \ac{bios} in this context should be transparent to the rest
of the software stack.
%
As such, the \ac{bios} remains the most privileged software component of the
software stack, even after the end of the boot sequence.

\subsection{HSE Mechanisms Implemented by the BIOS}
\label{subsec:usecase:firm:sec}

The \ac{bios} is provided by the manufacturer of the hardware architecture.
%
In most cases, it is a proprietary software, and the computer owner has little
control over it.
%
The rest of the software stack is considered untrusted, and one goal of the
\ac{bios} is to keep the computer in a working state, even in the presence of an
erroneous or malicious software stack.
%
To that end, the \ac{bios} relies on several \ac{hse} mechanisms to enforce its
isolation from the rest of the software stack.
%
% To the best of our knowledge, the security policy targeted by the \ac{bios} as
% runtime is always discussed through the prism of the \ac{hse} mechanism used
% to enforced this policy.
%%
% \PC{la phrase précédente est un peu bizarre: la politique de sécurité est
% discutée (?!) à travers le prisme \ldots}
%%
% That is, documentation such as the Intel manual, the processor datasheet or
% the \ac{pch} datasheet ---in practice involved
%%
% \PC{pas clair: c'est la documentation qui est impliquée dans le hse ?!}
%%
% in the \ac{hse} mechanisms implemented by the \ac{bios}--- focus on \emph{how}
% implementing the security policy rather than specifying \emph{what} this
% policy is\thomasrk{Is it really true?}.
%%
% \PC{le paragraphe précédent/en cours me semble à retravailler, je ne le trouve
% pas clair}
%%
% \PC{La phrase qui suit fait un peu ``citation en vrac''}
%%
% Another useful source of information on this matter is the various research
% projects which discuss these mechanisms, for instance to demonstrate a
% security
% vulnerability\,\cite{duflot2009smram,wojtczuk2009smram,bulygin2014summary}.
%
From the information we gathered in the Intel manual\,\cite{intel2014manual},
datasheets\,\cite{intel2009mch,intel2012pch}, and in the academic
literature\,\cite{bulygin2014summary}, the isolation required by the \ac{bios}
can be divided into three complementary security policies.

\begin{description}
\item [Volatile Memory Access Control]
  %
  The \ac{bios} is assigned a region of the volatile memory to support its
  execution at runtime. This region is protected against \IOs issued by the rest
  of the software stack.
  %
\item [Availability]
  %
  The rest of the software stack is not authorized to prevent the execution of
  the \ac{bios} at runtime, \emph{i.e.} the \ac{bios} can preempt the execution
  of the rest of the software stack.
  %
\item [Non-volatile Memory Access Control]
  %
  The \ac{bios} is assigned a non-volatile memory region (in practice, a portion
  of the flash memory) to store its code and data.
  %
  The rest of the software stack is not authorized to modify the content of this
  memory, in order to avoid a scenario where attackers modify the \ac{bios} code
  according to their needs and provoke a reboot of the platform.
\end{description}

These three security policies are implemented by the means of \ac{hse}
mechanisms which rely on hardware features exposed by the processor and the
\ac{pch}.
%
% \TODO{il manque ici une phrase de transition avec la suite /annonce de plan}

\paragraph{System Management Mode.}
%
To handle several software components with different levels of privilege, Intel
processors provide several execution modes, which can be assimilated to sets of
hardware capabilities.
%
For instance, in a given execution mode, a core may refuse to execute certain
assembly instructions.
%
Contrary to common belief, x86 execution modes are not organized in a linear
hierarchy, but are rather a matrix of complementary hardware features:
protection rings, paging configuration, virtualization technologies, etc.
%
As for the \ac{bios}, Intel provides the so-called \ac{smm}\,\cite[Volume 3,
Chapter 34]{intel2014manual}, introduced in the Intel manual as follows:

% GUILLAUME: \TODO{ ref, ref and ref...}  NOTE: On a déjà évoqué plusieurs fois
% /avant/ ces technos, donc si ref il doit y avoir (et je crois qu'elles y
% sont), alors elles sont déjà présentes avant.

\begin{quote}
  \ac{smm} is a special purpose operating mode provided for handling systemwide
  functions like power management, system hardware control, or proprietary
  OEM-designed code.
  %
  It is intended for use only by system firmware, not by application software or
  general-purpose system software.
  %
  The main benefit of \ac{smm} is that it offers a distinct and easily isolated
  processor environment that operates transparently to the operating system or
  executive and software applications.

  \hfill \small \emph{Intel 64 and IA-32 Architectures Software Developer’s
    Manual}
\end{quote}

The \ac{smm} is the foundation of the \ac{bios} isolation at runtime, but it is
not sufficient.

\paragraph{System Management RAM.}
%
The SMRAM is the name given by Intel to a memory region located inside the
\ac{dram}, and dedicated to the \ac{smm}.
%
The exact location and size of the SMRAM are architecture dependent.
%
To locate it, the processor uses a dedicated register named \texttt{SMBASE}.
%
The \ac{bios} should configure it during the boot sequence.
%
As its name suggests, the \texttt{SMBASE} value should point to the base of the
SMRAM.
%
As for the end of the SMRAM, the hardware architecture does not expose it
explicitly, and the \ac{bios} developers need to refer to the processor
datasheet in order to find it.

At the beginning of the boot sequence, the SMRAM is left unprotected, meaning
arbitrary memory accesses targeting the SMRAM are authorized.
%
This design allows the \ac{bios} to initialize the SMRAM content.
%
Once the \ac{smm} code ---the \ac{bios} code intended to be executed at runtime
in \ac{smm}--- has been correctly loaded into the SMRAM, and prior to starting
the execution of a system software component, the \ac{bios} has to lock the
SMRAM.
%
% \TODO{expliquer ce que veut dire invisible -> la mémoire n'est plus mappé dans
% la plage d'adresse et soit le processeur accède à une autre mémoire soit il y
% a une exception (je suppose)}
A locked SMRAM can only be accessed by a processor in \ac{smm}.
%
To that end, the memory map is dynamically modified with respect to the current
state of the processor, and the physical addresses dedicated to the SMRAM from
the \ac{bios} perspective are used by the rest of the software stack to access
the VGA controller memory.
%
% \TODO{en fait, c'est le mapping de la mémoire qui est différent. Il faudrait
% l'expliquer comme çà, c'est un peu implicite dans ton explication et j'ai peur
% que le lecteur non averti ne comprenne pas.}
%
The \texttt{SMRAMC} register, exposed by the processor, controls this access
control mechanism \emph{via} its \texttt{D\_LCK} bit.
%
The \ac{bios} locks the SMRAM by setting the \texttt{D\_LCK} bit.
%
The memory controller of a processor with the \texttt{D\_LCK} bit set will
prevent \IOs targeting the SMRAM if the processor is not in \ac{smm}.
%
In addition, the only way to clear the \texttt{D\_LCK} bit is by performing a
complete reboot of the platform.
%
This leaves no opportunity for the rest of the software stack to modify the
content of the SMRAM, because the \ac{bios} will lock it again during the boot
sequence, prior to executing another software component.

\paragraph{System Management Interrupt.}

The \ac{smi} is a hardware interrupt which makes cores ``enter'' \ac{smm}.
%
More precisely, when a core receives a \ac{smi}, it saves its current state
(\emph{e.g.} its registers, current execution mode, etc.) in the SMRAM, then it
reconfigures itself;
%
in particular, it sets its program counter register to the value
$\mathtt{SMBASE} + \mathrm{0x8000}$.
%
From this point, the core is in \ac{smm} and starts to execute what should be
the \ac{smm} code.
%
Once the \ac{smm} code has performed the task it has been requested for, the
\texttt{rsm} instruction can be used.
%
This instruction, specific to the \ac{smm}, tells the core to exit \ac{smm} and
to restore its previous state.
%
This way, the execution of the software component previously halted by the
\ac{smi} can resume.
%
From the system point of view, it is almost like if nothing has happened.

Finally, the \ac{pch} exposes a register called \texttt{APM\_CNT} that a system
software component can write to in order to make the \ac{pch} trigger a
\ac{smi}\,\cite{intel2012pch}.
%
In practice, this mechanism is used by a system software component in order to
request the execution of the \ac{bios}, for the purpose of carrying out a given
service, \emph{e.g.} modifying the content of a given \ac{uefi} variable.
%
\footnote{\ac{uefi} variables are stored in the flash memory, alongside the
  \ac{bios} code.}

\paragraph{Flash Memory Lockdown.}
%
The content of the flash memory has to be protected from arbitrary write
accesses, similarly to the SMRAM protection.
%
That is, only the \ac{smm} code should be able to overwrite the content of the
flash memory.
%
This access control mechanism is implemented by the \ac{pch}, and is
configurable \emph{via} the \texttt{BIOS\_CNTL} control register.
%
Two bits of this register are of interest: the \texttt{BIOSWE} (\ac{bios} Write
Enable) bit, and the \texttt{BLE} (\ac{bios} Lock Enable) bit.

The semantics of the \texttt{BIOSWE} and \texttt{BLE} bits is as follows.
%
When the \texttt{BIOSWE} bit is clear, the \ac{pch} only authorizes read
accesses to the flash memory.
%
If a core sets the \texttt{BIOSWE} bit, the behavior of the \ac{pch} depends on
the value of the \texttt{BLE} bit.
%
If the \texttt{BLE} bit has been set by the \ac{bios} during the boot sequence,
then the \ac{pch} triggers a \ac{smi}.
%
As a consequence, the cores stop their current
executions and enter in \ac{smm}.
%
This prevents a system software component from modifying the content of the
flash memory, even if the \ac{pch} now authorizes write accesses.
%
It is the \ac{smm} code responsibility to clear the \texttt{BIOSWE} bit before
using the \texttt{rsm} instruction.
%
On the contrary, if the \texttt{BLE} bit is not set, setting the \texttt{BIOSWE}
bit will not cause a \ac{smi}, leaving the system software component free to
modify the content of the flash memory.
%
Similarly to the \texttt{D\_LCK} bit of the \texttt{SMRAMC} register, the
\texttt{BLE} bit cannot be cleared without a reboot.

The system software component and the \ac{bios} often use the flash memory
lockdown mechanism as a communication channel.
%
The system software sets the \texttt{BIOSWE} bit in order to notify the \ac{smm}
code that a \ac{bios} update is available.

\paragraph{}
%
The combination of the SMRAM, the \ac{smi} and the flash memory lockdown
explains why the \ac{smm} is often referred to as the x86 ``most privileged
execution mode.''
%
In a nutshell, the \ac{smm} code can leverage the same hardware capabilities
as the system software, including manipulating memories used by the system
software.
%
On the contrary, the system software cannot modify either the SMRAM content or
the \ac{smm} code stored in the flash memory, and cannot prevent the \ac{smm}
code execution, \emph{i.e} intercept or mask \acp{smi}.

\section{BIOS HSE Mechanism and Compositional Attacks}
\label{sec:usecase:hse}

To stay isolated from the rest of the software stack at runtime, the \ac{bios}
implements a \ac{hse} mechanism whose key hardware feature is the \ac{smm}, a
dedicated execution mode of x86 processor.
%
The \ac{smm} provides the necessary features to enforce its isolation from the
rest of the software stack.
%
Despite the key importance of \ac{smm}, several compositional attacks have been
disclosed over the past decade.
%
In this section, we detail three attacks which have defeated the three security
policies targeted by the \ac{bios}.
%
The SMRAM Cache Poisoning Attack allowed for circumventing the SMRAM access
control mechanism (\ref{subsec:usecase:hse:smram}).
%
The so-called \texttt{SENTER} Sandman attack prevented the execution of the
\ac{bios} and led to modifying the content of the flash memory
(\ref{subsec:usecase:hse:sandman}).
%
The Speed Racer attack resulted into an authorized modification of the flash
memory as well (\ref{subsec:usecase:hse:speed}).

\subsection{SMRAM Cache Poisoning Attack}
\label{subsec:usecase:hse:smram}

Between 1986, when the \ac{smm} has first been introduced, and 2009, it was
believed that the \texttt{SMRAMC} register alone was sufficient to enforce SMRAM
access control.
%
Loïc Duflot \emph{et al.}\,\cite{duflot2009smram} and Rafal Wojtczuk \emph{et
  al.}\,\cite{wojtczuk2009smram} independently showed that this belief was
misplaced when they disclosed the SMRAM Cache Poisoning Attack.

\paragraph{Attack Path.}
%
The SMRAM Cache Poisoning leverages the write-back strategy
(\ref{subsec:usecase:cachemem}) of the cache to circumvent the \texttt{D\_LCK}
bit protection.
%
The attack proceeds as follows:

\begin{enumerate}
\item Attackers set the cache strategy to be used for the SMRAM addresses to
  write-back.
  %
  This can be done by a malicious system software component, or even by a
  malicious application under certain circumstances (for instance, if an
  operating system exposes a software interface to manage the cache from the
  userland).
%
\item They write to the \texttt{APM\_CNT} register in order to trigger of a
  \ac{smi}.
%
\item The \ac{bios} code stored in SMRAM is executed in SMM, leading the cache
  to be filled with copies of that code, and the processor leaves \ac{smm} when
  it executes the \texttt{rsm} instruction.
%
\item Attackers write to an address which belongs to the SMRAM, and because of
  the write-back cache strategy, the processor updates the copies within the
  cache, and does not forward the \IO to the memory controller.
%
\item Attacker trigger another \ac{smi}, and the processor uses the modified
  copy of the \ac{smm} code inside its cache.
\end{enumerate}
%
This attack is a perfect illustration of a compositional attack:
%
both the memory controller and the cache work as expected.
%
The former prevents authorized accesses to the SMRAM, that is a subset of the
\ac{dram}, by a processor not in \ac{smm};
%
the latter is keeping copies of successful accesses to decrease latency due to
memory accesses.
%
However, the composition of the cache and the memory controller breaks the BIOS
Integrity property.

\paragraph{Countermeasure.}
%
The solution implemented by Intel to prevent further exploitation of this
vulnerability was to modify the behavior of the cache, when some memory access
target the SMRAM.
%
Because the SMRAM size and location remain specific to each architecture, this
means it requires an additional step of configurations to tell the cache the
physical addresses that belong to the SMRAM.

\paragraph{}
%
The SMRAM Cache Poisoning attack is a textbook case of compositional attacks.
%
It is interesting to notice that six years later, Christopher Domas has
disclosed another x86 vulnerability called Sinkhole\,\cite{domas2015sinkhole},
which relies on a similar approach ---but different hardware features--- to
trick a processor in \ac{smm} to execute arbitrary instructions.
%
Both attacks leave the content of the SMRAM in \ac{dram} intact, and leverage
only legitimate hardware features.

\subsection{Speed Racer}
\label{subsec:usecase:hse:speed}

In 2015, Corey Kallenberg \emph{et al.} showed that the scenario detailed
previously, such that setting \texttt{BIOSWE} triggers a \ac{smi} to suspend the
execution of the system software, suffered from a race condition if two cores
cooperate\,\cite{kallenberg2015racecondition}.

\paragraph{Attack Path.}
%
On a typical x86 hardware architecture, all the x86 cores of the platform will
\emph{eventually} enter \ac{smm} when a \ac{smi} is triggered.
%
On the contrary, the \texttt{BIOSWE} flag is set as soon as the
\texttt{BIOS\_CNTL} register is modified.
%
If two cores cooperate, they can benefit from a sufficient window for action and
successfully tamper with the flash memory content.
%
The attack proceeds as follows:

\begin{enumerate}
\item One core tries \emph{ad infinitum} to overwrite the content of the flash
  memory.
  %
  Because the \texttt{BIOSWE} bit is initially clear, the \ac{pch} discards its
  attempts, and the flash memory content is correctly protected.
%
\item At the same time, another core set the \texttt{BIOSWE} bit.
%
\item A \ac{smi} is triggered, but by the time it propagates to the first core,
  it may have successfully modified the flash memory content.
%
\end{enumerate}

\paragraph{Countermeasure.}
%
To prevent this race condition, Intel has introduced a new configuration bit to
the \texttt{BIOS\_CNTL} register: the \texttt{SMM\_BWP} (\ac{smm} \ac{bios}
Write Protection).
%
If the \texttt{SMM\_BWP} is set, the \ac{pch} discards any write access which
targets the flash memory \emph{unless all processors are in \ac{smm}}.

\subsection{\texttt{SENTER} Sandman}
\label{subsec:usecase:hse:sandman}

Another attack has defeated the flash memory lockdown protection.
%
In 2015, Xeno Kovah \emph{et al.} showed it was possible to leverage the Intel
TXT technology to circumvent the flash memory lockdown
protection\,\cite{kovah2015senter}.

\paragraph{Intel TXT.}
%
The Intel Trusted eXecution Technology (TXT)\cite{intel2015txt} is a feature of
some x86 processor, whose purpose is twofold: it attests the integrity of the
system software component program without the need to trust the \ac{bios}, and
it provides a trusted execution environment to the system software component.
%
% Its purpose is to address the limitations of both the Secure Boot and Trusted
% Boot approaches, which both rely on a software root of trust stored on the
% flash memory.
%%
% TXT provides a set of dedicated instructions to load a system software
% component, independently from previously executed software components.
%%
% \GH{Je ne comprends pas "making the processor the new root of trust". Il
% faudrait donner un peu plus de contexte pour qu'on puisse comprendre cette
% affirmation}
%
% Similarly to the Trusted Boot approach, TXT entrusts measurement of the system
% software component program to the \ac{tpm}.
% %
% These measurements can be leveraged to conditionally seal an encryption key or
% be part of an attestation protocol, as they would have been with in the
% context of a Trusted Boot.
% %
% However, contrary to the Trusted Boot approach, the component responsible for
% measuring the program is the processor itself.
% %
% This makes a big difference regarding the trust we can place in the
% measurements.
% %
% On the one hand, if the root of trust ---which is part of the \ac{bios}--- has
% been corrupted or if the \ac{bios} suffers from an exploitable vulnerability,
% the authenticity of the measurements is no longer guaranteed.
% %
% In such a case, neither the system software component nor a third party
% involved in an attestation protocol has a reliable solution to detect the
% attack.
% %
% On the other hand, only an error within the implementation of the TXT
% instructions can lead to a similar scenario\,\cite{wojtczuk2011txtbug}.
% %
% \TODO{Comment çà marche? Comment le processeur sait qu'il doit faire ces
% mesures sans que le BIOS ne puisse l'influencer sur sa décision? Comme tu as
% choisi de détailler le fonctionnement de cette techno (ce qui peut
% s'enterndre) il faudrait que tu nous donnes quelques billes supplémentaire
% pour qu'on comprenne le principe de fonctionnement.}

\paragraph{Attack Path.}
%
The flash memory lockdown mechanism was based on the assumption that unlocking
the flash memory would force the execution of the \ac{bios}, so that the latter
could lock it again.
%
To that end, the \ac{pch} triggers a \ac{smi} at the same time as it unlocks the
flash memory.
%
This mechanism was introduced at a time when software components could not
configure x86 processors to ignore \acp{smi}.
%
This assumption became incorrect when Intel introduced the first version of TXT,
whose ``trusted execution environment'' provided by TXT-capable processors
explicitly disabled \acp{smi} handling.
%
As a consequence, adversarial system software components whose execution were
initiated with TXT were able to unlock the flash memory, without being
interrupted by the \ac{smi} triggered by the \ac{pch} in response.
%
Then, they could freely modify the content of the flash memory, left unprotected
by the \ac{pch}.
% GUILLAUME: \TODO{La description de l'attaque est trop succinte. Il faut
% répeter que sans SMI, d'après le mécanisme de lock down, cela veut dire que la
% flash est inscriptible mais qu'on ne passe pas en SMM du coup n'importe quel
% logiciel peut modifier le contenu de la flash. C'est un manuscript de
% thèse. Il en faut pas faire (trop) de digression mais il faut être (très)
% pédagogique. Tu n'est pas à une page près!}  \PC{Ack avec Guillaume, la
% description est trop succinte, et le style télégraphique}
% \thomasrk[inline]{Est-ce que c’est mieux comme ça ? Au final, la version
% actuelle n'est pas forcément plus longue, mais elle se concentre plus sur
% l'essentiel et donne les éléments qui manquaient dans les précédentes
% itérations.}  \TODO{Je ne sais pas si c'est mieux qu'avant (je ne me souvient
% plus) mais je trouve très bien comme çà. Comme tu le dis, la description se
% concentre sur l'essentiel (en gros, un mécanisme de sécurité qui a été rajouté
% désactive les SMI). Par contre, on peut se poser la question de l'intérêt de
% la description de TXT qui précède. Peut-être qu'il est suffisant de dire que
% TXT est un mécanisme de sécurité additionnel qui a été rajouté par Intel et
% qui avait la mauvaise idée de désactiver les SMI. Sinon, si tu souhaite
% détailler TXT, tu détaille (un peu) actuellement le fonctionnement du
% mécanisme de vérification d'intégrité mais pas le mécanisme de "trusted
% execution environment" or c'est celui-là qui est en cause dans l'attaque
% visiblement...}

\paragraph{Countermeasure.}
%
The \texttt{SMM\_BWP} configuration bit makes this attack ineffective.
%
Besides, recent x86 processors do not disable \ac{smi} during the initialization
of a system software component with TXT anymore.

\section{Conclusion}
\label{sec:usecase:conclusion}

In this Chapter, we gave an overview of the x86 hardware architecture and of the
role played by the \ac{bios} within the software stack.
%
We then have detailed how the \ac{bios} relies on several hardware features to
remain isolated from the rest of the software stack. This illustrates real life
\ac{hse} mechanisms.
%
Finally, we have presented three compositional attacks, to better illustrate
the threats they pose.

The rest of this manuscript will use the SMRAM Cache Poisoning Attack as a
recurring application use case for our contributions, because it has motivated
our will to formally specify and verify \ac{hse} mechanisms, and it is a good
illustration of compositional attacks.
