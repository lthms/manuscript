%!TEX root = ../main.tex
\chapter{Intel x86 Architecture and BIOS Background}
\label{chapter:usecase}

\endquote{``\emph{You’re building your own maze, in a way, and you might just
    get lost in it.}''

  \hfill\footnotesize --- Marijn Haverbeke}

\vspace{1cm}\noindent
%
Our motivation to formally specify and verify \ac{hse} mechanisms comes from
various architectural attacks targeting the x86 hardware
architecture\,\cite{duflot2009smram,wojtczuk2009smram,kallenberg2015racecondition,domas2015sinkhole,kovah2015senter}.
%
Besides, Intel has introduced several \ac{hse} mechanisms over the past decades:
hardware-based virtualization (VT-x, VT-d)\,\cite{intel2014manualvt}, dynamic
root of trust (TXT)\,\cite{intel2015txt}, or applicative enclaves
(SGX)\,\cite{intel2014manualsgx,costan2016sgxexplained}.
%
From a market share perspective, x86 \acp{cpu} are widely used for laptops,
desktops and servers.
%
For these reasons, we have evaluated our approaches against the x86 hardware
architecture.
%
However, our contributions are intended to be applicable to other hardware
architectures.
%
We have chosen to focus on the \ac{hse} mechanism implemented by the BIOS at
runtime to stay isolated from the rest of the software stack, including the
system software.
%
Because the BIOS is the most privileged piece of software executed by the
hardware architecture, this \ac{hse} mechanism is of key importance.
%
Despite this fact, it has been the object of several architectural
attacks\,\cite{duflot2009smram,wojtczuk2009smram,domas2015sinkhole}, and
therefore, it illustrates perfectly our motivations.
%
However, it is important to emphasize that other mainstream architecture
(\emph{e.g.}  ARM) work on a similar basis and suffer similar issues.

The rest of this Chapter proceeds as follows.
%
We describe how a typical x86 hardware architecture is organized, and the
consequences of this organization in terms of security
(Section~\ref{sec:usecase:architecture}).
%
We then focus on the key role played by the \ac{bios}, and the security model it
requires so it can play this role (Section~\ref{sec:usecase:firmware}).
%
Once the role of the \ac{bios} has been established, we detail \ac{hse}
mechanisms it implements, and several architectural attacks which have defeated
these \ac{hse} mechanisms in the past (Section~\ref{sec:usecase:hse}).

\section{Introduction to x86 Architecture Security}
\label{sec:usecase:architecture}

Describing hardware architectures in depth is challenging, because they tend to
be made of many interconnected components of various natures.
%
From this perspective, the x86 hardware architecture is a textbook case, and
this is probably best illustrated by the scale of its documentation.
%
At the time of writing this thesis\,\footnote{Spring 2018.}, the \emph{Intel 64
  and IA-32 Architectures Software Developer’s Manual} is 4842 pages long.
%
Executed on the computer used to write the present thesis, the GNU/Linux
command-line application \texttt{lshw} lists about 30 hardware components, which
come with their own documentation, often in the form of large datasheet.
%
\thomasrk{Donne quelques exemples de composants, sans être exaustif}

\subsection{Basics of x86 Architecture}

A simplified x86 hardware architecture is pictured in
Figure~\ref{fig:speccert:x86arch}.
%
We now describe how the different hardware components are integrated together to
form a typical x86 computing platform, from the processor which executes
software components to various peripherals which allows the system to interact
with the rest of the world.

\begin{figure}
  \centering
  \begin{tikzpicture}
    \node [draw, inner sep=25pt] (MCH) {Processor};%

    \node [draw, inner sep=8pt, right=15pt of MCH, yshift=-20pt, text badly
    centered, text width=45pt] (DRAM) {DRAM Controller};%
    \node [draw, inner sep=8pt, right=15pt of MCH, yshift=20pt, text badly
    centered, text width=45pt] (Display) {Display Controller};%

    \draw (MCH.east) |- (Display.west);%
    \draw (MCH.east) |- (DRAM.west);%

    \node [draw, inner sep=8pt, left=15pt of MCH, text width=50pt, text badly
    centered] (PCIe) {PCI Express Controller};%

    \draw (PCIe) -- (MCH);%

    \node [draw, below=30pt of MCH, inner sep=40pt, text badly centered] (PCH)
    {PCH};%

    \draw (MCH) -- (PCH);%

    \node [draw, inner sep=5pt, below=10pt of PCH, text width=35pt,
    xshift=-55pt, text badly centered] (Flash) {Flash Memory};%
    \draw ([xshift=-10pt]PCH.south) |- (Flash.east);%

    \node [draw, inner sep=10pt, below=10pt of PCH, text width=25pt,
    xshift=55pt, text badly centered] (TPM) {TPM};%
    \draw ([xshift=10pt]PCH.south) |- (TPM.west);%

    \node [draw, inner sep=10pt, yshift=25pt, left=20pt of PCH, text width=45pt,
    text badly centered] (USB) {USB Controller};%

    \node [draw, inner sep=10pt, yshift=-25pt, left=20pt of PCH, text
    width=45pt, text badly centered] (PCI) {PCI Controller};%

    \draw (PCH.west) |- (USB.east);%
    \draw (PCH.west) |- (PCI.east);%

    \node [draw, inner sep=10pt, right=20pt of PCH, text badly centered, text
    width=50pt] (HD) {Hard Drive Controller};%

    \draw (PCH) -- (HD);%
  \end{tikzpicture}

  \caption{High-level view of the x86 hardware architecture}
  \label{fig:speccert:x86arch}
\end{figure}

\paragraph{Processor, Architecture and Microarchitecture.}
%
The main component of the architecture is the \emph{processor}.
%
It embeds several execution units called \emph{cores}.
%
They are responsible for executing sequences of assembly instructions which form
software components programs.
%
The concrete hardware implementation of the processor is often referred as the
Intel microarchitecture, by opposition to the Intel architecture which describes
the expected behavior and properties of a x86 system as seen by software
developers.
%
While the microarchitecture is often modified, the architecture has remained
backward compatible for decades\thomasrk{ref introduction to intel archi}.
%
The microarchitecture implements many optimizations, such as
multithreading\thomasrk{ref}, instruction pipelining\thomasrk{ref}, predictive
branching\thomasrk{ref}, out-of-order execution\thomasrk{ref} and many others.
%
It is expected that these optimizations do not violate properties of the
architecture.

For instance, x86 \acp{cpu} may start executing the wrong branch of a
conditional jump, due to their predictive branching technology.
%
This scenario is called \emph{branch misprediction}.
%
Misprediction recovery requires to revert the \ac{cpu} back to its state before
the jump.
%
We illustrate branch misprediction with program snippet, written with x86
assembly instructions (left) and its C equivalent (right)\,\footnote{Provided no
  optimization from the compiler nor the processor.}:
%
\begin{center}
  \begin{minipage}[t]{0.35\linewidth}
    \inputminted{asm}{Listings/predict.S}
  \end{minipage}
  \begin{minipage}[t]{0.35\linewidth}
    \inputminted{c}{Listings/predict.c}
  \end{minipage}
\end{center}
%
During the first 10 steps of the loop, when the core will execute the
\texttt{jle} instruction (l. 5), it will take the same branch arnd jumps to the
\texttt{loop} label.
%
During the last loop step, the register \texttt{rax} will have the value
\( 11 \).
%
According to the semantics of \texttt{jle}, the core should execute the
\texttt{movq \$0, \%rax} (l.6) next.
%
From a microarchitectural point of view, the branch prediction technology may
have tried to optimized the execution of the program, by eagerly jumping to the
\texttt{loop} label and continuing the execution of the program.
%
This means, at one point, it is possible that the \texttt{\%rax} register
contains the value \( 12 \).
%
When the core realizes it has mispredicted a branching, it reverts the incorrect
change it has made to \texttt{\%rax}.
%
% TODO : avant, limiter au temps d'accès cache. depuis le début de l'année 2018
% et meltdown/spectre, d'autres mécanismes ont pu être détournés de leur
% objectif de performance (branch prediction, tlb)

Intel microarchitecture blurs the frontier between hardware and software.
%
Indeed, an important part is not implemented as hardware circuit, but rather
under the form of \emph{microcode} programs.
%
That is, the processor is a programmable device, whose behavior ---including the
semantics of several x86 instructions\thomasrk{ref sgx explained?}--- is partly
determined by the microcode it has loaded.

% GUILLAUME : Je dirais directement un CPU est aujourd'hui composé de plusieurs
% unités d'exécution indépendantes qu'on appelle des coeurs. Ces composants sont
% eux même constitués de différents blocs qui forme un pipeline d'exécution
% (fetch décode ALU, etc.) Ces coeurs peuvent être physiquement indépendants
% (i.e des unités qui possèdent leurs propres pipeline indépendant) ou bien
% virtuel. Dans ce cas (hyperthreading). Dans ce cas la micro-architecture
% partage les blocs d'un même coeur et offre l'illusion aux composants logiciels
% qu'ils peuvent s'exécuter sur différents coeurs indépendants. Chaque coeur
% virtuel possède un état qui lui est propre (compteur ordinal, etc.) mais
% partage les blocs d'exécution avec les autres coeurs virtuels d'un même coeur
% physique). Mettre quelques schéma pour illustrer ces concepts.
%
% A core basically repeats the following tasks:
%%
% \begin{enumerate}
% \item Read the content of its (internal) \emph{program counter} register, and
%   interpret it as the address of the next instruction.
%%
% \item Fetch the content of this address.
%%
% \item Decode the instruction, that is identified the desired operation to
%   perform
%%
% \item Act accordingly, by modifying its internal state, and interacting with
%   memories and features exposed by other hardware components
%%
% \item Update the content of its \emph{program counter} register according to
%   the semantics of the instruction it has executed. Most of the time, it
%   increases it to fetch the next instruction, but it can also set it to an
%   arbitrary value (\emph{e.g.} with jump instructions).
% \end{enumerate}

\paragraph{Memories and Cores \IOs.}
%
The Intel x86 instructions set contains hundreds of instructions\thomasrk{ref}.
%
Together, they allow software developers to modify the hardware architecture
state, from the internal processor registers to the peripherals.
%
Interactions between a core and the rest of the hardware architecture are called
\IOs, for Input/Output: a core receives data during an input and it sends data
during an output.
%
The main target of cores \IOs is the \ac{dram}.
%
\ac{dram} forms the \emph{system memory}, by exposing an array of contiguous,
addressable memory cells.
%
These memory cells contains the instructions executed by the cores, \emph{and}
the data they manipulate.
%
Therefore, the semantics of a memory cell will be decided by a core depending on
the context, and the same binary sequence can be interpreted as an instruction
to execute or as an operand of an arithmetic operation.

Although \ac{dram} is the primary source of memory within the hardware
architecture, it is not the only one.
%
Other hardware components expose additional memory region.
%
For instance, a x86 processor integrates a display controller which exposes a
framebuffer to the cores.
%
By writing to the framebuffer, a core changes the pictures displayed by the
computer screen.
%
In practice, the processor handles \IOs with high-speed hardware components
(\emph{eg.} \ac{dram}, display controller, etc.).
%
In addition, it is connected to a companion chipset, called the
\ac{pch}\thomasrk{ref}, which handles slower hardware components (\emph{eg.}
hard drives, USB devices, etc.).
%
In previous iterations of the architecture, the processor was connected to a
\emph{northbrige} (low-latency), which was itself connected to a
\emph{southbridge} (slower \IOs).

Historically, x86 \acp{cpu} use two distinct memory spaces to communicate
either with the \ac{dram} or with the other hardware components.
%
On the one hand, most of the x86 instructions modify the content of the system
memory.
%
For instance, operands of the \texttt{add} instruction discussed previously can
be addresses of the system memory.
%
On the other hand, software developers use two dedicated instructions
---\texttt{in} and \texttt{out}--- to target the memories exposed by other
software components.
%
In addition to this approach, x86 \acp{cpu} are also capable of handling
\emph{memory-mapped} \IOs, where the memories exposed by the peripherals are
mapped within the system memory address space.
%
That is, the system memory manipulating by the cores is no longer formed by the
\ac{dram} only.
%
As a consequence, when a core reads from or write to the system memory, the
processor and the \ac{pch} dispatches its request to the correct hardware
component.
%
The mapping between a so-called \emph{physical address} manipulated by the
processor and its concrete memory location is called the \emph{memory map}.
%
The memory map is configurable, that is it can be changed dynamically \emph{via}
a set of configuration registers exposed by the processor and the \ac{pch}.
%
The two approaches are complementary.
%
As an example, PCI devices exposes a so-called configuration space, which is a
dedicated memory region whose locations have a well-specified
semantics\thomasrk{Figure}.
%
To read from and write to a given PCI device configuration space, a core
primarily relies on two configuration registers called
\texttt{PCI\_CONFIG\_ADDRESS} and \texttt{PCI\_CONFIG\_DATA}.
%
PCI devices are identified in the hardware architecture with a tuple
\( \langle \mathrm{bus}, \mathrm{device}, \mathrm{function} \rangle \) (denoted
by \texttt{bus:device:function}).
%
To read from or write to the configuration space of the device identified by the
tuple \texttt{b:d:f} at the offset \texttt{o}, the value
\[
  \mathtt{0x80000000}\,|\,\texttt{b} <\!\!< 16\,|\,\texttt{d} <\!\!<
  11\,|\,\texttt{f} <\!\!< 8\,|\,\texttt{o}
\]
%
where \(\cdot\,|\,\cdot\) is the bit-wise OR operator, and \(\cdot <\!\!< \cdot\)
is the logical shift left operator.
%
\IOs which target the \texttt{PCI\_CONFIG\_DATA} register are dispatched to a
PCI device according to the current value of Once the
\texttt{PCI\_CONFIG\_ADDRESS}.
%
The PCI specification states that the offset \texttt{0x10} of the PCI
configuration spaces is dedicated to so-called \texttt{BAR} (Base Address
Registers)\thomasrk{figure}.
%
Using the \texttt{BAR} of a given device, it becomes possible to remap its
configuration space within the system memory address space.

It should be noted that cores do not manipulate physical addresses directly.
%
On the contrary, they manipulate \emph{virtual} addresses.
%
The processor translates virtual addresses to corresponding physical addresses
thanks to two complementary and configurable features: segmentation and
pagination\thomasrk{ref}.
%
As such, determining which hardware component will handle a core \IO targeting a
given virtual address requires to have a complete knowledge of the x86 remapping
and virtual memory mechanisms and of their exact configurations at a given time.

\paragraph{Peripherals \IOs.}
%
Cores are not the only active hardware components present within a typical x86
hardware architecture.
%
For instance, several hardware components can also read from or write to the
\ac{dram} using a technology called \ac{dma}.
%
Hardware components can also interact with the processor by sending hardware
interrupts of various nature.
%
When a user presses a key of its keyboard, the lattes sends an interrupt
(\texttt{\#IRQ1}).
%
Interrupt handlers, that is programs executed by the core when it receives
interrupts, are configurable \emph{via} a so-called \ac{idt}\thomasrk{ref}.
%
Each line of the \ac{idt} corresponds to a given interrupt whose semantics is
specified by Intel.
%
When a core received an interrupt, it saves its current context inside the
\ac{dram}, then starts executing the corresponding interrupt handler.
%
Not all x86 interrupts come from a hardware component.
%
Several are used by the core itself, for instance to recover from errors.
%
For instance, if a core is not able to translates a virtual address into a
proper address, it raises a so-called Page Fault (denoted by \texttt{\#PF} or
\texttt{\#IRQ14}).

In addition, Intel has implemented several out-of-band management technologies
for its products.
%
They allow for administrating a computer \emph{via} the network, without the
need for a physical access.
%
In practice, the software components responsible for implementing the
out-of-band management features are not executed on the main processor: the ASF
technology was implemented by the network card\thomasrk{ref}, and since 2008 x86
\acp{cpu} embed within the \ac{pch} a so-called Management Engine.
%
ASF-capable network cards and the Management Engine require important
capabilities to implement their features.
%
For instance, the Management Engine can download from the Internet an executable
image, provoke the reboot of the computing platform and force the main \ac{cpu}
to execute the software components within the image it has
downloaded\thomasrk{ref}.
%
As of 2018, it becomes harder and harder to completely disable the Management
Engine on a typical x86 computing platform.
%
The Management Engine will notably the reboot of the platform if it cannot
initialize correctly.

\subsection{HSE Mechanisms Targeted Security Policies}

All being told, x86 computing platforms are formed of concurrent, interconnected
components.
%
They are of various natures, from physical components to sequences of
instructions.
%
They share common resources, like the \ac{dram} which can be modified by the
cores (which execute a software stack) and \ac{dma}-capable hardware components.
%
This allows for cooperation, \emph{eg.} when a network card receives a TCP
packet, it copies its content to a predefined address in \ac{dram}, then
notifies the operating system by sending an interrupt to the core.
%
This also raises a significant challenges from a security perspective.

\paragraph{Principle of Least Privilege}
%
Hardware and software components come from various places and can be of various
quality.
%
Once integrated together, they all can be potentially leveraged by an attacker
to threaten the security of the system.
%
In this context, the principle of least privilege\thomasrk{ref} applies: a given
component should only be able to leverage capabilities it needs to work
according to its purpose, where a ``capability'' refers to the right to perform
an \IO targeting a given component.
%
In practice, the implementation of this principle is challenging, for various
reasons.

Firstly, security checks can have an important impact over performance.
%
Partly for this reason, the hardware components have long been assumed
trustworthy\thomasrk{ref marion}.
%
This has not been without consequences from a security
perspective\,\cite{nohl2014badusb,hudson2015thunderstrike,chifflier2013uefi}.
%
Secondly, the ``least principle'' may vary from one execution to another.
%
To handle the variety of use cases of the x86 architecture, its default
configuration is very permissive.
%
It remains this way until software components such as the \ac{bios} or an
operating system modifies it to fit their needs, thanks to \ac{hse} mechanisms.

% OLD VERSION: For instance, it has also been repeatedly shown that the various
% physical interface exposed by the computer (\emph{e.g.}
% USB\,\cite{nohl2014badusb}, Thunderbolt\,\cite{hudson2015thunderstrike},
% PCI\,\cite{chifflier2013uefi}) could be leveraged to take the control of a
% given computing platform, because peripherals connected \emph{via} these
% interfaces could perform \ac{dma}.

\paragraph{Access Control}
%
The x86 hardware architecture provides many \ac{hse} mechanisms to determine
which \IO each component of the system can or cannot perform.
%
In other words, these \ac{hse} mechanisms enforce access control policies,
modelled as Access Control List (ACL)\,\thomasrk{ref}.
%
Subjects of these policies include the software and hardware components of the
system.
%
Objects ultimately come down to the memory locations of various nature scattered
within the hardware architecture.
%
Actions comprise reading from and writing to a memory location; from a core
perspective, it is also common to distinguish between reading a data and reading
an instruction.
%
For instance, thanks to the \ac{mmu}, an operating system can attribute ranges
of \ac{dram} to user applications it manages and isolates its code and data from
these applications.
%
The \ac{mmu} alone is not sufficient, because its scope does not cover its
configuration.
%
That is, it is not possible to configure the \ac{mmu} in order to prevent a
software component to modify the \ac{mmu} configuration.
%
As a consequence, an additional hardware feature has to be used: the protection
rings\thomasrk{ref}.
%
Indeed, ring 3 imposes several restrictions to software components, including
the capability to modify the \texttt{CR3} register which identifies the base of
the page table hierarchy the \ac{mmu} uses.
%
Finally, the scope of the \ac{mmu} is limited to the software component executed
by a core at a given time.
%
To impose an access control policy to hardware components as well, an operating
system can use the so-called VT-d feature, which implements an \IO-\ac{mmu} for
x86 computing platform\thomasrk{ref}.

%
% GUILLAUME: Ce qu'il faudrait dire, c'est que le CPU et la RAM sont des
% ressources partagées par les différents composants logiciels et matériels. Du
% coup, d'un point de vue de la sécurité, comme ces composants ne se font pas
% nécessairement confiance (parsqu'ils viennent de fournisseur différent,
% parqsu'il peuvent être malveillant ou vulnérables, etc.) il est nécessaire de
% les isoler les un les autres. La plupart des mécanismes de sécurité de la
% plateforme fournisse don c un mécanisme d'isolation}
%

However, the complexity of the x86 \IO resolution mechanism obliges to take into
account the numerous redirection features exposed by the architecture.
%
Memory locations, \emph{eg.} inside the \ac{dram} can have an arbitrary number
of aliases, in several layered address spaces: the \ac{dram} controller assigns
an address to each memory cell the \ac{dram} contains; the processor maps
physical addresses to \ac{dram} addresses; the \ac{mmu} maps virtual addresses
to physical addresses.
%
As a consequence, modifying the content of a memory cell may not be the only way
at the disposal of attackers to defeat a given access control policy.
%
For instance, if the access control policy refer to virtual addresses \( v \),
modifying the \ac{mmu} configuration results in modifying the content associated
with \( v \).

\paragraph{Availability}
%
Another class of security policies targeted by \ac{hse} mechanism is
availability security policies.
%
They focus on ensuring given software components are executed over time, without
the need of cooperation from untrusted software components.
%
For instance, a user application which contains an infinite loop should not be
able to prevent the execution of the operating system, and by extension other
user applications.
%
To that end, the x86 hardware architecture relies on hardware interrupts and
exceptions.
%
Going back to the example of user applications, an operating system can leverage
one of the numerous hardware timers (\emph{eg.} PIT\thomasrk{ref}, APIC
timer\thomasrk{ref}) at its disposal to periodically suspend the execution of
applications.
%
That is, hardware timers are the foundation of preemptive multi-tasking
strategies.

%\TODO{Beaucoup de confusion dans ce paragraphe. 1) si tu parles de information
%  flow et que tu dis que tu ne peux pas traiter la noninterference, tu vas te
%  faire allumer... 2) la non-interference ne traite généralement pas des canaux
%  auxilliaires mais seulement des flux directe et indirecte qui résulte d'une
%  écriture explicite dans un conteneur d'informatin ou canal public et 3) je ne
%  comprends pas ce que tu veux dire avec la citation du manuel Intel}:
%
%\begin{quote}
%  The Intel architecture aims to provide protection against software
%  side-channel attacks at the cache line granularity.
%
%  \hfill\small \emph{Intel Software Guard Extensions, Developer Guide}
%\end{quote}
%
%Because computer programs rarely fit inside a cache line, preventing
%side-channel attacks requires additional software-based security enforcement
%mechanisms.

% \subsection{Security Challenges and Attack Paths}
%
% \paragraph{Privilege Escalation.}
%%
%
% \paragraph{Malicious Hardware Components.}
%%
% Historically, peripherals have been part of the \ac{tcb} of every x86 \ac{hse}
% mechanisms, with important consequences on the computing platform security.
%%
% \TODO{Tu n'as pas défini avant la TCB, il me semble.}
%%
% For instance, it has also been repeatedly shown that the various physical
% interface exposed by the computer (\emph{e.g.} USB\,\cite{nohl2014badusb},
% Thunderbolt\,\cite{hudson2015thunderstrike}, PCI\,\cite{chifflier2013uefi})
% could be leveraged to take the control of a given computing platform, because
% peripherals connected \emph{via} these interfaces could perform \ac{dma}.
%%
% In 2010, Loic Duflot \emph{et al.} have shown it was possible to take control
% of certain network card.
%%
% They exploit some vulnerability in the network card firmware, which process
% network packets, to that end\,\cite{duflot2010network}.
%%
% This attack path did not require a physical access to the target.
%
% In that respect, the current situation of x86 systems has improved.
%%
% Modern \acp{cpu} provide a feature called \IO MMU\thomasrk{ref!}, which allows
% for configuring the memory ranges each peripheral can access to.
%%
% A well-configured \IO MMU drastically limits the impact of a rogue
% peripherals.
%%
% However, it is worth mentioning that, since 2008, Intel embeds the Management
% Engine, a processor \emph{inside} the \ac{pch}\thomasrk{ref}.
%%
% The Management Engine executes a complete software stack, and is capable of
% interacting with the \ac{dram} as well as other hardware components.
%%
% Intel leverages the Management Engine to provide out-of-band management
% solution, which means it can interact with the rest of the world without the
% cooperation of the main \ac{cpu}.
%%
% Although the Management Engine supposedly operates in a transparent manner
% from the \ac{cpu} perspective, we emphasize that a Management Engine
% vulnerability, a scenario which is not without precedent, potentially defeats
% any security measures implemented at the \ac{cpu} level.
%
% \paragraph{Microarchitecture}
%%
% Intel microarchitectural design is increasingly considered from a security
% perspective.
%%
% A deep knowledge of a core internals can be leveraged to perform side channel
% attacks\,\thomasrk{ref}.
%%
% For instance, time variations in memory accesses induced by the x86 cache have
% been leveraged to \emph{eg.} retrieve secrets\thomasrk{ref} or defeat software
% components isolation\thomasrk{ref}.
%%
% Since the disclosure of Spectre\thomasrk{ref} and Meltdown in January
% 2018\thomasrk{ref}, many other microarchitectural features have been subverted
% in side-channels attacks\thomasrk{ref spectre variants}.


\section{BIOS Overview}
\label{sec:usecase:firmware}

The \ac{bios} plays a significant role in Intel x86 computing platform.
%
It is the first piece of software executed by the \ac{cpu}, and initiates both
the hardware architecture and the system software execution during the boot
sequence (\ref{subsec:usecase:firm:boot}).
%
At runtime, it remains active to perform various tasks, including and not
limited to platform-specific events, device emulation, or \ac{bios} update
management (\ref{subsec:usecase:firm:runtime}).
%
As such, it is a natural part of the rest of the software stack \ac{tcb}
\TODO{je ne comprends pas ce que veut dire ce début de phrase}, and it can only
operate if certain security requirements are met
(\ref{subsec:usecase:firm:sec}).

\subsection{During the boot sequence}
\label{subsec:usecase:firm:boot}

The \ac{bios} program is stored inside a flash memory, a small memory connected
to the \ac{pch} through the Serial Peripheral Interface (SPI) bus on modern x86
computing platform.
%
When the \ac{cpu} is powered up, it is programmed to fetch the code stored at a
hard-coded address within the flash memory.
%
The first task of the \ac{bios} is to initiate the hardware architecture,
starting with the \ac{dram}.\thomasrk{ref}
%
Once the hardware components of the computing platform have been initialized,
the \ac{bios} searches for a system software component to load into memory.
%
Historically, ``legacy'' \acp{bios} were looking for a Master Boot Record (MBR)
at the beginning of mass storage devices (\emph{e.g.} hard drive, USB stick).
%
The MBR, whose size is limited to 512 bytes, contains a small program to
initiate a loader for a system software component.
%
Modern \acp{bios} implement the \ac{uefi} \thomasrk{ref}, whose purpose is to
standardize the boot sequence process in order to favour interoperability of
\ac{bios} implementations.
%
The boot sequence is divided into several phases, and the \ac{bios} is packaged
into several software components accordingly.
%
In particular, \ac{uefi}-compliant \ac{bios} can load so-called \ac{uefi}
applications of arbitrary size, leading modern hypervisors and operating systems
to be packaged as \ac{uefi} applications.

Because the \ac{bios} is the first software component executed by the hardware
architecture, and is responsible for initiating the execution of following
software components (\emph{eg.} an operating system), it is commonly designated
as the root of trust \thomasrk{ref x86 considered harmful?} for the software
stack.
%
As such, the integrity of the \ac{bios} code is primordial, and several
strategies have been proposed to detect \ac{bios} code corruption during the
boot sequence, with the two most predominant being Secure Boot and Trusted Boot.
%
The \ac{uefi} standard defines a security mechanism called Secure Boot
\thomasrk{ref!}.
%
When Secure Boot is enable, \ac{uefi}-compliant \acp{bios} should only execute
applications which provide an acceptable cryptographic signature, with respects
to a key hierarchy configurable by the computing platform owner.
%
The \ac{tcg} has standardized a hardware component called the \ac{tpm}, which is
the foundation of the Trusted Boot.
%
Each component of the boot sequence is expected to measure any software
component it initiates prior to starting its execution, and to entrust these
measurements to the \ac{tpm}.
%
If one of the software components executed during the boot sequence has been
corrupted, the difference is captured by the \ac{tpm}, because its measurements
is different than expected.
%
The measurements entrusted to the \ac{tpm} can be leveraged at the end of the
boot sequence in at least two ways.
%
The \ac{tpm} can cryptographicaly signs the measurements during a protocol
called remote attestation\,\thomasrk{ref} which allows a third-party to verify
that a given computing platform executes the code it is supposed to.
%
Another popular approach is to entrust to the \ac{tpm} an encryption key which
protect sensitive information a corrupted \ac{bios} should not be allowed to
access.
%
In this scenario, the access to the encryption key is correlated to the
measurements received by the \ac{tpm}.
%
Secure Boot and Trusted Boot can uncover \ac{bios} corruptions prior to the
execution of the illegitimate code.
%
However, they both rely on a root of trust which is neither measured nor
verified against a cryptographic signature.
%
Recent efforts have been expended to overcome this limitation.
%
For instance, in 2013 HP has introduced a security mechanism called SureStart
\thomasrk{ref} whose purpose is to move the root of trust within another
hardware component.
%
More recently, the NIST has published the Special Communication 800-193
---\emph{Platform Firmware Resiliency Guidelines}\thomasrk{ref}--- on the
subject.

\subsection{At runtime}
\label{subsec:usecase:firm:runtime}

The boot sequence ends once a system software component has been selected and
loaded into memory by the \ac{bios}.

\paragraph{Software Interfaces.}
At runtime, the \ac{bios} provides various software interfaces to the system
software component.
%
For instance, the \ac{acpi} tables\thomasrk{ref} is a standardized interface to
configure various highly vendor-specific aspects of the hardware platform, such
as power management or thermal management.
%
Similarly, legacy \acp{bios} have exposed facilities to system software
components, in the form of so-called \ac{bios} Interrupt.
%
For instance, the interrupt \texttt{0x10} is dedicated to video services
(\emph{eg.} setting the video mode, setting the cursor shape and position,
etc.).
%
Nowadays, \ac{uefi}-capable \acp{bios} expose so-called \emph{Runtime Services}
to system software component \thomasrk{ref} under the form of a function
pointers table.

In either cases, these interfaces act as intermediary layer between a system
software component and the hardware architecture.
%
In doing so, they reduce the coupling between the software and hardware
components.
%
Sometimes, their use is optional, and are only provided as a facility.
%
Other are mandatory gates towards certain computing platform features, because
they are related to critical features of the platform and the hardware vendors
do not want to rely on a (potentially vulnerable or malicious) system software
component.
%
For instance, software update for the \ac{bios} code is performed by the
\ac{bios} itself, so that the latter can validate the validity of the new
version prior to applying it.

\paragraph{Proactive Features.}
%
In addition to supporting the execution of the rest of the software stack
through its interfaces, the \ac{bios} carries out several hardware-specific
tasks whose internals are not publicly documented.
%
This includes and is not limited to handling hardware errors, checking thermal
zones, adjusting \acp{cpu} speed.
%
The execution of the \ac{bios} in this context is expected to be transparent to
the rest of the software stack.
%
Some \acp{bios} also emulate complete hardware devices to the system software
component.
%
As such, the \ac{bios} remains the most privilege software component of the
software stack, even after the end of the boot sequence.

\subsection{Security Model}
\label{subsec:usecase:firm:sec}

The \ac{bios} is provided by the hardware architecture’s manufacturer.
%
In most cases, it is a proprietary software, and the computer owner has little
control over it.
%
The rest of the software stack is considered untrusted, and the one goal of the
\ac{bios} is to keep the computer in a working state.
%
To that end, the \ac{bios} relies on several \ac{hse} mechanisms to enforce its
isolation from the rest of the software stack.
%
To the best of our knowledge, this isolation has never been formally specified.
%
We believe it can be divided into two complementary security policies, that is
an access control policy and an availability policy.
%
This isolation can be divided into two complementary security policies.
the security model of a x86 \ac{bios} can be divided into
two complementary security properties.
%
\TODO{D'où viennent ces propriétés de sécurités? Elles sont clairement établies
  par Intel ou d'autres chercheurs. C'est toi qui les propose? A partir de quoi?
  Il faut clarifier cela.}

\begin{definition}[BIOS Integrity]
  \label{def:usecase:biosint}
  The \ac{bios} is assigned two memories: the flash memory and a subset of the
  \ac{dram}, and the rest of the software stack cannot tamper with the result of
  read accesses performed by the \ac{bios} which target one of these memories.
  %
  \TODO{J'écrirerais la propriété de manière plus succinte et directe du style
    (the rest of the software stack (i.e the OS, applications...) cannot tamper
    with the result of read accesses performed by the BIOS on its dedicated
    memory regions. Such regions typically consists of the flash memory and a
    subset of the DRAM. En outre "tamper with" est-elle une expression
    idiomatique qui correspond précisement à ce que tu veux dire?}
\end{definition}

The \ac{bios} should be the only software component with the capability to
modify the memory flash content.
%
This prevents the scenario where an arbitrary system software rewrite the
\ac{bios} code inside the flash memory, then provoke a reboot.
%
\TODO{Pourquoi "arbitrary system software" ? Dire tout simplement que, comme les
  vendeurs ne font pas confiance à l'OS, celui-ci ne peut modifier directement
  la flash. Cela permet de garantir qu'un OS compromis ou malveillant ne puisse
  modifier le code du firmware arbitrairement.}
%
Besides, the flash memory is not the only memory dedicated to the \ac{bios}.
%
Memory accesses which target the flash memory are slow, as the latter is
connected to the \ac{pch}, behind the Memory Controller.
%
\TODO{Ce n'est surement pas la seule raison : la technologie de la mémoire flash
  est intrinséquement plus lente que la RAM. En outre, je suppose que tu dois
  être limité en nombre de lecture/écriture en termes de durée de vie. Bref,
  c'est un choix technologique qui n'est pas spécifique au BIOS, on utilise de
  la mémoire rémanente qui est lente pour assurer la persistence et de la
  mémoire volatile qui est rapide pour l'exécution}
%
This is why Intel dedicates a portion of the system memory \TODO{DRAM? c'est
  quoi system memory?} to the \ac{bios} runtime code \TODO{c'est quoi runtime
  code? il y a du code non runtime?}, and for understandable reason \TODO{Eviter
  les expression du style "for understandable reason, obviously, etc." Reste
  factuel.} provides the means to prevent the system software to tamper with
this portion.

\begin{definition}[BIOS Availability]
  \label{def:usecase:biosav}
  If needed, the \ac{bios} can temporarily retrieve the \ac{cpu} control flow,
  without any cooperation \TODO{que veut dire "cooperation"? Peut-être que
    "preemptif" décrit ce que tu veux dire} required from the system software.
\end{definition}

The system software is untrusted, but it controls the \ac{cpu} control flow.
%
\TODO{Controle le control flow n'est pas très claire. L'OS peut préempter
  l'exécution des applicatons. Le firmware peut à son tour présempter
  l'exécution de l'OS et des applications.}
%
We want to avoid a scenario where an arbitrary system software prevent the
execution of the \ac{bios} at runtime.
%
\TODO{Eviter les formulation personnelle "we want" sur quelque chose qui ne
  relève pas de ton implémentation ou de ton approche. Se sont des choix fait
  par les designer de la plateforme x86, pas par toi.}

\TODO{Quid de la confidentialité? De manière générale, tu annonces avant trois
  propriétés générale puis maintenant tu en définis 2. Quel est le lien?
  Pourquoi redéfinir des propriétés?}

\section{BIOS HSE Mechanism and Architectural Attacks}
\label{sec:usecase:hse}

To stay isolated from the rest of the software stack at runtime, the \ac{bios}
implements a \ac{hse} mechanism whose key hardware feature is the \ac{smm}, a
dedicated execution mode of x86 \acp{cpu} (\ref{subsec:usecase:hse:smm}).
%
The \ac{smm} provides the necessary features to enforce both \ac{bios} Integrity
(Definition~\ref{def:usecase:biosint}) and \ac{bios} Availability
(Definition~\ref{def:usecase:biosav}).
%
Despite the key importance of \ac{smm}, several architectural attacks have been
disclosed over the past decade, including the SMRAM Cache Poisoning Attack
(\ref{subsec:usecase:hse:smram}), SENTER Sandman
(\ref{subsec:usecase:hse:sandman}) and Speed Racer
(\ref{subsec:usecase:hse:speed}).

\subsection{System Management Mode}
\label{subsec:usecase:hse:smm}

To handle several software components with different level of privileges, Intel
\acp{cpu} provide several execution modes.
%
An execution mode can be assimilated to a set of hardware capabilities.
%
For instance, in a given execution mode, a \ac{cpu} may refuse to execute
certain assembly instructions, and the software component executed at the time
will not be able to leverage the side effects involved by the execution of these
instructions.
%
Contrary to a common belief, x86 execution modes are not organized in a linear
hierarchy, but are rather a matrix of complementary hardware features: ring
levels, paging configuration, virtualization technologies, \TODO{ ref, ref and
  ref...} etc.
%
Intel describes the \ac{smm} as follows:

\begin{quote}
  \ac{smm} is a special purpose operating mode provided for handling systemwide
  functions like power management, system hardware control, or proprietary
  OEM-designed code.
  %
  It is intended for use only by system firmware, not by application software or
  general-purpose system software.
  %
  The main benefit of \ac{smm} is that it offers a distinct and easily isolated
  processor environment that operates transparently to the operating system or
  executive and software applications.

  \hfill \small \emph{Intel 64 and IA-32 Architectures Software Developer’s
    Manual}
\end{quote}

This sentence roughly \TODO{banni roughly de ton vocabulaire!} corresponds to
our definitions of \ac{bios} Integrity and \ac{bios} Availability.
%
The \ac{smm} requires two additional hardware features: the SMRAM and the
\ac{smi}.

\paragraph{System Management RAM}
%
The SMRAM is the name given by Intel to the subset of the system memory
dedicated to the \ac{smm}.
%
The exact location and size of the SMRAM are architecture dependent.
%
To locate it, the \ac{cpu} uses a dedicated register named \texttt{SMBASE}, that
has to be configured by the \ac{bios} during the boot sequence.
%
As its name suggests, the \texttt{SMBASE} value should point to the base of the
SMRAM.
%
\TODO{Comment est identifié la fin de la plage de la SMRAM?}

At the beginning of the boot sequence, the SMRAM is left unprotected, meaning
arbitrary memory accesses targeting the SMRAM are authorized.
%
This design has been made to help \ac{bios} initialize the SMRAM content.
%
Once the \ac{smm} code ---the \ac{bios} code intended to be executed at runtime
in \ac{smm}--- has been correctly loaded into the SMRAM, and prior to giving the
\ac{cpu} control flow to the system software, the \ac{bios} has to lock the
SMRAM.
%

A locked SMRAM can only be accessed by a \ac{cpu} in \ac{smm}, and is invisible
to a \ac{cpu} not in \ac{smm}. \TODO{expliquer ce que veut dire invisible -> la
  mémoire n'est plus mappé dans la plage d'adresse et soit le processeur accède
  à une autre mémoire soit il y a une exception (je suppose)}
%
This access control mechanism is implemented by the Memory
Controller. \TODO{ref}
%
It can be configured by the \ac{bios} during the boot sequence, by setting the
\texttt{D\_LCK} bit of the \texttt{SMRAMC} configuration register.
%
A Memory Controller with the \texttt{D\_LCK} bit set will prevent a \ac{cpu} not
in \ac{smm} to tamper with \TODO{je n'aime pas trop tamper with. Access? Read?
  Write?} the SMRAM content.
%
In addition, the \texttt{D\_LCK} cannot be cleared once it has been set, except
by performing a complete reboot of the platform.
%
This leaves no opportunity to the system software to tamper with the SMRAM
content.

\paragraph{System Management Interrupt}

The \ac{smi} is a hardware interrupt which makes the \ac{cpu} ``enters''
\ac{smm}. \TODO{ref}
%
More precisely, when a \ac{cpu} receives a \ac{smi}, it saves its current state
(\emph{e.g.} its registers, current execution mode, etc.), either in the SMRAM
or, for most recent versions, in dedicated internal registers. \TODO{ref}
%
Once this preliminary step is done, the \ac{cpu} reconfigures itself;
%
in particular, it sets its program counter register to the value
$\mathtt{SMBASE} + \mathrm{0x8000}$.
%
From this point, the \ac{cpu} is in \ac{smm} and starts to execute what should
be the \ac{smm} code.
%
Once the \ac{smm} code has performed the task it has been requested for, the
\texttt{rsm} instruction can be used.
%
This instruction, which can be used only in \ac{smm}, tells the \ac{cpu} to
restore its previous state.
%
This way, the execution of the pieces of software previously halted by the
\ac{smi} can resume.
%
From the system point of view, it is like if nothing has happened. \TODO{excepté
  les effets de bord (le temps s'est écoulé, l'état du hardware a pu evoluer...}

\paragraph{Flash Memory Lock-down}
%
The content of the flash memory has to be protected from arbitrary tampering,
similarly to the SMRAM protection.
%
That is, only the \ac{smm} code should be able to overwrite the content of the
flash memory.
%
This second access control mechanism is implemented by the \ac{pch}.
%
It is configurable \emph{via} the \texttt{BIOS\_CNTL} control register.
%
Two bits of this register are of interest: the \texttt{BIOSWE} (\ac{bios} Write
Enable) bit, and the \texttt{BLE} (\ac{bios} Lock Enable) bit.

The semantics of the \texttt{BIOSWE} and \texttt{BLE} bits is as follows.
%
When the \texttt{BIOSWE} bit is clear, the \ac{pch} only authorizes read
accesses to the flash memory.
%
If the \texttt{BIOSWE} bit is set by a \ac{cpu}, the behaviour of the \ac{pch}
depends on the value of the \texttt{BLE} bit.
%
If the \texttt{BLE} bit has been set by the \ac{bios} during the boot sequence,
then the \ac{pch} triggers a \ac{smi}.
%
As a consequence, the \ac{cpu} stops its current execution and enters in
\ac{smm}.
%
This prevents the system software from tampering with the content of the Flash
Memory, even if the \ac{pch} now authorizes write accesses.
%
It is the \ac{smm} code responsibility to clear the \texttt{BIOSWE} bit before
using the \texttt{rsm} instruction.
%
On the contrary, if the \texttt{BLE} bit is not set, setting the \texttt{BIOSWE}
bit will not cause a \ac{smi}.
%
Similarly to the \texttt{D\_LCK} bit of the \texttt{SMRAMC} register, the
\texttt{BLE} bit cannot be cleared without a reboot.

The Flash Memory \TODO{parfois tu mets Flash Memory en majuscul, parfois non. Je
  pense qu'il ne faut pas de majuscule, harmonise à tout le document.} Lock-down
mechanism is often used as a communication channel between the system software
and the \ac{smm} code, in order to initiate a \ac{bios} update.
%
The system software sets the \texttt{BIOSWE} bit in order to notify the \ac{smm}
code that a \ac{bios} update is available.

\paragraph{}
%
The combination of the SMRAM, the \ac{smi} and the Flash Memory Lock-down
explains why the \ac{smm} is often introduced as the ``most privileged execution
mode'' of a x86 \ac{cpu}.
%
In a nutshell, the \ac{smm} code can leverage the same hardware capabilities
than system software, including tampering with memories assigned to the system
software.
%
On the contrary, the system software cannot modify neither the SMRAM content nor
the \ac{smm} code stored in the flash memory, and cannot prevent the \ac{smm}
code execution \TODO{"i.e, intercept or mask SMI " par exemple}.

\subsection{SMRAM Cache Poisoning Attack} % -------------------------------
\label{subsec:usecase:hse:smram}

Between 1986, when the \ac{smm} has first been introduced, and 2009, it was
believed that the \texttt{SMRAMC} register alone was sufficient to enforce the
Integrity (Definition~\ref{def:usecase:int}) of the SMRAM.
%
Loic Duflot \emph{et al.}\,\cite{duflot2009smram} and Rafal Wojtczuk \emph{et
  al.}\,\cite{wojtczuk2009smram} have independently shown that this belief was
misplaced when they disclosed the \emph{SMRAM Cache Poisoning Attack}.

\paragraph{Cache Memory}

Interacting with the \ac{dram} remains slow, in regard to the speed of a
\ac{cpu}.
%
To improve performance, Intel \acp{cpu} come with several (often three\TODO{J'ai
  l'impression que les proc ont maintenant tous 3 niveaux de cache et ce depuis
  pas mal de temps. Plutôt que often, tu pourrais dire depuis l'archi xxx})
levels of caches, from the smaller and quicker, to the bigger and
slower. \TODO{Tu pourrais donner un peu plus de détail en t'appuyant sur un
  schéma d'archi (L1 intégré au coeur, L2, L3, différence cache de données et
  cache d'instruction...}
%
When a \ac{cpu} successfully reads the memory at a given address, it keeps a
copy of the result in its cache.
%
Therefore, the next time it needs to read some data at this address, these data
are retrieved from the cache.
%
Regarding write accesses, Intel x86 \acp{cpu} provide five different caching
strategies (uncacheable, write combining, write-through, write-back and
write-protected)\,\cite{intel2014manualcache}.
%
Fine-grained cache strategy configuration is achieved through several hardware
mechanisms, including (but not limited to):
%
\begin{itemize}
\item The \texttt{CR0} register has a flag called \texttt{CD}, which enables
  caching once set.
%
\item The \ac{cpu} has several registers called Memory Type Range Registers
  (MTRR), to specify a cache strategy for pre-defined memory regions.
  %
  Each MTRR
\item The Page Table Attribute (PAT) allows for configuring a cache strategy at
  a memory page granularity.
\end{itemize}
%
The most commonly used strategy is write-back strategy.
%
It is summarized in Figure~\ref{fig:usecase:writeback}.
%
The purpose of the write-back strategy is to reduce the number of write accesses
forwarded to the system memory.
%
To that end, each cache block has a ``dirty bit'' which is set by the cache when
its value is updated by a write access.
%
In case of cache eviction, the cache verifies the value of the ``dirty bit'', so
it can update the underlying memory cell if necessary.
%
Therefore, as long as the cache block is not evicted, the \ac{cpu} does not
issue write access to the underlying memory.
%
The write-back strategy is not suitable for memories of hardware components
mapped into the \ac{cpu} address space.
%
For instance, caching write accesses to a framebuffer does not make sense.
%
This is why Intel provides four complementary strategies.
%
The uncacheable strategy disables the cache for the given addre

For cache-friendly programs, the gain in performance can be huge.
%
For instance, for a Pentium M Intel have stated that an access to L1 cache takes
3 \ac{cpu} cycles, an access to the L2 cache takes 12 cycles and an access to
the DRAM takes 240 cycles\,\cite{drepper2007memory}.

\begin{figure}
  \centering
  \begin{tikzpicture}
    \node [draw, circle] (EP) {};%

    \node [draw, below=of EP] (Sel) {Select a cache block};%
    \draw [-latex] (EP) -- (Sel);%

    \node [draw, signal, signal to=east and west, below=of Sel] (CH) {Cache
      hit?};%
    \draw [-latex] (Sel) -- (CH);

    \node [below=of CH] (CHbranch) {};%
    \draw (CH) to node [right] {[Yes]} (CHbranch.center);%
    \node [left=70pt of CH] (nCHbranch) {};%
    \draw [below] (CH) to node {[No]} (nCHbranch.center);

    \node [draw, signal, signal to=east and west, below=of nCHbranch.center]
    (Dirty) {Cache block dirty?};%
    \draw [-latex] (nCHbranch.center) -- (Dirty);%

    \node [draw, below=of Dirty, text width=70pt, text badly centered] (WB)
    {Write the content of the cache block back to the DRAM};%
    \draw [-latex] (Dirty) to node [right] {[Yes]} (WB);%

    \node [draw, below=of WB, text width=80pt, text badly centered] (Rf) {Read
      data from lower memory and fill the cache block};%
    \draw [-latex] (WB) -- (Rf);%

    \node [draw, below=of WB, text width=80pt, text badly centered] (Rf) {Read
      data from lower memory and fill the cache block};%
    \draw [-latex] (WB) -- (Rf);%

    \node [draw, below=of Rf, text width=80pt, text badly centered] (NotDirty)
    {Mark the cache block as ``not dirty''};%
    \draw [-latex] (Rf) -- (NotDirty);%

    \draw (Dirty) to node [below] {[No]} ([xshift=-90pt]Dirty.center);%
    \draw ([xshift=-90pt]Dirty.center) -- ([xshift=-90pt]Rf.center);%
    \draw [-latex] ([xshift=-90pt]Rf.center) -- (Rf);%
    \draw [-latex] (Rf) -- (NotDirty);%

    \node [below=230pt of CHbranch.center] (Join) {};%
    \node [draw, signal, signal to=east and west, below=5pt of Join] (RxW) {Read
      or Write?};%
    \draw [-latex] (CHbranch.center) -- (RxW);%
    \draw (NotDirty) -| (Join.center);%

    \node [draw, text badly centered, text width=70pt, right=50pt of RxW] (R)
    {Return data};%
    \draw [-latex] (RxW) to node [below] {[Read]} (R);%
    \node [draw, text badly centered, text width=70pt, left=50pt of RxW] (W)
    {Write new value in cache block};%
    \draw [-latex] (RxW) to node [below] {[Write]} (W);%

    \node [draw, text badly centered, text width=70pt, below=of W] (MakeDirty)
    {Mark cache block as dirty};%
    \draw [-latex] (W) -- (MakeDirty);

    \node [draw, fill=black, below=100pt of RxW, circle] (End) {}; \draw
    [-latex] (MakeDirty.south) |- (End.west); \draw [-latex] (R.south) |-
    (End.east);
  \end{tikzpicture}

  \caption{The Write-Back cache strategy}
  \label{fig:usecase:writeback}
\end{figure}

\paragraph{Attack Path}
%
The SMRAM Cache Poisoning leverages the write-back strategy of the \ac{cpu}
cache to circumvent the \texttt{D\_LCK} bit protection.
%
The attack proceeds as follows:

\begin{enumerate}
\item Attackers set the cache strategy to be used for the SMRAM addresses to
  write-back.  \TODO{Il faut insister sur le fait qu'il est possible de modifier
    la stratégie de cache de la SMRAM en dehors du SMM. Il n'est pas nécessaire
    de disposer de privilèges particuliers. }

%
\item They trigger a \ac{smi}, using the x86 hardware architecture mechanisms at
  their disposal \TODO{Tu n'as jamais expliqué comment on pouvait déclencher une
    SMI}
%
\item The \ac{smm} code is executed in SMM, leading the cache to be filled with
  copies of that code
%
\item Attackers issue a memory write access targeting the SMRAM, and because of
  the write-back cache strategy, the \ac{cpu} updates the copies within the
  cache, but does not pass the access to the Memory Controller.
%
\item Attacker trigger another \ac{smi}, and the \ac{cpu} uses the tampered copy
  of the \ac{smm} code inside its cache.
\end{enumerate}
%
This attack is a perfect illustration of an architectural attack:
%
both the Memory Controller and the \ac{cpu} cache work as expected.
%
The former prevents authorized accesses to the SMRAM, that is a subset of the
\ac{dram}, by a \ac{cpu} not in \ac{smm};
%
the former is keeping copies of successful accesses to decrease latency due to
memory accesses.
%
However, the composition of the cache and the memory controller breaks the BIOS
Integrity property.

\paragraph{Countermeasure}
%
The solution implemented by Intel to prevent further exploitation of this
vulnerability was to modify the behaviour of the cache, when the memory accesses
target the SMRAM.
%
Because the SMRAM size and location remain specific to each architecture, this
means it requires an additional step of configurations to tell the cache where
the SMRAM is. \TODO{to specifiy the location and the size of the SMRAM?}
%

\paragraph{}
%
The SMRAM Cache Poisoning attack is a textbook case \TODO{attention, tic de
  langage, cela fait plusieurs fois que tu utilises "textbook case"... ->
  perfectly illustrate?} of architectural attacks.
%
It is interesting to notice that six years later, Christopher Domas has
disclosed another x86 vulnerability called Sinkhole\,\cite{domas2015sinkhole},
which relies on a similar approach to trick a \ac{cpu} in \ac{smm} to execute
arbitrary instructions.
%
Both leave the content of the SMRAM in \ac{dram} intact.
%
Both leverage legitimate hardware features.

\TODO{Il faudrait ajouter ici une transition vers la suite}

\subsection{SENTER Sandman}
\label{subsec:usecase:hse:sandman}

In 2015, Xeno Kovah \emph{et al.} have shown \TODO{attention à l'utilisation des
  temps. Au passé, si tu parles d'une action révolu, utilise le preterit ; "in
  2015 Keno showed". L'utilisation du past perfect est très particulière en
  anglais. Je pense notamment que lorsque tu date précisemment un événement, le
  preterit s'impose. } it was possible to leverage the Intel TXT technology to
circumvent the Flash Memory Lock-down protection\,\cite{kovah2015senter}.

\paragraph{Intel TXT}
The Intel Trusted eXecution Technology (TXT) is a recent feature of some x86
\acp{cpu}, whose aim is to reduce the \ac{tcb} of the system software, by
removing most of the \ac{bios} from it.
%
Using a new family of instructions called \texttt{SENTER}, system software can
initiate an isolated execution environment.
%
The motivation behind TXT is to reduce the size of the architecture \ac{tcb}.
%
In a regular computing platform, the latter includes most of the hardware
components and the \ac{bios} code.
%
With TXT, the goal was to reduce it to the \ac{cpu} itself, which was acting as
a minimal loader for the system software.

\TODO{Ce paragraphe décrivant TXT est à revoir. 1) TXT n'est pas décrit
  suffisamment précisemment pour qu'on comprenne ce qu'il fait réellement. 2) la
  partie finale à partir de "The motivation behin TXT ..." est redondante avec
  le début. On a compris qu'il s'agissait de réduire la TCB, tu le dis dès le
  début. Ce que l'on ne comprend pas, c'est pourquoi et comment cela réduit la
  TCB. ". "system software can initiate an isolated execution environment" est
  trop vague. Il faudrait 1) expliquer le but au début (réduire la TCB) et
  ensuite 2) expliquer succinctement comment çà marche }

\paragraph{Attack Path}
%
The \texttt{SENTER} instruction family had the undesirable side effect of
disabling \acp{smi}.
%
In the context of TXT, this might make sense, as the \ac{bios} is considerered
untrusted.
%
However, this totally contradicts the security model of the \ac{smm} code.
%
Without \ac{smi}, the \ac{bios} Availability property cannot be enforced.
%
In addition, the Flash Memory Lock-down mechanism is deprived of one of its key
hardware features. \TODO{La description de l'attaque est trop succinte. Il faut
  répeter que sans SMI, d'après le mécanisme de lock down, cela veut dire que la
  flash est inscriptible mais qu'on ne passe pas en SMM du coup n'importe quel
  logiciel peut modifier le contenu de la flash. C'est un manuscript de
  thèse. Il en faut pas faire (trop) de digression mais il faut être (très)
  pédagogique. Tu n'est pas à une page près!}

\paragraph{Countermeasure}
%
Recent x86 \acp{cpu} do not disable \ac{smi} when using the \texttt{SENTER}
instructions.
%
This leaves the \ac{smm} code inside the TXT \ac{tcb}.
%
This is consistent with the security model of the \ac{smm} code, while the first
version of TXT was clearly misconceived with respect to the existing state of
the hardware architecture.

\subsection{Speed Racer}
\label{subsec:usecase:hse:speed}

There is at least another attack which has defeated the Flash Memory Lock-down
\TODO{attention à l'utilisation des majuscule (remarque récurrente...)}
protection.
%
In 2015, Corey Kallenberg \emph{et al.} have shown \TODO{attention à
  l'utilisation des temps. Je ne ferais plus cette remarque. Vérifie et corrige
  partout si nécessaire.} that the scenario detailed previously, such that
setting \texttt{BIOSWE} was triggering a \ac{smi} to suspend the execution of
the system software, suffered from a race condition in the presence of a second
\ac{cpu}\,\cite{kallenberg2015racecondition}.

\paragraph{Attack Path}
%
On a typical x86 hardware architecture, all the x86 \ac{cpu} of the platform
will \emph{eventually} enter \ac{smm} when a \ac{smi} is triggered.
%
On the contrary, the \texttt{BIOSWE} is set as soon as the \texttt{BIOS\_CNTL}
register is modified.
%
If two \acp{cpu} cooperate, they can benefit from a sufficient window for action
and successfully tamper with the Flash Memory content.
%
The attack proceeds as follows:

\begin{enumerate}
\item One \ac{cpu} tries \emph{ad infinitum} to overwrite the content of the
  Flash Memory. Because the \texttt{BIOSWE} bit is initially clear, the \ac{pch}
  discards its attempts, and the Flash Memory content is correctly protected.
%
\item At the same time, another \ac{cpu} set the \texttt{BIOSWE} bit.
%
\item A \ac{smi} is triggered, but by the time it propagates to the first
  \ac{cpu}, the latter \TODO{"the latter" me parait un peu confusant. Tu veux
    dire le premier CPU? Peu-être faudrait-il donner des noms à ces deux CPU
    (d'ailleurs s'agit-il de CPU ou de coeur?)} may have successfully tampered
  with the Flash Memory content.
%
\end{enumerate}

\paragraph{Countermeasure}
%
To prevent this race condition, Intel has introduced a new configuration bit to
the \texttt{BIOS\_CNTL} register: the \texttt{SMM\_BWP} (\ac{smm} \ac{bios}
Write Protection).
%
If the \texttt{SMM\_BWP} is set, the \ac{pch} discards any write access which
targets the Flash Memory \emph{unless all processors are in \ac{smm}}.

\section{Conclusion}
\label{sec:usecase:conclusion}

In this Chapter, we have detailed in depth \TODO{in depth me semble un peu
  exagéré} the relevant x86 hardware mechanisms involved in the \ac{hse}
mechanism implemented by the \ac{bios} at runtime to stay isolated from the
system software.
%
In addition, we have presented three architectural attacks, to better illustrate
the threats they pose.
%
We have used the SMRAM Cache Poisoning Attack as a recurring application use
case for our contributions, because it has motivated our will to formally
specify and verify \ac{hse} mechanisms, and it is a good illustration of
architectural attacks. \TODO{Je tournerais cette phrase au futur. Tu as décris
  cet exemple et tu vas l'utiliser comme use case pour illustrer ton approche.}
