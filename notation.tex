\chapter{Notations}
\label{frontmatter:notations}

\paragraph{Predicates}
%
\( \top \) (\emph{top}) is the logical statement which is always true, while \(
\bot \) (\emph{bottom}) is the logical statement which is always false.

\paragraph{Constructors}
%
We often define sets of values in terms of functions to construct these values.
%
These functions are called ``constructors,'' and they have mutually exclusive
images, i.e. it is not possible to construct the same value with two different
constructors.
%
Constructor names begin with a capital letter.
%
We adopt a notation similar to Haskell sum types to define sets \emph{via}
constructors.
%
For instance, this is how we define the disjoint union operator $\uplus$:
%
\[
  \begin{array}{rcl}
    A \uplus B & \triangleq & \func{Left} : A \rightarrow A \uplus B \\
               & |          & \func{Right} : B \rightarrow A \uplus B
  \end{array}
\]

Given $a \in A$, we write $\func{Right}(a) \in A \uplus B$ for the injection of
$a$ inside $A \uplus B$.

\paragraph{Named Tuples}
%
We adopt a notation similar to Haskell record types to manipulate ``named''
tuples, that is tuples where each component is a field identified by a name.
%
\[
  T \triangleq \langle~\func{field_1} : A,\quad\func{field_2} : B~\rangle
\]

For $x \in T$, we write $x.\func{field_1}$ for selecting the value of the
\func{field_1} field. We write $x \{ \func{field_1} \leftarrow a \}$ for
updating the value of the field \func{field_1}. As a consequence,
%
\[
  x \{ \func{field_1} \leftarrow a \}.\func{field}_k = \begin{cases}
    a & \text{if } k = 1 \\
    x.\func{field}_k & \text{otherwise}
  \end{cases}
\]
%
We use a similar notation for functions, that is given
\( f : A \rightarrow B \), \( f \{ x \leftarrow y \} \) is a new function such
that
%
\[
  \forall z \in A, f \{ x \leftarrow y \}(z) = \begin{cases}
    y & \text{if } z = x \\
    f(z) & \text{otherwise}
  \end{cases}
\]
%
Finally, we ease the definition of nested updates by gathering the nested fields
on the right side of \( \leftarrow \).
%
For instance, we write \( r \{ s(t) \leftarrow u \} \) for
\( r \{ s \leftarrow r.s \{ t \leftarrow z \} \} \).
