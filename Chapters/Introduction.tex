%!TEX root = ../main.tex
\chapter{Introduction}
\label{chapter:introduction}

\endquote{``\emph{All problems in computer science can be solved by another
    level of indirection.}''

  \hfill\footnotesize --- David Wheeler}

\vspace{1cm}\noindent
%
To manage complexity, computing platforms are commonly built as successions of
abstraction layers, from the hardware components to high-level software
applications.
%
Each layer leverages the interface of its predecessor to expose a higher-level,
more constrained set of functionalities for its successors.
%
This enables separation of concerns ---each layer encapsulates one dimension of
the overall complexity--- and modularity ---two layers which expose the same
interface can be seamlessly interchanged.

From a security perspective, each layer is often more privileged than its
successors.
%
For instance, system software components (\emph{e.g.} an operating system, a
hypervisor) manage the life cycle of upper layers (\emph{e.g.} applications,
guest OSes).
%
In such context, one layer implicitly trusts its predecessors.
%
On the one hand, it is important to keep this fact in mind when we consider the
security of the computing platform.
%
Trust in lower levels of abstraction can be misplaced, \emph{e.g.} hardware
implants and backdoors pose a significant threat to the platform
security\,\cite{yang2016a2}, and the platform firmware has been used as a
persistent attack vector across machine restarts\,\cite{embleton2013smm}.
%
On the other hand, one layer may constrain the execution of its successors, with
respect to a targeted security policy, \emph{e.g.}  an operating system enforces
availability ---fair share of \ac{cpu} time---, confidentiality and integrity
---exclusive partition of physical memories--- properties for the applications
it manages.

\section{Hardware-based Security Enforcement Mechanisms}

Constraining an execution can be achieved in various ways.
%
Concerning the lowest layers of a software stack, the common approach is to rely
on features provided by the hardware architecture to reduce the hardware
capabilities that can be used by upper layers.
%
For instance, system software components often leverage among other mechanisms a
\ac{mmu} to partition the system memory.
%
Thus, when an application is executed, it can only access a subset of the system
memory.
%
Besides, the \ac{cpu} can leverage a hardware timer to stop applications
execution, without the need for these applications to cooperate.
%
This scenario characterizes a class of security mechanisms we call \ac{hse}
mechanisms.

\begin{definition}[Hardware-based Security Enforcement Mechanism]
  \label{def:intro:hse}
  %
  A \ac{hse} mechanism consists in the configuration by a trusted software
  component of the underlying hardware architecture in order to constrain the
  execution of untrusted software components with respect to a targeted security
  policy.
\end{definition}

A \ac{hse} mechanism enforces its targeted security property when
%
\begin{inparaenum}[(1)]
\item the trusted software components correctly configure the hardware features
  at their disposal, and
%
\item these features are sufficient to constrain the untrusted software
  execution as expected.
\end{inparaenum}
%
Both requirements remain challenging.

\paragraph{Software Errors.}
%
Part of vulnerabilities in \ac{hse} mechanism implementations by trusted
software components are due to some misuse of hardware
features\,\cite{bulygin2014summary}.
%
In the past, most lower-level pieces of software, such as firmware components,
have not been conceived and implemented with security as primary focus.
%
The increasing complexity of hardware architectures can also be held partly
responsible.
%
While computers are made of dozens of components, software developers have to
read and understand as many, independent and often large documents of various
forms (\emph{e.g.} data sheets, developer manuals), and they rarely focus on
security.
%
When software developers misunderstand the documentation, as it happened for
instance for the \texttt{mov ss} and \texttt{pop ss} x86
instructions\,\cite{movsspopss}, the impact in terms of security can be
significant.

\paragraph{Hardware Errors.}
%
Over the past decades, vendors have regularly added security features to their
products.
%
Intel, for instance, has notably introduced hardware-based virtualization (VT-x,
VT-d)\,\cite{intel2014manualvt}, dynamic root of trust
(TXT)\,\cite{intel2015txt}, or applicative enclaves
(SGX)\,\cite{intel2014manualsgx,costan2016sgxexplained}.
%
It is crucial to notice that most of them have been circumvented due to
implementation bugs\,\cite{wojtczuk2011txtbug,sang2010iommu}.
%
This is not surprising, as novel hardware features tend to be more and more
complex.
%
Hence, using advanced, novel hardware features in a security-sensitive context
may be counterproductive.
%
In addition to these implementation errors, the fact that hardware architectures
often comprise hundreds of features implemented by dozens of interconnected
devices complicates the conception of new hardware features.
%
Indeed, these new features should not interfere with the security properties
enforced by the existing components of the platform.
%
For instance, the flash memory (where lives the \ac{bios} code) is supposedly
protected against arbitrary write accesses from system software components,
thanks to a particular hardware interrupt.
%
When Intel introduced TXT\,\cite{intel2015txt}, they did not anticipate that
this novel security feature had the particular side effect of disabling the
hardware interrupt used to protect the flash memory\,\cite{kovah2015senter}.
%
Such inconsistencies in hardware specifications pave the road toward
compositional attacks\,\cite{wing2003compositionalattack}.

\begin{definition}[Compositional Attack]
  Compositional attacks characterize scenarios where each component is working
  as expected in isolation, yet their composition creates an attack path which
  prevents end-to-end security enforcement.
\end{definition}

In the context of \ac{hse} mechanisms, this means untrusted software components
can leverage one hardware component to defeat a \ac{hse} mechanism implemented
to constrain its execution.
%
% PC: “Compositional attacks are /often/ due...”  Réponse Thomas: Je ne vois pas
% d’autres raisons, en fait.  En tout cas, l’idée de base est qu’on ne
% s’intéresse qu’a celle là.
%
Compositional attacks are due to a flaw in the specifications of the computing
platform.
%
As such, they precede implementation errors, and their countermeasures often
require a change in the hardware interface.
%
To prevent them, it is mandatory to reason about the computing platform as a
whole.

\section{Formal Verification of HSE Mechanisms}
\label{sec:intro:verif}

The significant impact of previously disclosed compositional
attacks\,\cite{duflot2009smram,wojtczuk2009smram,domas2015sinkhole,kallenberg2015racecondition,kovah2015senter}
have motivated our will to formally specify HSE mechanisms.
%
We believe this would benefit both hardware designers and software developers.
%
Firstly, a formal specification of HSE mechanisms can be leveraged as a
foundation for a systematic approach to verify hardware specifications.
%
For each novel hardware feature introduced, it is necessary to check that the
previous proofs hold, meaning this feature does not introduce any compositional
attack.
%
Secondly, it provides unambiguous specifications to firmware and system software
developers, in the form of a list of requirements to comply with, and the
provided security properties.
%
We believe these specifications are a valuable addition to the existing
documentation, because they gather at one place information that is normally
scattered across many documents which sometimes suffer from lack of security
focus.
%
%\TODO{Cet argumentaire est repris à plusieurs endroits (dans l'abstract, ici
%  dans l'intro et dans la section 5.5. Il faut à la fois s'assurer que ces
%  différentes occurences sont cohérentes dans la manière de présenter les choses
%  (il faudrait s'en assurer en les comparants car les différentes parties ont
%  évoluées à des rythmes différent) et que ces différentes occurences ne sont
%  pas redondantes. En gros, il faut que chaque nouvelle occurrence apporte
%  quelque chose (un peu plus de détail, une mise en contexte, etc.}
%
% Thomas: J’ai un peu épuré la version de l’abstract, parce que je ne voyais pas
% vraiment comment déveloper de manière non artificielle cette occurrence-ci.

We steer a middle course between two domains: hardware and system software
verification.
%
Generally, hardware verification focuses on properties that are transparent to
the executed software, and require no configuration from its part, while
software verification abstract as much as possible of the hardware architecture
complexity.
%
However, compositional attacks can come from unsuspected places, with no apparent link
with the considered security mechanism.
%
For instance, the SENTER Sandman attack\,\cite{kovah2015senter} leveraged a
dedicated execution mode of x86 \acp{cpu} to disable the protection of the flash
memory wherein the \ac{bios} code is stored.
%
Hence, the composition of the numerous configurable hardware features is less
subject to formal verification.
%
At the same time, some components are individually complex: for example, the
Intel Architectures Software Developer Manual\,\cite{intel2014manual} is 4842
pages long, the Memory Controller Hub datasheet\,\cite{intel2009mch} is 430
pages long, and the Platform Controller Hub datasheet\,\cite{intel2012pch} is
988 pages long.
%
Besides, new hardware components and new versions of already existing components
are frequently released.
%
As a consequence, the more modular our models and proofs are, the more
practicable our approach becomes.
%
Otherwise, each modification of the hardware architecture will have an important
impact on the proofs already written.

\section{Contributions}

During the first stage of this thesis, we propose a formal definition of
\ac{hse} mechanisms in the form of requirements trusted software components have
to satisfy to see a targeted security policy enforced at all time by the
hardware architecture.
%
To evaluate our approach, we formally specify and verify a \ac{hse} mechanism
implemented by the \ac{bios} of x86-based computing platform at runtime to
remain isolated from the rest of the software stack.
%
Our choice has been motivated by the prominent position of the Intel hardware
architecture on the personal computer market, the compositional attacks
previously disclosed, and the large number of publications of the security
research community which form a significant addition to its documentation.
%
The resulting formal definition assumes as little as possible about the actual
implementation of the \ac{bios}, and constitutes, to the extent of our
knowledge, the first formalization of the \ac{bios} security model at runtime.
%
Our model and its related proofs of correctness have been implemented in the Coq
theorem prover, to increase our confidence in our result.
%
This work has been presented at the $21^{th}$ International Symposium on Formal
Methods (FM2016)\,\cite{letan2016speccert}.
%
Besides, the resulting project, called SpecCert, has been made available as a
free software\,\cite{letan2016speccertcode}.

After this first contribution, we focus our attention on the challenge posed by
the modeling of a hardware architecture as complex as x86.
%
In this thesis, we advocate for the use of a general-purpose model of a hardware
architecture to support the specification of \ac{hse} mechanisms which can be
implemented on this architecture.
%
Our experiment with our formal definition of \ac{hse} mechanisms convinced us
that such a general-purpose model should obey certain requirements, notably in
terms of readability and modularity, so it could remain applicable to real life
verification problem.
%
Our second contribution is a novel approach, inspired by algebraic effects, to
enable modular verification of complex systems made of interconnected
components.
%
This approach is not specific to hardware models, and could also be leveraged to
reason about composition of software components as well.
%
This work has been presented at the $22^{th}$ International Symposium on Formal
Methods (FM2018)\,\cite{letan2018freespec}.
%
Besides, we have implemented our approach in Coq, and the resulting framework,
called FreeSpec, takes advantages of theorem prover automation features to
provide general-purpose facilities to reason about components interactions.
%
Similarly to SpecCert, FreeSpec has been published as a free
software\,\cite{letan2018freespeccode}.

\section{Outline}

The rest of this manuscript proceeds as follows.

First of all, Part~\ref{part:context} provides the context in which this thesis
falls.
%
In Chapter~\ref{chapter:usecase}, we give an introduction to the x86 hardware
architecture and the particular role played by the \ac{bios}, in order to
present three illustrative compositional attacks.
%
In Chapter~\ref{chapter:relatedwork}, we review existing formal verification
approaches that have been successfully used in order to verify hardware and
software system.
%
We justify, in this context, our choices in terms of formalism and tools.

Part~\ref{part:speccert} focuses on our first contribution.
%
In Chapter~\ref{chapter:speccert}, we present our general-purpose definition to
support the specification and verification of
\ac{hse} mechanisms.
%
In Chapter~\ref{chapter:speccert2}, we leverage this definition to reason
about the \ac{hse} mechanism implemented by the \ac{bios} to remain isolated
from the rest of the software stack at runtime.

Part~\ref{part:freespec} focuses on our second contribution.
%
In Chapter~\ref{chapter:freespec}, we present our compositional reasoning
framework for Coq, to modularly specify and verify systems made of
interconnected components

Finally, we conclude this thesis in Chapter~\ref{chapter:conclusion}, where we
suggest some possible directions for future work.
