\chapter{Remerciements}

\begin{otherlanguage}{french}
  Je tiens à remercier en premier lieu les membres de mon jury de thèse, à
  commencer par Gilles Barthe et Laurence Pierre pour avoir rapporté cette
  thèse. Merci aussi à Emmanuelle Encrenaz-Tiphene et Alastair Reid. Je suis
  honoré de l’intérêt que vous avez porté à mes travaux.

  % Encadrement
  % Ludo
  Je remercie ensuite Ludovic Mé, mon directeur de thèse. Venir te voir à la fin
  de l’un de tes cours pour te demander s’il n’était pas possible de faire un
  stage \og{}sur le kernel\fg{}, aura été avec le recul mon premier vrai pas en
  direction du monde de la recherche. Merci de m’avoir proposé de rester un an
  de plus dans l’équipe CIDRE après la fin de mon stage et merci, surtout, de
  m’avoir parlé de l’ANSSI et encouragé à y postuler.
  %
  Pierre, Guillaume, votre encadrement et votre soutien tout au long de cette
  thèse explique pour beaucoup sa qualité finale, quand bien même la direction
  de mes recherches s’est finalement révélée assez éloignée de vos domaines de
  prédilections. Merci de m’avoir laissé libre d’explorer les sujets qui
  m’intéressaient. Je n’oublierai jamais comment vous avez su à plusieurs
  reprises m’expliquer mon propre travail, avec ces mots que je n'arrivais pas à
  trouver. J’en étais d’autant plus impressionné que la direction que j’ai donné
  à mes travaux a finalement été assez assez éloignée de vos sujets de
  prédilections.
  %
  Merci, donc, à Guillaume Hiet. Début 2013, tu acceptais avec Frédéric Tronel
  de m’encadrer une première fois pour mon stage de fin d’études. J’étais loin
  de me douter, alors, que nous travaillerions ensemble aussi longtemps, ni
  combien notre collaboration m’apporterait.
  %
  Merci, ensuite, à Pierre Chifflier. Nos discussions pendant ces quatre années
  ont nourri ma curiosité et forgé mon esprit critique. Te voir t’intéresser à
  tant de sujets a été une véritable source d’inspiration.
  %
  Merci, enfin, Benjamin Morin, pour m'avoir fait confiance. L’évolution de ta
  carrière ne nous a pas permis de travailler ensemble jusqu’au bout de cette
  thèse, mais je n’oublie pas que c’est en grande partie grâce à toi qu’elle a
  pu débuter.

  \paragraph{}
  %
  Tout au long de cette thèse, j’ai pu compter sur le soutien de beaucoup de
  personnes. Je tiens ainsi à remercier chaleureusement Yves-Alexis Perez. En
  ta qualité de chef d’équipe, tu as dû composer avec un agent dont les
  disponibilités n’étaient pas toujours dépendantes des tâches que tu lui
  confiais. Si j’ai pu tout à la fois conduire avec succès un travail de thèse
  et participer aux missions de l’Agence, c’est aussi grâce à toi. J’ai souvent
  été sensible au temps de recherche que tu me ménageais, à la liberté que tu me
  laissais pour choisir mes sujets et à la confiance que tu m’as témoignée en me
  confiant des missions passionnantes.
  %
  Merci à Pierre Néron. Tu m’as énormément aidé à revoir ma copie après le refus
  de ma première soumission ; tes conseils et relectures ne sont pas étrangers à
  l’acceptation de mon premier article.
  %
  Merci aussi à Yann Régis-Gianas d’avoir accepté de me rencontrer pour que je
  lui présente mes travaux. Ton aide aura été précieuse pour avoir la chance de
  pouvoir présenter mes travaux à Oxford.

  Merci à Anael Beaugnon. Ton soutien, ton aide, tes encouragements m’ont
  indéniablement aidé à tenir ce marathon qu'est une thèse. Tu t’es toujours
  spontanément proposée pour relire mes articles et ce manuscrit, quand bien
  même il n’était pas tout à fait aligné avec ton domaine de recherche. Ton
  courage et ta ténacité face à l’adversité auront quant à eux été une véritable
  source d’inspiration.
  %
  Merci à Marion Daubignard, pour ta gentillesse, ta bonne humeur et tes
  conseils. J’ai précieusement gardé la citation de ta grand-mère : elle trône
  encore sur le côté de mon écran. Je la regarde souvent et elle m’aide à me
  remotiver quand j’en ai besoin.
  %
  Je ne peux pas ne pas remercier Aurélien Deharbe, qui appréciera je l’espère
  cette double négation à sa juste valeur. Je suis heureux que nos
  pérégrinations respectives nous aient amenés à nous rencontrer. J’ai toujours
  pu compter sur ton soutien, ton énergie et ta bonne humeur communicative quand
  j’en avais besoin.

  Je remercie mes collègues à l’ANSSI. Depuis mon arrivée en octobre 2014, j’ai
  eu la chance de travailler avec des personnes extrêmement compétentes et
  passionnées par ce qu’elles faisaient. Vous voir vous investir dans vos
  missions a nourri ma propre motivation à mener les miennes du mieux
  possible. J’ai une pensée toute particulière pour ceux avec qui j’ai partagé
  un bureau. J’en ai déjà cité plusieurs déjà, mais il me faut nommer Alain
  Ozanne et Philippe Thierry pour être complet.
  %
  Je remercie aussi chaleureusement l’équipe CIDRE, pour qui je garderai pendant
  longtemps encore une affection particulière. C’est en vous côtoyant que j’ai
  découvert le monde de la recherche. La bienveillance et l’investissement dont
  vous avez tous fait sans cesse preuve forcent le respect. J’ai toujours pris
  beaucoup de plaisir à vous retrouver et vous présenter l’avancée de mes
  travaux pendant nos séminaires d’équipe et je regrette seulement ne pas avoir
  eu l’occasion de le faire plus souvent.
  %
  J’ai préféré ne pas citer vos noms à tous, car vous êtes trop nombreux. Je me
  connais et je sais que le risque est grand que j’oublie l’un d’entre vous.

  \paragraph{}
  %
  Je ne peux pas conclure ces remerciements sans prendre le temps d’exprimer ma
  gratitude à ma famille, qui a toujours su répondre présente ; aussi bien dans
  les petits riens du quotidien que dans les étapes clefs de ma vie. Je crois
  que je ne mesurerai jamais vraiment tout ce que vous avez fait pour
  moi. Maman, je crois que je ne t’ai jamais explicitement remercié d’avoir pris
  la décision de mettre entre parenthèses ta carrière professionnelle pendant si
  longtemps pour tes enfants. Il m’a fallu longtemps pour prendre la mesure de
  tout ce que cela représentait. Je profite donc de l’opportunité que m’offre ce
  manuscrit pour me rattraper. Papa, je t’ai toujours vu redoubler d’efforts
  pour pourvoir au besoin des tiens. Tu voulais, toi aussi, t’assurer que tes
  enfants puissent faire leurs choix de vie, sans regret. Il est ici utile de le
  dire : vous avez, l’un comme l’autre et l’un avec l’autre, réussit. C’est à
  nous, maintenant, de tracer le reste de notre route --- sans jamais oublier la
  bonne fortune qui fut la nôtre d’être si bien lotis à notre point de départ.
  N’est-ce pas, Marion, ma petite sœur que j’aime tant ? Merci à toi aussi,
  évidemment. Je sais que, peu importe ce qui m’arrive, peu importe mes choix de
  vie, tu seras là.
\end{otherlanguage}
