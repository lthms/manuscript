In this thesis, we consider a class of security enforcement mechanisms we called
\emph{Hardware-based Security Enforcement} (HSE), such that trusted software
components rely on the underlying hardware architecture to constrain the
execution of untrusted software components with respect to targeted security
properties.
%
For instance, a system software which sets up a page table to isolate
applications implements a HSE mechanism.
%
More generally, HSE mechanisms are particularly used to keep the software
components of modern software stacks isolated from each other.

For a HSE mechanism to correctly enforce a targeted security property, it
requires both hardware components and trusted software components to play their
parts.
%
During the past decades, several vulnerability disclosures have defeated such
HSE mechanisms.
%
We focus on vulnerabilities which are the result of errors at a specification
level, rather than implementation errors.
%
Several critical vulnerabilities are the result of a legitimate use by an
attacker to one hardware component in order to circumvent the security
enforcement of another.
%
We call this class of security vulnerabilities \emph{architectural attacks}.
%
Our objective is to explore systemic approaches to specify and verify HSE
mechanisms using formal methods, in the hope of uncovering potential
architectural attacks ahead of time.
%
This is in line with an ongoing effort, for many years and by industrial
manufacturers and researchers alike, to formally verify computing platforms.
%
Unfortunately, defining a comprehensive model in terms of hardware and software
components remains challenging, yet such a model is a prerequisite to verify a
computing platform in terms of HSE mechanisms.

Our contribution is two-side:
%
\begin{itemize}
\item We have proposed a three-step methodology for specifying and verifying HSE
  mechanisms against hardware architecture models.
  %
  We have written a minimal model for a single core x86-based computing
  platform, and used it to specify and verify a real HSE mechanism to
  isolate the code executed while the CPU is in System Management Mode (SMM), a
  highly privileged execution mode of x86 microprocessors.
\item We have proposed a novel approach to verify compositions of
  components, to enable modular verification of a complex hardware architecture.
  %
  This approach is not specific to hardware models, and could also be leveraged
  to reason about software applications as well.
\end{itemize}
%
We have implemented proofs of concepts of our approaches using the proof
assistant Coq in two projects called SpecCert and FreeSpec. Both have been
released as Free Software.
