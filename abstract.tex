In this thesis, we consider a class of security enforcement mechanisms we called
\emph{Hardware-based Security Enforcement} (HSE).
%
In such mechanisms, some trusted software components rely on the underlying
hardware architecture to constrain the execution of untrusted software
components with respect to targeted security properties.
%
For instance, an Operating System that sets up page tables to isolate userland
applications implements a HSE mechanism.

For a HSE mechanism to correctly enforce a targeted security property, it
requires both hardware and trusted software components to play their parts.
%
During the past decades, several vulnerability disclosures have defeated HSE
mechanisms.
%
We focus on the vulnerabilities that are the result of errors at a specification
level, rather than implementation errors.
%
In some critical vulnerabilities, the attacker makes a legitimate use of one
hardware component to circumvent the HSE mechanism provided by another
one.
%
For instance, cache poisoning attacks leverages inconsistencies between cache
and DRAM's access control mechanisms.
%
We call this class of security vulnerabilities \emph{architectural attacks}.
%
Our objective is to explore systemic approaches to specify and verify HSE
mechanisms using formal methods, in the hope of uncovering potential
architectural attacks ahead of time.
%
This is in line with an ongoing effort, for many years and by industrial
manufacturers and researchers alike, to formally verify computing platforms.
%
Unfortunately, defining a comprehensive model in terms of hardware and software
components remains challenging, yet such a model is a prerequisite to verify HSE
mechanisms.

Our contribution is two-side:
%
\begin{itemize}
\item We proposed a three-step methodology for specifying and verifying HSE
  mechanisms against hardware architecture models.
  %
  To evaluate our approach, we proposed a minimal model for a single core
  x86-based computing platform.
  %
  We used it to specify and verify the HSE mechanism provided by Intel to
  isolate the code executed while the CPU is in System Management Mode (SMM), a
  highly privileged execution mode of x86 microprocessors.
\item We proposed a novel approach to verify compositions of components and to
  enable modular verification of a complex hardware architecture.
  %
  This approach is not specific to hardware models, and could also be leveraged
  to reason about software applications as well.
\end{itemize}
%
We have implemented proofs of concepts of our approaches using the proof
assistant Coq in two projects called SpecCert and FreeSpec. Both have been
released as free software.
