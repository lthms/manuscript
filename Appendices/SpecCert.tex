\chapter{A Formal Definition of HSE Mechanisms in Coq}
\label{appendix:speccert}

This Appendix presents an implementation of the formal definition of HSE
mechanisms detailed in Chapter~\ref{chapter:speccert}, and follows a similar
outline.
%
The main purpose of this development is to provide rigorous, machine-checked
proofs of the lemmas and theorems discussed in the Chapter.
%
We assume the reader is familiar with Coq, and we discuss several key fragments
of the development.

\section{Hardware Model}

\subsection{Definition}

A hardware model in our formalism is a tuple
$\langle S, L_S, L_H, \rightarrow \rangle$
(Definition~\ref{def:speccert:model}), with $\rightarrow$ being a predicate on
$H \times (L_S \uplus L_H) \times H$.

The three sets $H$, $L_S$ and $L_H$ are introduced as variables of our
development.
%
This means they are implicit arguments of any further definition which uses
them.

\inputminted[gobble=2,firstline=2,lastline=2]{coq}{Listings/SpecCert.v}

The disjoint union $\uplus$ is modelled through a dedicated inductive type
called \texttt{label}.

\inputminted[gobble=2,firstline=4,lastline=8]{coq}{Listings/SpecCert.v}

Because $S$, $L_S$ and $L_H$ are \emph{implicit} arguments of our development, a
hardware model can be reduced to its relation transition~$\rightarrow$.

\inputminted[gobble=2,firstline=10,lastline=12]{coq}{Listings/SpecCert.v}

A transition in our formalism is a tuple $(h, l, h')$ which satisfies the
transition relation of the model.
%
Subsets in Coq are usually modelled using so-called sigma-type:
%
\texttt{\{ x: A | P x \}} is the subset of elements of type \texttt{A} which
satisfy the predicate \texttt{P}.
%
We define \texttt{transition m}, the set of transitions of a model \texttt{m},
using a sigma-type.

\inputminted[gobble=2,firstline=29,lastline=32]{coq}{Listings/SpecCert.v}

Because {\textsc Gallina} is a strongly-typed language, manipulating a sigma-type
can be cumbersome.
%
In particular, there is no implicit coercion from \texttt{\{ x: A | P x \}} to
\texttt{A} by default.
%
The function \texttt{proj1\_sig} can be used to explicitly coerce a sigma-type
value, and we leverage the \texttt{Notation} feature of Coq, in order to ease
the coercion.

\inputminted[gobble=2,firstline=34,lastline=35]{coq}{Listings/SpecCert.v}

That is, when we write \texttt{\#x}, Coq will unwrap the sigma-type.

\subsection{Traces}

The next step is to model traces (Definition~\ref{def:speccert:trace}).
%
We first introduce \texttt{sequence}, a parameterized type which cannot be
empty.
%
This simplifies several definitions, such as \texttt{init} (which returns the
initial state of a trace).

\inputminted[gobble=2,firstline=39,lastline=46]{coq}{Listings/SpecCert.v}

Not all sequences are traces, as a trace is a sequence where, given two
consecutive transitions, the initial state of the second one is the resulting
state of the first.
%
Similarly to the \texttt{transition} type, we define \texttt{trace} with a
sigma-type.
%
To define the predicate to distinguish between valid and invalid trace, we first
define \texttt{init} and \texttt{trace} as functions on \texttt{sequence
  (transition~m)}, with the set of transitions returned by \texttt{trace} is
modelled as a predicate on \texttt{transition m}.
%
Then, we define \texttt{is\_trace}, an inductive predicate on \texttt{sequence
  (transition~m)}.

\inputminted[gobble=2,firstline=62,lastline=71]{coq}{Listings/SpecCert.v}

Finally, we use the \texttt{is\_trace} predicate to define \texttt{trace m}.

\inputminted[gobble=2,firstline=73,lastline=76]{coq}{Listings/SpecCert.v}

\subsection{Security Policies}

\inputminted[gobble=2,firstline=90,lastline=92]{coq}{Listings/SpecCert.v}

\inputminted[gobble=2,firstline=94,lastline=99]{coq}{Listings/SpecCert.v}

\inputminted[gobble=2,firstline=101,lastline=109]{coq}{Listings/SpecCert.v}

\section{HSE Mechanisms}

\subsection{Definition and HSE Laws}

\inputminted[gobble=2,firstline=111,lastline=120]{coq}{Listings/SpecCert.v}

\inputminted[gobble=2,firstline=122,lastline=138]{coq}{Listings/SpecCert.v}

\subsection{Trace Compliance}

\inputminted[gobble=2,firstline=146,lastline=155]{coq}{Listings/SpecCert.v}

\inputminted[gobble=2,firstline=157,lastline=181]{coq}{Listings/SpecCert.v}

\inputminted[gobble=2,firstline=183,lastline=228]{coq}{Listings/SpecCert.v}

\subsection{HSE Mechanism Correctness}

\inputminted[gobble=2,firstline=230,lastline=235]{coq}{Listings/SpecCert.v}

\inputminted[gobble=2,firstline=237,lastline=259]{coq}{Listings/SpecCert.v}

\subsection{HSE Mechanism Composition}

\inputminted[gobble=2,firstline=400,lastline=406]{coq}{Listings/SpecCert.v}

\inputminted[gobble=2,firstline=408,lastline=421]{coq}{Listings/SpecCert.v}

\inputminted[gobble=2,firstline=422,lastline=426]{coq}{Listings/SpecCert.v}

\inputminted[gobble=2,firstline=427,lastline=442]{coq}{Listings/SpecCert.v}

\inputminted[gobble=2,firstline=470,lastline=510]{coq}{Listings/SpecCert.v}

\inputminted[gobble=2,firstline=512,lastline=535]{coq}{Listings/SpecCert.v}

\section{Case Study: Code Injection Policies}

\subsection{The Software Stack}

\inputminted[gobble=2,firstline=261,lastline=268]{coq}{Listings/SpecCert.v}

\inputminted[gobble=2,firstline=270,lastline=271]{coq}{Listings/SpecCert.v}

\subsection{Code Injection}

\inputminted[gobble=2,firstline=337,lastline=343]{coq}{Listings/SpecCert.v}

\subsection{Code Injection Policies}

\inputminted[gobble=2,firstline=273,lastline=280]{coq}{Listings/SpecCert.v}

\inputminted[gobble=2,firstline=345,lastline=351]{coq}{Listings/SpecCert.v}

\inputminted[gobble=2,firstline=537,lastline=543]{coq}{Listings/SpecCert.v}

\inputminted[gobble=2,firstline=545,lastline=552]{coq}{Listings/SpecCert.v}

\inputminted[gobble=2,firstline=554,lastline=582]{coq}{Listings/SpecCert.v}