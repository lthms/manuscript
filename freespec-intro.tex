\documentclass[oneside,a4paper]{memoir}
\setsecnumdepth{subsection}

\usepackage[chapter]{minted}
\usemintedstyle{bw}
\setminted{framesep=10pt,xleftmargin=18pt,frame=leftline,linenos=true}

% ABOUT THIS FILE
% ---------------
% The goal of this document is to prepare the next (and final?) version of the
% chapter 3.

\usepackage[sc]{mathpazo}    % Palatino with smallcaps
\usepackage[scaled]{helvet}  % Helvetica, scaled 95%
\usepackage{eulervm}
\usepackage{inconsolata}

\usepackage{phdcmd}
\usepackage{speccert}
\usepackage{freespec}

\usepackage{bigcenter}
\usepackage{geometry}
\usepackage{paralist}
\usepackage{hyperref}
\usepackage{amssymb}
\usepackage{amsmath}
\usepackage{bussproofs}

\usepackage[T1]{fontenc}
\usepackage[utf8]{inputenc}
\usepackage{inconsolata}

\usepackage[dvipsnames]{xcolor}
\usepackage{xargs}
\usepackage{todonotes}
\newcommandx{\thomasrk}[2][1=]{\todo[linecolor=Plum,backgroundcolor=Plum!25,bordercolor=Plum,#1]{#2}}

\usepackage{mdframed} % or, "mdframed"
\usepackage[amsthm,framed]{ntheorem}
\newcommand{\powerset}{\raisebox{.15\baselineskip}{\Large\ensuremath{\wp}}}

\definecolor{statementline}{HTML}{b7e2d6}

\theoremstyle{break}
\theorembodyfont{\fontshape\rmdefault\selectfont}

\mdfdefinestyle{quoted}{
hidealllines=true,
leftmargin=-15pt,
rightmargin=-15pt,
leftline=true,
innertopmargin=10pt,
innerbottommargin=10pt,
innerrightmargin=15pt,
linewidth=5pt,
linecolor=gray!40,
backgroundcolor=gray!3,
usetwoside=false,
skipabove=\topsep,
skipbelow=\topsep}

\mdfdefinestyle{definition}{
style=quoted,
linecolor=Violet!20,
backgroundcolor=Violet!2}

\mdfdefinestyle{statement}{
style=quoted,
linecolor=PineGreen!30,
backgroundcolor=PineGreen!2}

\newmdtheoremenv[style=definition]{definition}{Definition}[chapter]
\newmdtheoremenv{notation}{Notation}[chapter]
\newmdtheoremenv[style=quoted]{example}{Example}[chapter]
\newmdtheoremenv[style=statement]{corollary}{Corollary}[chapter]
\newmdtheoremenv[style=statement]{lemma}{Lemma}[chapter]
\newmdtheoremenv[style=statement]{theorem}{Theorem}[chapter]

\usepackage{tikz}
\usetikzlibrary{shapes.geometric, positioning, arrows, intersections, fit,
  matrix, shapes, shapes.symbols}

\usepackage{acro}
\DeclareAcronym{cpu}{
  short = CPU,
  long  = Central Processing Unit,
  class = abbrev
}
\DeclareAcronym{mmu}{
  short = MMU,
  long  = Memory Management Unit,
  class = abbrev
}
\DeclareAcronym{bios}{
  short = BIOS,
  long  = Basic Input/Output System,
  class = abbrev
}
\DeclareAcronym{smm}{
  short = SMM,
  long  = System Management Mode,
  class = abbrev
}
\DeclareAcronym{smi}{
  short = SMI,
  long  = System Management Interrupt,
  class = abbrev
}
\DeclareAcronym{apic}{
  short = APIC,
  long  = Advanced Programmable Interrupt Controller,
  class = abbrev
}
\DeclareAcronym{tcb}{
  short = TCB,
  long  = Trusted Computing Base,
  class = abbrev
}


\begin{document}

\chapter{FreeSpec}

\section{A Compositional Reasoning Framework}

\ac{hse}

\subsection{{\scshape Minx86} Limitations}

\paragraph{Abstract Model}
%
\begin{itemize}
\item Mix of a concrete and abstract model
\item As a consequence, the model is not really reusable
\item May work for similar security properties only
\item We need a clear separation between the two
\end{itemize}

\paragraph{Monolithic Model}
%
Hum?

\paragraph{Transition Granularity}
%
Some transitions of {\scshape Minx86} imply several state updates may happen in
sequence.
%
This is particularly true for \( \mathrm{Read} \) and \( \mathrm{Write} \)
transitions, as both the cache and memories are updated in case of cache misses
or cache evictions.

The complexity of a transition postcondition has a direct impact on the proofs
we have to construct in order to verify the correctness of a given system.
%
For instance, let \( M \) be a transition system, such that the transition
relation \( R \) of \( M \) is defined as follows:
%
\[
  R \triangleq \{ (h, h')\ |\ h' = t(s(r(h))) \}
\]
%
Let \( \iota \) be a predicate on \( M \) states, such that \( \iota \) is
expected to be an invariant with respect to \( R \), that is
%
\[
  h' = t(s(r(h))) \vdash \iota(h) \Rightarrow \iota(h')
\]
%
Regarding the definition of \( r \), \( s \) and \( t \), proving this statement
all at once can quickly become unmanageable.
%
One possible approach to face this challenge is to manually divide the
transition into several change updates ---\emph{i.e.} one per state update---
and prove that \( \iota \) is preserved throughout the transition, as
illustrated in Figure~\ref{freespec:figure:seqproof}.
%
Proving \( \Delta_{\mathtt{bios}} \) was correct with respect to
\( I_{\mathtt{bios}} \) using Coq led us to successfully adopt this approach.
%
In practice, we relied on a recurring pattern we call
``\texttt{remember};~\texttt{destruct};~\texttt{assert};~\texttt{apply}'', after
the {\scshape Ltac} tactics.
%
We illustrate the use of this pattern with the proof obligation with the proof
goal pictured in Figure~\ref{fig:freespec:postcondition}.
%
We have been extracted this goal directly from the SpecCert project; we have
only slightly simplified it so it can fit into the page.
%
\begin{enumerate}
\item Using \texttt{remember}, we can make the sequence of state updates more
  visible (see Figure~\ref{fig:freespec:remember}, compared to
  Figure~\ref{fig:freespec:entrypoint}).
  %
  As a result, the Coq goal is more readable, which greatly simplify the task to
  write the proof.
\item Using \texttt{destruct}, we can explore the alternative paths.
  %
  For instance, regarding the initial state of the system, the content of the
  address \( \mathtt{pa} \) may or may not be present in the cache
  (Figure~\ref{fig:freespec:remember}, line 7).
  %
  We explore both solution, though two complementary goals (see the two
  simplified goals in Figure~\ref{fig:freespec:destruct})
  %
\item Using \texttt{assert}, we can introduce now goals to prove intermediary
  results (like the one in Figure~\ref{fig:freespec:assert}, for instance).
  %
  In our case, we will prove that the predicate \texttt{inv} remains satisfied
  after each state update.
  %
\item Using \texttt{apply}, we can leverage lemmas about \texttt{inv}
  preservation for a given state update.
  %
  For instance, the function \texttt{update\_cache\_content}
  (Figure~\ref{fig:freespec:remember}, line 11) have a lemma called
  \texttt{update\_cache\_content\_with\_context\_preserves\_inv}.
\end{enumerate}

\begin{figure}
{\bigcentering%
  \AxiomC{}%
  \LeftLabel{\footnotesize {\scshape Id}} \UnaryInfC{\(
    \begin{array}{l}
      \iota(h) \\
      \hline
      \vdash \iota(h)
    \end{array}
    \)}%
  \AxiomC{}%
  \RightLabel{\footnotesize {\scshape Inv}\( _r \)} \UnaryInfC{\(
    \begin{array}{l}
      \iota(h) \\
      h_1 = r(h) \\
      \hline
      \vdash \iota(h) \Rightarrow \iota(h_1)
    \end{array}
    \)}%
  \LeftLabel{\footnotesize {\scshape M-P}}%
  \BinaryInfC{\(
    \begin{array}{l}
      \iota(h) \\
      h_1 = r(h) \\
      \hline
      \vdash \iota(h_1)
    \end{array}
    \)}%
  \AxiomC{}%
  \RightLabel{\footnotesize {\scshape Inv}\( _s \)} \UnaryInfC{\(
    \begin{array}{l}
      \iota(h) \\
      h_1 = r(h) \\
      h_2 = s(h_1) \\
      \hline
      \vdash \iota(h_1) \Rightarrow \iota(h_2)
    \end{array}
    \)}%
  \LeftLabel{\footnotesize {\scshape M-P}}%
  \BinaryInfC{\(
    \begin{array}{l}
      \iota(h) \\
      h_1 = r(h) \\
      h_2 = s(h_1) \\
      \hline
      \vdash \iota(h_2)
    \end{array}
    \)}%
  \AxiomC{}%
  \RightLabel{\footnotesize {\scshape Inv}\( _t \)} \UnaryInfC{\(
    \begin{array}{l}
      \iota(h) \\
      h_1 = r(h) \\
      h_2 = s(h_1) \\
      h' = t(h_2) \\
      \hline \vdash \iota(h_2) \Rightarrow \iota (h')
    \end{array}
    \)}%
  \RightLabel{\footnotesize {\scshape M-P}}%
  \BinaryInfC{\(
    \begin{array}{l}
      \iota(h) \\
      h_1 = r(h) \\
      h_2 = s(h_1) \\
      h' = t(h_2) \\
      \hline \vdash \iota (h')
    \end{array}
    \)}%
  \DisplayProof%

  \vspace{0.5cm} where {\scshape M-P} is the \emph{Modus ponens}, and {\scshape
    Inv}\( _r \), {\scshape Inv}\( _s \), and {\scshape Inv}\( _t \) are
  intermediary lemmas previously proved true.%
}

\caption{Dividing a transition into a sequences of state updates}
\label{freespec:figure:seqproof}
\end{figure}

\begin{figure}
  \inputminted{coq}{Listings/StateHell.v}

  \caption{Raw postcondition of a \texttt{Write} transition with a writeback
    strategy}
  \label{fig:freespec:entrypoint}
\end{figure}

\begin{figure}
  \inputminted{coq}{Listings/RememberPattern.v}

  \caption{Postcondition of a \texttt{Write} transition with a writeback
    strategy, after the use of the \texttt{remember} tactic}
  \label{fig:freespec:remember}
\end{figure}

\begin{figure}
  \inputminted{coq}{Listings/DestructGoal.v}

  \inputminted{coq}{Listings/DestructGoal2.v}

  \caption{Exploring alternative paths thanks to the \texttt{destruct} tactics}
  \label{fig:freespec:destruct}
\end{figure}

\begin{figure}
  \inputminted{coq}{Listings/InvGoals.v}

  \caption{Intermediary statements, generated thanks to \texttt{assert} tactics}
  \label{fig:freespec:assert}
\end{figure}

The ``\texttt{remember};~\texttt{destruct};~\texttt{assert};~\texttt{apply}''
structures the proof, but it remains cumbersome to use.
%
Besides, it emphasizes another important limitation of {\scshape Minx86}, that
is it is potentially subject to tho so-called \emph{temporary violation
  problem}\,\cite{muller2006tempviolation}, where a predicate is satisfied
\emph{before} and \emph{after} a given transition, yet is temporary violated
\emph{during} the transition.
%
For instance, if it exists a state \( h \) such that \( \neg r(h) \) or
\( \neg s(r(h)) \) holds true, then \( M \) is subject to the temporary
violation problem.
%
A well known illustration of this threat is ---once again--- the airlock system.

\begin{example}[Temporary Violation Problem in an Airlock System]
  %
  In Example~\ref{example:sota:airlocklts}, we have carefully defined the
  labeled transition system of the airlock system so that it is unambiguously
  not subject to the temporary violation problem.
  %
  We did that by avoiding transitions between states
  \( (\mathtt{open}, \mathtt{close}) \) and
  \( (\mathtt{close}, \mathtt{open}) \).
  %
  In practice, it is possible to define a transition system with these
  transitions, such that it is correct with respect to the safety property that
  at least one door should be close at anytime, for instance:

  \begin{center}
    \begin{tikzpicture}
      \node [draw, circle split, text width=30pt, text badly centered] (cc)
      {\( \mathtt{close} \) \nodepart{lower} \( \mathtt{close} \)};%
      \node [right=60pt of cc] (x) {};%
      \node [draw, circle split, above=of x, text width=30pt, text badly
      centered] (oc) {\( \mathtt{open} \) \nodepart{lower}
        \( \mathtt{close} \)};%
      \node [draw, circle split, below=of x, text width=30pt, text badly
      centered] (co) {\( \mathtt{close} \) \nodepart{lower}
        \( \mathtt{open} \)};%
      \node [draw, circle split, right=60pt of x, text width=30pt, text badly
      centered] (oo) {\( \mathtt{open} \) \nodepart{lower} \( \mathtt{open}
        \)};%

      \draw [-latex] (cc) edge [bend left] node [xshift=-5pt, left]
      {\( \mathtt{Req}_1 \)} (oc);%

      \draw [-latex] (cc) edge [bend right] node [xshift=-5pt, left]
      {\( \mathtt{Req}_2 \)} (co);%

      \draw [-latex] (oc) edge [bend left] node [right] {\( \mathtt{Req}_2 \)}
      (co);%
      \draw [-latex] (co) edge [bend left] node [left] {\( \mathtt{Req}_1 \)}
      (oc);%
    \end{tikzpicture}
  \end{center}

  We can imagine several scenarios:
  %
  \begin{inparaenum}[(1)]
  \item Doors states updates are simultaneous and instantaneous, in other words
    the transitions \emph{are} atomic.
    %
  \item The airlock system first closes one door before opening the
    other. \label{enum:freespec:goodairlock}
    %
  \item The airlock system first opens one door before closing the
    other. \label{enum:freespec:badairlock}
    %
  \end{inparaenum}
  %
  The systems modeled in Examples~\ref{example:sota:airlocklts},
  ~\ref{example:sota:airlockinterface} and~\ref{example:sota:airlockprocess}
  corresponds to the scenario (\ref{enum:freespec:goodairlock}).
  %
  On the contrary, the transition system pictures in this example is not precise
  enough to discard the eventuality that both doors are, at a given point in
  time, both open.
\end{example}

Although the \texttt{remember;~destruct;~assert;~apply} pattern can help to
uncover temporary violation of an invariant, nothing prevents the verifier from
forgetting a given state updates and yet being able to conclude the proof.
%
The temporary violation problem undermines the second \ac{hse} law, which
stipulates that the requirement on hardware state of a \ac{hse} mechanism is an
invariant of the model with respect to the requirement on software transitions
(see Definition~\ref{def:speccert:laws}).
%
For the \ac{hse} mechanism \( \Delta_{\mathtt{bios}} \) ---defined as
\( \langle S, T, \mathrm{context}, \mathrm{hardware\_req}, \mathrm{bios\_req}
\rangle \), see Definition~\ref{def:speccert2:minx86}--- this translates as
follows:
%
\[
  \begin{array}{l}
    \forall (h, l, h') \in H(S) \times L \times H(S), \\
    \qquad\func{hardware\_req}(h) \\
    \qquad\Rightarrow (l \in L_S \Rightarrow \func{bios\_req}(h, l)) \\
    \qquad\Rightarrow \func{hardware\_req}(h')
  \end{array}
\]
%
In practice, transitions in {\scshape Minx86} are not atomic (they are made of
several state updates), and an attacker may be able to leverage a temporary
violation of invariant, similarly to the SENTER Sandman
attack\,\cite{kovah2015senter} detailed in
Subsection~\ref{subsec:usecase:hse:sandman}.
%
To model that threat without changing the modeling structure of {\scshape
  Minx86} requires to increase the granularity of the transitions, with the
direct consequence to increase the model overall complexity.

\bibliographystyle{unsrt}%
\bibliography{manuscript}
\end{document}