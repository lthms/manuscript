\chapter{A Formal Definition of HSE Mechanisms in Coq}
\label{appendix:speccert}

This Appendix presents an implementation of the formal definition of HSE
mechanisms detailed in Chapter~\ref{chapter:speccert}, and follows a similar
outline.
%
The main purpose of this development is to provide rigorous, machine-checked
proofs of the lemmas and theorems discussed in the Chapter.
%
We assume the reader is familiar with Coq, and we discuss several key fragments
of the development.

\section{Hardware Model}

\subsection{Definition}

A hardware model in our formalism is a tuple
$\langle H, L_S, L_H, \rightarrow \rangle$
(Definition~\ref{def:speccert:model}), with $\rightarrow$ being a predicate on
$H \times (L_S \uplus L_H) \times H$.

The three sets $H$, $L_S$ and $L_H$ are introduced as variables of our
development.
%
This means they are implicit arguments of any further definition which uses
them.

\inputminted[gobble=2,firstline=2,lastline=2]{coq}{Listings/SpecCert.v}

The disjoint union $\uplus$ is modeled through a dedicated inductive type
called \texttt{label}.

\inputminted[gobble=2,firstline=4,lastline=8]{coq}{Listings/SpecCert.v}

Because $H$, $L_S$ and $L_H$ are \emph{implicit} arguments of our development, a
hardware model can be reduced to its relation transition~$\rightarrow$.

\inputminted[gobble=2,firstline=10,lastline=12]{coq}{Listings/SpecCert.v}

A transition in our formalism is a tuple $(h, l, h')$ which satisfies the
transition relation of the model.
%
Subsets in Coq are usually modeled using so-called sigma-type:
%
\texttt{\{ x: A | P x \}} is the subset of elements of type \texttt{A} which
satisfy the predicate \texttt{P}.
%
We define \texttt{transition m}, the set of transitions of a model \texttt{m},
using a sigma-type.

\inputminted[gobble=2,firstline=29,lastline=32]{coq}{Listings/SpecCert.v}

Because {\textsc Gallina} is a strongly typed language, manipulating a sigma-type
can be cumbersome.
%
In particular, there is no implicit coercion from \texttt{\{ x: A | P x \}} to
\texttt{A} by default.
%
The function \texttt{proj1\_sig} can be used to explicitly coerce a sigma-type
value, and we leverage the \texttt{Notation} feature of Coq, in order to ease
the coercion.

\inputminted[gobble=2,firstline=34,lastline=35]{coq}{Listings/SpecCert.v}

That is, when we write \texttt{\#x}, Coq will unwrap the sigma-type.

\subsection{Traces}

The next step is to model traces (Definition~\ref{def:speccert:trace}).
%
We first introduce \texttt{sequence}, a parameterized type which cannot be
empty.
%
This simplifies several definitions, such as \texttt{init} (which returns the
initial state of a trace).

\inputminted[gobble=2,firstline=39,lastline=46]{coq}{Listings/SpecCert.v}

Not all sequences are traces, as a trace is a sequence where, given two
consecutive transitions, the initial state of the second one is the resulting
state of the first.
%
Similarly to the \texttt{transition} type, we define \texttt{trace} with a
sigma-type.
%
To define the predicate to distinguish between valid and invalid trace, we first
define \texttt{init} and \texttt{trace} as functions on \texttt{sequence
  (transition~m)}, with the set of transitions returned by \texttt{trace} is
modeled as a predicate on \texttt{transition m}.
%
Then, we define \texttt{is\_trace}, an inductive predicate on \texttt{sequence
  (transition~m)}.

\inputminted[gobble=2,firstline=62,lastline=71]{coq}{Listings/SpecCert.v}

Finally, we use the \texttt{is\_trace} predicate to define \texttt{trace m}.

\inputminted[gobble=2,firstline=73,lastline=76]{coq}{Listings/SpecCert.v}

\subsection{Security Policies}

We have detailed how security policies can be modeled in transition systems, in
Subsection~\ref{subsec:sota:security} for the general case and in
Subsection~\ref{subsec:speccert:security} for the context of HSE mechanisms.
%
A security policy is either a predicate on sets of traces, a predicate on
traces or a predicate on transitions.
%
In this development, we keep the former (predicate on sets of traces) as the
generic definition.

\inputminted[gobble=2,firstline=90,lastline=92]{coq}{Listings/SpecCert.v}

We then express the two latter (predicate on traces, and predicate on transitions)
as particular sub-cases of this generic definition.

\inputminted[gobble=2,firstline=94,lastline=99]{coq}{Listings/SpecCert.v}

\inputminted[gobble=2,firstline=101,lastline=110]{coq}{Listings/SpecCert.v}

\section{HSE Mechanisms}

We now give a formal definition of HSE mechanisms in the Coq theorem prover,
as stated in Definition~\ref{def:speccert:hse}.

\subsection{Definition and HSE Laws}

First, we introduce an helper definition to easily express predicates of the
form \( l \in L_S \Rightarrow P(l) \).

\inputminted[gobble=2,firstline=112,lastline=121]{coq}{Listings/SpecCert.v}

The \texttt{if\_software} allows for hiding the pattern matching that is
necessary to express various definitions of our theory of HSE mechanisms.
%
Then, we define a type of HSE mechanisms using the Coq \texttt{Record} syntax,
as follows.

\inputminted[gobble=2,firstline=123,lastline=139]{coq}{Listings/SpecCert.v}

By making the two laws part of the HSE mechanisms definition, we ensure that no
inconsistent HSE mechanism can be defined in Coq.

\subsection{Trace Compliance}

Whether trusted software components are correctly implementing a given HSE
mechanism is a safety property.
%
This means we shall be able to derive a subset of ``compliant'' traces from the
set of traces of the hardware model.

\inputminted[gobble=2,firstline=147,lastline=156]{coq}{Listings/SpecCert.v}

This definition is straightforward: for a trace to be compliant with a HSE
mechanism, its initial state has to comply with the hardware requirements of
this HSE mechanism, while its software transitions (that is, transitions whose
label belongs to \texttt{Ls}) satisfy its software requirements.

In order to prove Lemma~\ref{lemma:speccert:hseinv}, we first prove an
intermediary result: if a trace complies with a given HSE mechanism, then the
subtrace obtained by removing its first transition also complies with the HSE
mechanism.

\inputminted[gobble=2,firstline=158,lastline=180]{coq}{Listings/SpecCert.v}

We use this result in the proof by induction of
Lemma~\ref{lemma:speccert:hseinv}.
%
This lemma states that hardware requirements of a consistent HSE mechanism are
invariant of compliant traces.

\inputminted[gobble=2,firstline=184,lastline=196]{coq}{Listings/SpecCert.v}

The first case to consider is the singleton sequence, with only one transition.
%
By definition of the compliant traces, the initial state of the only transition
of \texttt{rho} satisfies \texttt{hardware\_req}.
%
In order to prove that the resulting state of this transition also satisfies the
requirement, we use the second law of the HSE mechanism definition.

\inputminted[gobble=2,firstline=197,lastline=202]{coq}{Listings/SpecCert.v}

The second case to consider is a trace made of an initial transition and a
subtrace.
%
If we consider the associated set of transitions, we again have to cover two
cases.
%
Firstly, we can focus on the initial transition: the proof is here very similar to
the singleton trace case.
%
Secondly, we can consider the transitions of the subtrace.
%
We know this subtrace is compliant using \texttt{compliant\_trace\_rec}.
%
This allows us to use the induction hypothesis, and conclude the proof.

\inputminted[gobble=2,firstline=203,lastline=215]{coq}{Listings/SpecCert.v}

\subsection{HSE Mechanism Correctness}

\inputminted[gobble=2,firstline=231,lastline=236]{coq}{Listings/SpecCert.v}

\inputminted[gobble=2,firstline=238,lastline=260]{coq}{Listings/SpecCert.v}

\subsection{HSE Mechanisms Composition}

HSE mechanisms are commonly implemented simultaneously by modern software stack,
\emph{e.g.} an operating system configures the MMU, while firmware relies on the
processor \ac{smm}.
%
In Chapter~\ref{chapter:speccert}, we have proposed a first definition of a
composition operator to reason about such scenario.
%
This operator is not total, and cannot compose together two arbitrary HSE
mechanisms.
%
Therefore, we first define a predicate to determine whether two HSE mechanisms
are compatible or not.
%
The main characteristic of two compatible HSE mechanisms is that they use the
same hardware-software mapping.
%
Maybe surprisingly for the reader unfamiliar with Coq, this property is not
straightforward to express and requires to use the \texttt{eq\_rect} function
which allows to cast a value from one type to another, assuming we can provide
the proof both types are actually the same.

\inputminted[gobble=2,firstline=401,lastline=407]{coq}{Listings/SpecCert.v}

We then define the \texttt{HSE\_cap} function (\texttt{cap} being the name of
the latex command of \( \cap\)) which computes the composition of two compatible
HSE mechanisms. We rely on the \texttt{refine} tactic in order to write the
proofs of consistency of the resulting HSE mechanisms in interactive mode (using
tactics) rather than providing them directly.

\inputminted[gobble=2,firstline=409,lastline=423]{coq}{Listings/SpecCert.v}

As stated in Definition~\ref{def:speccert:laws}, we prove the first law...

\inputminted[gobble=2,firstline=429,lastline=445]{coq}{Listings/SpecCert.v}

... and the second law (\texttt{hardware\_req} is an invariant).

\inputminted[gobble=2,firstline=424,lastline=428]{coq}{Listings/SpecCert.v}

We discussed in Chapter~\ref{chapter:speccert} several expected properties of
the HSE mechanisms composition.
%
For instance, Lemma~\ref{lemma:speccert:compinter} states that the set of
compliant traces of the compositions of two HSE mechanisms is the intersection
of the sets of compliant traces of each HSE mechanism.
%
In the context of this development, sets are defined as predicates (this is a
common approach in Coq).
%
To reason about set equality, we prove two complementary implications.
%
Firstly, a compliant trace of the composition of two HSE mechanisms necessarily
complies with one of these HSE mechanisms.
%
The associated proof covers both possible cases whose are very
similar\,\footnote{We can probably reduce the size of the proof by half with
  some refactoring.}.

\inputminted[gobble=2,firstline=472,lastline=512]{coq}{Listings/SpecCert.v}

Secondly, a trace which complies with one of two HSE mechanisms also complies
with their composition.

\inputminted[gobble=2,firstline=514,lastline=537]{coq}{Listings/SpecCert.v}

\section{Case Study: Code Injection Policies}

\subsection{The Software Stack}

We consider the execution of a software stack made of a firmware component, an
operating system and an infinite number of applications, identified by a natural
number.

\inputminted[gobble=2,firstline=262,lastline=269]{coq}{Listings/SpecCert.v}

Because our hardware model is left as a parameter of this Coq development, so
are the hardware-software mapping and transition-software mapping that we will
use to reason about the software stack execution.

\inputminted[gobble=2,firstline=271,lastline=272]{coq}{Listings/SpecCert.v}

\subsection{Code Injection}

Implementing the code injection definition
(Definition~\ref{def:speccert:tempering}) in Gallina is straightforward.

\inputminted[gobble=2,firstline=338,lastline=344]{coq}{Listings/SpecCert.v}

\subsection{Code Injection Policies}

The code injection policy is defined against a relation between software
components to tell whether one software component is authorized to perform a
code injection against another.
%
In our case, we implement this relation as an inductive predicate.

\inputminted[gobble=2,firstline=274,lastline=281]{coq}{Listings/SpecCert.v}

The proof that the relation \texttt{stack\_ge} is antisymmetric relies on a
case analysis.

\inputminted[gobble=2,firstline=586,lastline=592]{coq}{Listings/SpecCert.v}

Using these components, the definition of the code injection policy is simple.

\inputminted[gobble=2,firstline=346,lastline=352]{coq}{Listings/SpecCert.v}

As explained in Chapter~\ref{chapter:speccert}, the enforcement of such security
policy is done thanks to several HSE mechanisms.
%
A first HSE mechanism will be devoted to enforce that the \ac{bios} remains
isolated from the rest of the software stack, that is it will enforce the
following policy:

\inputminted[gobble=2,firstline=539,lastline=545]{coq}{Listings/SpecCert.v}

A second HSE mechanism will be devoted to enforce that the applications cannot
perform illegitimate code injection against the rest of the software stack, that
is:

\inputminted[gobble=2,firstline=547,lastline=554]{coq}{Listings/SpecCert.v}

Our theory of HSE mechanisms allows us to prove that the concurrent
implementation of two HSE mechanisms ---correct with respect to their respective
security policy--- is a sufficient condition for the enforcement ``global'' code
injection policies, by demonstrating that the composition of these HSE
mechanisms is correct with respect to the code injection policy.

\inputminted[gobble=2,firstline=556,lastline=583]{coq}{Listings/SpecCert.v}
