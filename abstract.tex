%!TEX root = ./main.tex
In this thesis, we consider a class of security enforcement mechanisms we called
\emph{Hardware-based Security Enforcement} (HSE).
%
In such mechanisms, some trusted software components rely on the underlying
hardware architecture to constrain the execution of untrusted software
components with respect to targeted security policies.
%
For instance, an Operating System that sets up page tables to isolate userland
applications implements a HSE mechanism.

For a HSE mechanism to correctly enforce a targeted security policy, it requires
both hardware and trusted software components to play their parts.
%
During the past decades, several vulnerability disclosures have defeated HSE
mechanisms.
%
We focus on the vulnerabilities that are the result of errors at the
specification level, rather than implementation errors.
%
In some critical vulnerabilities, the attacker makes a legitimate use of one
hardware component to circumvent the HSE mechanism provided by another one.
%
For instance, cache poisoning attacks leverage inconsistencies between cache
and DRAM's access control mechanisms.
%
We call this class of security vulnerabilities \emph{architectural attacks} \TODO{Ce n'est pas très cohérent d'appeler des vulnerabilité des "attaques". Pour être cohérent, on pourrait dire que l'exploitation de ces vulnérabilités de spécification s'appelle des "architectural attacks".}.

Our goal is to explore approaches to specify and verify HSE mechanisms using
formal methods.
%
We believe this would benefit both hardware designers and software developers.
%
Firstly, a formal specification of HSE mechanisms can be leveraged as a
foundation for a systematic approach to verify hardware specifications, in the
hope of uncovering potential architectural attacks ahead of time.
%
Secondly, it provides unambiguous specifications to software developers, in the
form of a list of requirements.
%
These requirements have to be satisfied by trusted software components in order
for the hardware architecture to enforce the targeted security policy.
%
We believe these specifications can be a valuable addition to the existing
documentation, and should eventually be leveraged in a software verification
process as well. \TODO{Je suis en phase avec l'argumentaire développé ici (firstly... et secondly...). Tu devrais le reprendre (en le développant) dans ta conclusion en section 5.5 ou l'arguentaire est moins claire et ne semble pas "calqué" sur celui là (il devrait).}

Our contribution is two-fold:
%
\begin{itemize}
\item We propose a formal definition of HSE mechanisms against hardware
  architecture models. This definition can be used to specify and verify such mechanisms.
  %
  To evaluate our approach, we propose a minimal model for a single core
  x86-based computing platform.
  %
  We use it to specify and verify the HSE mechanism provided by Intel to isolate
  the code executed while the CPU is in System Management Mode (SMM), a highly
  privileged execution mode of x86 microprocessors.
  %
  We have written machine-checked proofs in the Coq proof assistant to that
  end.
\item We propose a novel approach to verify compositions of components inspired
  by algebraic effects, to enable modular verification of complex hardware
  architectures. \TODO{Cette description est un peu vague et surtout ne fait pas assez le lien avec la contribution précédente.}
  %
  This approach is not specific to hardware models, and could also be leveraged
  to reason about software applications as well.
  %
  In addition, we have implemented FreeSpec, a framework for the Coq proof
  assistant. \TODO{Tu nommes explicitement FreeSpec, en tant que "framework" alors que tu ne nommes pas "SpecCert et que tu le cantonnes à des preuves vérifiées en Coq. Ce sont deux travaux de la même importances pour moi et qui devrait être présentée de manière homogène}
  %
  FreeSpec is a complete implementation of our approach, and takes advantages of
  the proof automation features of Coq to ease the verification process. \TODO{C'est aussi le cas de SpecCert. Peut-être virer cette phrase ou englober les deux implémentations}
\end{itemize}

\paragraph{Keywords:}
%
Security $\bullet$ Hardware Verification $\bullet$ Formal Specification
$\bullet$ Formal Methods $\bullet$ Coq
