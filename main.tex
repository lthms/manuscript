%%%%%%%%%%%%%%%%%%%%%%%%%%%%%%%%%%%%%%%%%
% Masters/Doctoral Thesis
% LaTeX Template
% Version 2.5 (27/8/17)
%
% This template was downloaded from:
% http://www.LaTeXTemplates.com
%
% Version 2.x major modifications by:
% Vel (vel@latextemplates.com)
%
% This template is based on a template by:
% Steve Gunn (http://users.ecs.soton.ac.uk/srg/softwaretools/document/templates/)
% Sunil Patel (http://www.sunilpatel.co.uk/thesis-template/)
%
% Template license:
% CC BY-NC-SA 3.0 (http://creativecommons.org/licenses/by-nc-sa/3.0/)
%
%%%%%%%%%%%%%%%%%%%%%%%%%%%%%%%%%%%%%%%%%

%----------------------------------------------------------------------------------------
%	PACKAGES AND OTHER DOCUMENT CONFIGURATIONS
%----------------------------------------------------------------------------------------

\documentclass[
11pt, % The default document font size, options: 10pt, 11pt, 12pt
%oneside, % Two side (alternating margins) for binding by default, uncomment to switch to one side
english, % ngerman for German
singlespacing, % Single line spacing, alternatives: onehalfspacing or doublespacing
%draft, % Uncomment to enable draft mode (no pictures, no links, overfull hboxes indicated)
%nolistspacing, % If the document is onehalfspacing or doublespacing, uncomment this to set spacing in lists to single
%liststotoc, % Uncomment to add the list of figures/tables/etc to the table of contents
%toctotoc, % Uncomment to add the main table of contents to the table of contents
%parskip, % Uncomment to add space between paragraphs
%nohyperref, % Uncomment to not load the hyperref package
headsepline, % Uncomment to get a line under the header
%chapterinoneline, % Uncomment to place the chapter title next to the number on one line
%consistentlayout, % Uncomment to change the layout of the declaration, abstract and acknowledgements pages to match the default layout
]{MastersDoctoralThesis} % The class file specifying the document structure

\usepackage[utf8]{inputenc} % Required for inputting international characters
\usepackage[T1]{fontenc} % Output font encoding for international characters

\usepackage{mathpazo} % Use the Palatino font by default

\usepackage[backend=bibtex,style=numeric,natbib=true]{biblatex} % Use the bibtex backend with the authoryear citation style (which resembles APA)

\addbibresource{manuscript.bib} % The filename of the bibliography

\usepackage[autostyle=true]{csquotes} % Required to generate language-dependent quotes in the bibliography

\usepackage{amssymb}
\usepackage{amsmath}
\usepackage{bussproofs}
\usepackage{paralist}

\usepackage{bigcenter}
\usepackage{speccert}

\newtheorem{definition}{Definition}
\newtheorem{lemma}{Lemma}
\newtheorem{theorem}{Theorem}
\newtheorem{proof}{Proof}

%----------------------------------------------------------------------------------------
%	MARGIN SETTINGS
%----------------------------------------------------------------------------------------

\geometry{
	paper=a4paper, % Change to letterpaper for US letter
	inner=2.5cm, % Inner margin
	outer=3.8cm, % Outer margin
	bindingoffset=.5cm, % Binding offset
	top=1.5cm, % Top margin
	bottom=1.5cm, % Bottom margin
	%showframe, % Uncomment to show how the type block is set on the page
}

%----------------------------------------------------------------------------------------
%	THESIS INFORMATION
%----------------------------------------------------------------------------------------

\thesistitle{Specifying and Verifying Hardware-based Security Enforcement Mechanisms}
\supervisor{Ludovic \textsc{Mé}} % Your supervisor's name, this is used in the title page, print it elsewhere with \supname
\examiner{} % Your examiner's name, this is not currently used anywhere in the template, print it elsewhere with \examname
\degree{Doctor of Computer Science} % Your degree name, this is used in the title page and abstract, print it elsewhere with \degreename
\author{Thomas \textsc{Letan}} % Your name, this is used in the title page and abstract, print it elsewhere with \authorname
\addresses{} % Your address, this is not currently used anywhere in the template, print it elsewhere with \addressname

\subject{Computer Sciences} % Your subject area, this is not currently used anywhere in the template, print it elsewhere with \subjectname
\keywords{} % Keywords for your thesis, this is not currently used anywhere in the template, print it elsewhere with \keywordnames
\university{\href{http://www.centralesupelec.fr}{CentraleSupélec}} % Your university's name and URL, this is used in the title page and abstract, print it elsewhere with \univname
\department{\href{http://department.university.com}{CentraleSupélec Rennes}} % Your department's name and URL, this is used in the title page and abstract, print it elsewhere with \deptname
\group{\href{https://team.inria.fr/cidre/}{CIDRE}} % Your research group's name and URL, this is used in the title page, print it elsewhere with \groupname
\faculty{\href{http://faculty.university.com}{CentraleSupélec Rennes}} % Your faculty's name and URL, this is used in the title page and abstract, print it elsewhere with \facname

\AtBeginDocument{
\hypersetup{pdftitle=\ttitle} % Set the PDF's title to your title
\hypersetup{pdfauthor=\authorname} % Set the PDF's author to your name
\hypersetup{pdfkeywords=\keywordnames} % Set the PDF's keywords to your keywords
\hypersetup{allcolors=BlueViolet}
}

\begin{document}

\frontmatter % Use roman page numbering style (i, ii, iii, iv...) for the pre-content pages

\pagestyle{plain} % Default to the plain heading style until the thesis style is called for the body content

% ----------------------------------------------------------------------------------------
% TITLE PAGE
% ----------------------------------------------------------------------------------------

\begin{titlepage}
  \begin{center}

    \vspace*{.06\textheight} {\scshape\LARGE
      \univname\par}\vspace{1.5cm} % University name
    \textsc{\Large Doctoral Thesis}\\[0.5cm] % Thesis type

    \HRule \\[0.4cm] % Horizontal line
    {\huge \bfseries \ttitle\par}\vspace{0.4cm} % Thesis title
    \HRule \\[1.5cm] % Horizontal line

    \begin{minipage}[t]{0.4\textwidth}
      \begin{flushleft} \large
	\emph{Author:}\\
	\href{http://www.johnsmith.com}{\authorname} % Author name - remove the \href bracket to remove the link
      \end{flushleft}
    \end{minipage}
    \begin{minipage}[t]{0.4\textwidth}
      \begin{flushright} \large
	\emph{Supervisor:} \\
	\href{http://www.jamessmith.com}{\supname} % Supervisor name - remove the \href bracket to remove the link
      \end{flushright}
    \end{minipage}\\[3cm]

    \vfill

    \large \textit{A thesis submitted in fulfillment of the requirements\\ for
      the degree of \degreename}\\[0.3cm] % University requirement text
    \textit{in the}\\[0.4cm]
    \groupname\\\deptname\\[2cm] % Research group name and department name

    \vfill

    {\large \today}\\[4cm] % Date
    % \includegraphics{Logo} % University/department logo - uncomment to place it

    \vfill
  \end{center}
\end{titlepage}

% ----------------------------------------------------------------------------------------
% DECLARATION PAGE
% ----------------------------------------------------------------------------------------

\begin{declaration}
  \addchaptertocentry{\authorshipname} % Add the declaration to the table of contents
  \noindent I, \authorname, declare that this thesis titled, \enquote{\ttitle}
  and the work presented in it are my own. I confirm that:

  \begin{itemize}
  \item This work was done wholly or mainly while in candidature for a research
    degree at this University.
  \item Where any part of this thesis has previously been submitted for a degree
    or any other qualification at this University or any other institution, this
    has been clearly stated.
  \item Where I have consulted the published work of others, this is always
    clearly attributed.
  \item Where I have quoted from the work of others, the source is always
    given. With the exception of such quotations, this thesis is entirely my own
    work.
  \item I have acknowledged all main sources of help.
  \item Where the thesis is based on work done by myself jointly with others, I have made clear exactly what was done by others and what I have contributed myself.\\
  \end{itemize}

  \noindent Signed:\\
  \rule[0.5em]{25em}{0.5pt} % This prints a line for the signature

  \noindent Date:\\
  \rule[0.5em]{25em}{0.5pt} % This prints a line to write the date
\end{declaration}

\cleardoublepage

% ----------------------------------------------------------------------------------------
% QUOTATION PAGE
% ----------------------------------------------------------------------------------------

\vspace*{0.2\textheight}

\noindent\enquote{\itshape Thanks to my solid academic training, today I can
  write hundreds of words on virtually any topic without possessing a shred of
  information, which is how I got a good job in journalism.}\bigbreak

\hfill Dave Barry

% ----------------------------------------------------------------------------------------
% ABSTRACT PAGE
% ----------------------------------------------------------------------------------------

\begin{abstract}
  \addchaptertocentry{\abstractname} % Add the abstract to the table of contents
  In this thesis, we consider \emph{Hardware-based Security Enforcement} (HSE)
  mechanisms, a class of security enforcement mechanisms such that trusted
  software components rely on the underlying hardware architecture to constrain
  the execution of untrusted software components with respect to targeted
  security properties.

  HSE mechanisms are particularly used to keep the software components of modern
  software stacks isolated from each others.
  %
  As a consequence, hardware components play an essential role in securing
  computing platforms.
  %
  Yet during the past decades, several critical vulnerability disclosures have
  show that trust in hardware can be misplaced.
  %
  Modern computing platform have grown in complexity, and now involve multiple
  hardware components executing sophisticated software stacks.
  %
  This paves the way for \emph{architectural attacks}, where the attacker is
  able to leverage one component to threaten the security enforced by another.

  With SpecCert, we have proposed a three-step methodology, supported by a Coq
  framework, for specifying and verifying HSE mechanisms against hardware
  architecture models.
  %
  We have written a minimal model for single core x86-based computing platforms,
  and used SpecCert to specify and verify a real HSE mechanism to isolate the
  code executed while the CPU is in System Management Mode (SMM), a highly
  privileged execution mode of x86 microprocessors.

  For many years, industrial manufacturers and researchers have aimed to
  formally specify and verify computing platform.
  %
  Defining a comprehensive model in terms of hardware and software components
  remains challenging, yet such a model is a prerequisite to verify a computing
  platform in terms of architectural attacks.
  %
  We have proposed FreeSpec, a Coq framework built upon a formalism to modularly
  specify and verify each component independently, then verify their
  \emph{composition}.
\end{abstract}

% ----------------------------------------------------------------------------------------
% ACKNOWLEDGEMENTS
% ----------------------------------------------------------------------------------------

\begin{acknowledgements}
  \addchaptertocentry{\acknowledgementname} % Add the acknowledgements to the table of contents
  The acknowledgments and the people to thank go here, don't forget to include
  your project advisor\ldots
\end{acknowledgements}

% ----------------------------------------------------------------------------------------
% LIST OF CONTENTS/FIGURES/TABLES PAGES
% ----------------------------------------------------------------------------------------

\tableofcontents % Prints the main table of contents

\listoffigures % Prints the list of figures

\listoftables % Prints the list of tables

% ----------------------------------------------------------------------------------------
% ABBREVIATIONS
% ----------------------------------------------------------------------------------------

\begin{abbreviations}{ll} % Include a list of abbreviations (a table of two columns)

  \textbf{LAH} & \textbf{L}ist \textbf{A}bbreviations \textbf{H}ere\\
  \textbf{WSF} & \textbf{W}hat (it) \textbf{S}tands \textbf{F}or\\

\end{abbreviations}

% ----------------------------------------------------------------------------------------
% DEDICATION
% ----------------------------------------------------------------------------------------

\dedicatory{For/Dedicated to/To my\ldots}

% ----------------------------------------------------------------------------------------
% THESIS CONTENT - CHAPTERS
% ----------------------------------------------------------------------------------------

\mainmatter % Begin numeric (1,2,3...) page numbering

\pagestyle{thesis} % Return the page headers back to the "thesis" style

% Include the chapters of the thesis as separate files from the Chapters folder
% Uncomment the lines as you write the chapters

%!TEX root = ../main-mini.tex
\chapter{Introduction}
\label{chapter:introduction}

\endquote{``\emph{All problems in computer science can be solved by another
    level of indirection.}''

  \hfill\footnotesize --- David Wheeler}

\vspace{1cm}\noindent
%
\TODO{Attention, software, hardware et firmware sont normalement indénombrable
  en anglais. Il faut que tu identifies les manières idiomatique d'utiliser ces
  termes.}
%
To manage complexity, computing platforms are commonly built as successions of
abstraction layers, from the hardware components to high-level software
applications.
%
Each layer leverages the interface of its predecessor to expose a higher-level,
more constrained set of functionalities for its successors.
%
This enables separation of concerns ---each layer encapsulates one dimension of
the overall complexity--- and modularity ---two layers which expose the same
interface can be seamlessly interchanged.
%
The typical computing platform can be broken down to the following layers:
%
\begin{description}
\item [Hardware Architecture]
  %
  A computer consist is made of several inter\-connected, physical hardware
  components.
  %
  Hardware components take many forms and serve as different purposes, from the
  \ac{cpu}, which executes the pieces of software which form the upper layers of
  abstraction, to the various input/output devices, which allow a user to
  interact with the computing platform.
  %
\item [Firmware/BIOS]
  %
  Hardware components often require some firm\-ware, \emph{i.e.} a dedicated
  piece of software, to operate.
  %
  Hardware manufacturers tend to embed some features in their firmware for
  several reasons.
  %
  For starters updating software is easier than updating hardware, as the latter
  often requires a recall.
  %
  Vendors can also implement additional high-level, business focus features in
  their firmware to differentiate from competitors.
  %
  For instance, Intel has started to provide out-of-band management features to
  its processors in 2008\,\cite{ruan2014me}.
  %
  In addition, some firmware is responsible for the initialization of the
  platform as a whole.
  %
  Afterwards, when we refer to this specific type of firmware, we use the term
  \ac{bios}.
  %
  \TODO{Dire qu'aujourd'hui la plupart des BIOS respecte le standard UEFI.}
  %
\item [System Software]
  %
  The computing environment which is provided by the hardware architecture alone
  is very generic.
  %
  As a consequence, it requires a non-trivial amount of configuration.
  %
  System software leverages this environment to provide high-level interfaces
  (\emph{e.g.} system calls), so that applications can focus on business
  logic. Thus, applications often do not care about which particular hardware
  architecture they are executed on.
  %
  In addition, system software deal with the complexity implied by the
  concurrent execution of several applications.
  %
\item [Applications]
  %
  Applications leverages the high-level interface exposed by system software in
  order to provide a service to the user of the computing platform.
  %
  The previous layers only exist so that applications can be executed.
\end{description}

Variations of these abstraction layers exist.
%
For instance, for embedded devices, it is common that the system software and
the applications form one single layer.
%
On the contrary, for mainstream computers, the system software can be formed by
an operating system only, or by the combination of one hypervisor and several
operating systems.
%
The latter case is common in the context of cloud computing, as modern
hypervisors (\emph{e.g.} Xen, Hyper-V) bring powerful features such as virtual
machines snapshots or migrations. \TODO{ajouter des références. De quels
  hyperviseurs veux tu parler?  c'est un peu flou}
%
It is also frequent that application internals are themselves built as
abstraction layers (\emph{e.g.} applicative framework, core logic, plugins,
etc.).

\TODO{Rajouter une jolie figure pour illustrer cet empilement de couches et y
  faire référence dans le texte}

\section{Hardware-based Security Enforcement mechanisms}

From a security perspective, each layer is often more privileged than its
successors.
%
For instance, system software manages the life cycle of the applications, by
actively tampering with their execution.

As a consequence, each layer implicitly trusts its predecessors.
%
On the one hand, it is important to keep this fact in mind when we consider the
security of the computing platform.
%
Trust in lowest levels of abstraction can be misplaced: hardware implants and
backdoors are no fantasy\,\cite{yang2016a2}, the \ac{bios} has been used as an
attack vector to tamper with its execution even after a machine
restart\,\cite{embleton2013smm}, etc.
%
On the other hand, one layer may constrain the execution of its successors
execution, with respect to a targeted security policy.
%
For instance, an operating system shares hardware resources among several
applications, and therefore is able to enforce availability ---fair share of
\ac{cpu} time---, confidentiality and integrity ---exclusive partition of
physical memories--- properties.

Constraining an execution can be achieved in various ways.
%
Concerning the lowest layers of a software stack, the common approach is to rely
on features provided by the hardware architecture.
%
In a nutshell, the main idea is to reduce the hardware capabilities one software
can leverage while it is executed.
%
For instance, system software often leverages among other mechanisms a \ac{mmu}
to partition the system memory.
%
Thus, when an application is executed, it can only access a subset of the system
memory.
%
In addition, the \ac{cpu} can leverage a hardware timer to stop applications'
execution, without the need for these applications to cooperate.

\begin{definition}[Hardware-based Security Enforcement]
  We call \ac{hse} the class of security enforcement mechanisms where a trusted
  software component configure the underlying hardware architecture to constrain
  the execution of untrusted software components with respect to a targeted
  security policy.
\end{definition}

A \ac{hse} mechanism enforces its targeted security property when
%
\begin{inparaenum}[(1)]
\item the trusted software components correctly configure the hardware features
  at their disposal, and
%
\item these features are sufficient to constrain the untrusted software
  execution as expected.
\end{inparaenum}
%
Both remain challenging.

\paragraph{Software Errors.}
%
Many facts can explain this state of things.
%
In the past, lower-level pieces of software, such as firmware components, may
not have been conceived and implemented with security as a primary focus.
%
The increasing complexity of hardware architectures can also be held partly
responsible.s
%
While one computer can be made of dozens of components, software developers have
to read and understand as many, independent and often large documents of various
forms (\emph{e.g.} data sheets, developer manuals), and they rarely focus on
security.
%
When software developers misunderstand the documentation, as it happened for
instance for the \texttt{MOV SS} and \texttt{POP SS} x86
instructions\,\cite{movsspopss}, the impact in terms of security can be
important.

\paragraph{Hardware Errors.}
%
Over the past decades, vendors have regularly added security features to their
products.
%
Intel, for instance, has notably introduced hardware-based virtualization (VT-x,
VT-d)\,\cite{intel2014manualvt}, dynamic root of trust
(TXT)\,\cite{intel2015txt}, or applicative enclaves
(SGX)\,\cite{intel2014manualsgx,costan2016sgxexplained}.
%
It is important to notice that most of them have been compromised due to
implementation bugs\,\cite{wojtczuk2011txtbug,sang2010iommu}.
%
This is not surprising, as novel hardware features tend to be more and more
complex.
%
Hence, using advanced, novel hardware features, in a security-sensitive context,
may be counterproductive.
%
In addition to these implementation errors, the fact that hardware architecture
often comprises hundreds of features implemented by dozens of interconnected
devices complicate the conception of new hardware features.
%
Indeed, the latter should not interfere with the security properties enforced by
the features which were present before.
%
For instance, the Flash Memory (where lives the \ac{bios} code) is supposedly
protected against arbitrary write accesses from system software.
%
This protection relies on a particular hardware interrupt.
%
This did not prevent Intel to introduce TXT, a novel security mechanism which
had the particular side effect of disabling this hardware interrupt.

\begin{definition}[Architectural Attack]
  We call Architectural Attacks the class of security vulnerabilities, where
  each component is working as expected, yet their composition creates an attack
  path untrusted software components can leverage to defeat a \ac{hse} mechanism
  implemented to constrain its execution.
\end{definition}

Architectural attacks result in a flaw in the specifications of the computing
platform.
%
As such, they precede implementation errors, and their countermeasures often
require a change in the hardware interface.
%
To prevent them, it is mandatory to reason about the computing platform as a
whole.

\section{Formal Verification of HSE mechanisms}
\label{sec:intro:verif}

The important impact of previously disclosed architectural attacks have
motivated our will to formally specify HSE mechanisms.
%
We believe this would benefit both hardware designers and firmware and system
software developers.
%
Firstly, a formal specification of HSE mechanisms can be leveraged as a
foundation for a systemic approach to verify hardware specifications.
%
For each novel hardware feature introduced, it would be necessary to check that
the previous proofs hold, meaning this feature does not introduce an
architectural attack.
%
Secondly, it provides unambiguous specifications to software developers.
%
We believe these specifications can be a valuable addition to the existing
documentation, because they gather at one place information that is normally
scattered across many documents.

To specify a HSE mechanism, we have to model the hardware architecture, and the
scale of the task represents an important challenge.
%
Because architectural attacks often come from unsuspected places, it is
necessary to consider the whole computing platform; at the same time, many
components are individually complex.
%
In addition, new hardware components are frequently released, and the same
hardware component can be found in different hardware architecture to play a
similar role.
%
As a consequence, the more modular our models and proofs are, the more
practicable our approach becomes.

\paragraph{}
%
While we have tried to propose generic formalisms which can potentially be
applied to large ranges of problems, we have adopted the systemic approach to
apply them to the x86 hardware architecture.
%
Because of its predominant position on the personal computer market, the x86
hardware architecture has been intensively studied.
%
Our decision to explore this research field was based on our experience with
several architectural attacks which have affected x86-based computing platform
x86 hardware architecture.
%
We have implemented our proofs of concepts in the Coq proof assistant.
%
Our choice is motivated by the following reasons:
%
\begin{itemize}
\item {\scshape Gallina}, the specification of Coq, is an expressive formal language,
  that combines both a higher-order logic and à powerful dependent type system,
  while {\scshape Ltac}, the proof tactic language of Coq, enables powerful
  automation features.
  %
  This suits well for specifying security properties to be enforced by HSE
  mechanisms, as they can take many forms.
\item The extraction mechanism of Coq allows for transpiling formal
  specifications written in {\scshape Gallina} to OCaml, meaning we can write
  \emph{executable specifications}.
  %
  In theory, we can leverage this feature to validate a model against a real
  implementation.
  %
  Because we focus on products which already exist, rather to intervene prior to
  their implementation, this possibility is worth taking into account.
\end{itemize}

\section{Notations}

As far as possible, we have tried to remain consistent while writing this
thesis.

\paragraph{Conventions}
%
We often define sets of values, and interfaces in particular, in terms of
functions to construct these values.
%
These functions are called ``constructors,'' and they have mutually exclusive
images, i.e. it is not possible to construct the same value with two different
constructors.
%
Functions are written in bold.
%
In addition, constructors begin with a capital letter.

\paragraph{Named Tuples}
%
We adopt a notation similar to Haskell record types to manipulate ``named''
tuples, that is tuples where each component is a field identified by a name.

\[
  \begin{array}{rccll}
    A & \triangleq & \{ & \func{field_1}: & T_1 \\
      &            & ;  & \func{field_2}: & T_2 \\
      &            & \} &
  \end{array}
\]

For $a \in A$, we write $a.\func{field_1}$ for selecting the value of the
\func{field_1} field. We write $a \{ \func{field_1} \leftarrow t \}$ for
updating the value of the field \func{field_1}. As a consequence,
%
\[
  a \{ \func{field_1} \leftarrow t \}.\func{field_1} = t
\]

\section{Contributions and Outline}

Our works fall within a context where the security of lower levels of
abstraction grows in importance.
%
In Chapter~\ref{chapter:usecase}, we describe in more detail the x86 hardware
architecture, and a selection of representative architectural attacks which have
been disclosed over the past decade.
%
In Chapter~\ref{chapter:relatedwork}, we revisit existing works and approaches
in line with the two challenges we tackle in this thesis, that is:
%
\begin{inparaenum}[(1)]
\item formally specifying and verifying \ac{hse} mechanisms against a hardware
  model, and
%
\item defining hardware models which can remain exploitable in presence of large
  architecture hardware.
\end{inparaenum}

In Chapter~\ref{chapter:speccert}, we present a generic formalism to specify and
verify against a hardware model HSE mechanism whose purpose is to isolate
trusted software components execution from the rest of the software stack.
%
We have implemented SpecCert, a framework for the Coq proof assistant based on
this formalism. In addition, we have implemented {\scshape Minx86}, a minimal
x86 model, and we have verified the HSE mechanism implemented by x86 BIOSes.
%
This work has been presented at the $21^{th}$ International Symposium on Formal
Methods (FM2016)\,\cite{letan2016speccert}.

In Chapter~\ref{chapter:freespec}, we present a systemic approach to specify and
verify large systems, organized as trees of components, in a modular way.
%
We apply ideas from verification of computer programs with side effects to the
domain of component-based modelling, and the resulting approach can be used both
for verifying the composition of hardware components and software components
alike.
%
In addition, we have developed FreeSpec, a framework for the Coq proof
assistant.
%
FreeSpec provides the necessary definitions and theorems to specify components,
and automation features to guide their verification.
%
This work has been presented at the $22^{th}$ International Symposium on Formal
Methods (FM2018)\,\cite{letan2018freespec}.

Finally, we conclude this thesis in Chapter~\ref{chapter:conclusion}, where we
suggest some possible directions for future works.


\chapter{Hardware-based Security Enforcement}

\section{Hardware Architecture Principles}

\section{Software Stacks and Software Isolation}

\section{Security Challenges}

\section{Architectural Attacks}

\chapter{Related Works}

\section{Industrial Efforts}

\begin{itemize}
\item Intel
\item ARM
\end{itemize}

\section{Modelling Existing Architecture}

\section{Secure by Construction Architecture}

\section{ISA Modelling}

\chapter{SpecCert}

Modern hardware architectures have grown in complexity.
%
They now are made of numerous devices which expose multiple programmable
functions.
%
In this article, we identify a class of security enforcement mechanisms we call
Hardware-based Security Enforcement (HSE) such that a set of software components
configures the hardware in a way which prevents the other software components to
break a security policy.
%
For instance, when an operating system uses the ring levels and memory paging
features of x86 microprocessors to isolate the userland applications, it
implements a HSE mechanism.
%
A HSE mechanism is sound when it succeeds in enforcing a security policy.
%
It requires (1) the hardware functions to provide the expected properties and
(2) the software components to make a correct use of these hardware functions.
%
In practice, both requirements are hard to meet.

First, hardware architectures comprise multiple interconnected devices which
interact together.
%
From a security perspective, it implies considering the devices both
individually and as a whole.
%
Hardware functions are not immune to security vulnerabilities.
%
For instance, early versions of the \texttt{sinit} instruction implementation of
the Intel TXT technology\,\cite{intel2015txt} allowed an attacker to perform a
privilege escalation\,\cite{wojtczuk2011txtbug}.
%
The legitimate use of a hardware mechanism can also break the security promised
by another.
%
For instance, until 2008, the x86 cache allowed to circumvent an access control
mechanism exposed by the memory
controller\,\cite{wojtczuk2009smram,duflot2009smram}.
%
Secondly, hardware architectures have grown in complexity and, as a consequence,
HSE mechanisms too.
%
There are many examples of security vulnerabilities which are the consequence of
an incorrect HSE mechanism
implementation\,\cite{kallenberg2014failure,bulygin2014bios,intel2014chipsec}.

In this paper, we introduce SpecCert, a framework for specifying and verifying
HSE mechanisms against hardware architecture models.
%
SpecCert relies on a three-steps methodology.
%
First, we model the hardware architecture specifications.
%
Then we specify the software requirements that must be satisfied by the trusted
software components which implement the HSE mechanism.
%
Finally, we prove that the HSE mechanism is sound under the assumption that the
software components complies to the specified requirements.
%
This implies the hardware involved in the HSE mechanism indeed provides the
security properties they promise.
%
We believe this approach to be beneficial to both hardware designers and
software developers.
%
The former can verify their hardware mechanism assumptions and the latter can
get a formal specification to implement the HSE mechanism.

In Section \ref{sec:speccert:framework}, we give a formal definition of the
SpecCert formalism.
%
In Section \ref{sec:speccert:hardware}, we define a model of x86-based hardware
architectures to verify HSE mechanisms targeting software isolation policies
using publicly available Intel specifications.
%
In Section \ref{sec:speccert:smm}, we verify the soundness of the HSE mechanism
implemented in many x86 computer firmware codes to isolate the code executed
while the CPU is in System Management Mode (SMM), a highly privileged execution
mode of x86 microprocessors.
%
Our model and proofs have been implemented using Coq, a proof assistant system
and have been released as an open source software\,\footnote{Which can can be
  found at: \url{https://github.com/lethom/speccert}}.
%
We discuss our results in Section \ref{sec:speccert:discuss}.

\section{SpecCert Formalism} \label{sec:speccert:framework}

In SpecCert, we model the hardware architecture and its features with a set of
states $\set{H}$, a set of events $\set{E}$ and a Computing Platform $\Sigma$
which defines a semantics of events as state-transformers. Hence, the execution
of a set of software components by a hardware architecture is a sequence of
state-transformations (denoted $\transitionLTS{Ez}{h}{ev}{h'}$) in this model.
In this paper, we consider exclusively Execution Monitoring (EM) enforceable
security policies\,\cite{schneider2000enforceable,basin2013enforceable} that are
security policies which can be enforced by monitoring the software execution. As
a consequence, we model a security policy with a predicate $P$ on sequences of
state-transformations.  Finally, we model a HSE mechanism $\Delta$ with a set of
requirements on states to characterize safe hardware configurations and a set of
requirements on state-transformations for trusted software components to
preserve the state requirements through software execution. A HSE mechanism is
sound when every sequence of state transformations which satisfies these
requirements also satisfies the security policy predicate.

\subsection{Computing Platforms} \label{subsec:speccert:computing}

We now dive more deeply into the SpecCert formalism and give a formal definition
of the Computing Platform. We model a hardware architecture which executes
several software components using states, events and a semantics of events as
states-transformers.

The state of a hardware architecture models the configuration of its devices at
a given time. This configuration may change over time with respect to the
hardware specifications and comprises any relevant data such as registers
values, inner memory contents, etc. A hardware architecture state update is
triggered by some events. We distinguish two classes of events: the software
events which are direct and forseeable side-effects of the execution of an
instruction and the hardware events which are not. The execution of an
instruction can be broken down into a sequence of software events.

For instance, to execute the x86 instruction\,\footnote{Written in AT\&T syntax
  here.} \texttt{mov (\%ecx),\%eax}, a x86 CPU:
\begin{compactitem}
\item reads the content of the register \texttt{ecx} as an address
\item reads the main memory at this address
\item writes this content into the register \text{eax}
\item updates the register \texttt{eip} with the address of the next instruction
  to execute
\end{compactitem}

We model this sequence of actions as four software events which trigger four
state updates. Note that if the content of the \texttt{ecx} register is not a
valid address, the scenario is different. In such a case, the read access to the
main memory fails and an interrupt is raised. This second scenario is modeled
with another sequence of events which involved a hardware event \emph{i.e.} the
interrupt.

The semantics of events as state-transformers is specified using preconditions
and postconditions. Preconditions specify the state requirements which are
necessary for an event to be observed. Postconditions specify the consequences
of an event on the hardware architecture state.

\begin{definition}[Computing System]
  Given $\set{H}$ a set of hardware architecture states and $\set{E}$ a set of
  events, a Computing Platform $\shortLTS{Ez}$ is a pair of
  $(pre\-condition, post\-condition)$ where $pre\-condition$ is a predicate on
  $\set{H} \times \set{E}$ and $post\-condition$ is a predicate on
  $\set{H} \times \set{E} \times \set{H}$. $\shortLTS{Ez}$ defines a semantics
  of events as state-transformers such as
  \begin{prooftree}
    \AxiomC{$precondition(h,ev)$} \AxiomC{$postcondition(h,ev,h')$}
    \BinaryInfC{\transitionLTS{Ez}{h}{ev}{h'}}
  \end{prooftree}

  \transitionLTS{Ez}{h}{ev}{h'} is called a state-transformation of\,
  $\shortLTS{Ez}$.
\end{definition}

% \begin{definition}[Software Stack]
%   Given $\set{S}$ a set of Software Components, $\set{H}$ a Hardware
%   Architecture and $\set{P}$ the Processor Unit of $\set{H}$,
%   $\zeta : \set{P} \rightarrow \set{S}$ is a Software Stack which dedicates
%   states of $\set{P}$ to each Software Component of $\set{S}$.
% \end{definition}

\subsection{Security Policies}

Given $\set{H}$ a set of states of a hardware architecture, $\set{E}$ a set of
events, $\shortLTS{Ez}$ a Computing Platform and $\set{S}$ a set of software
components being executed by the hardware architecture, a particular execution
of a set of software components is modeled with a sequence of
state-transformations we call a run of $\shortLTS{Ez}$.

\begin{definition}[Run]
  A run of the Computing Platform $\shortLTS{Ez}$ is a sequence of
  state-transformations of\, $\shortLTS{Ez}$ such that for two consecutive
  transformations, the resulting state of the first is the initial state of the
  next. We denote $\pathesLTS{Ez}$ the set of runs of the Computing Platform
  $\shortLTS{Ez}$ and $init(\rho)$ the initial state of a run $\rho$.
\end{definition}

We consider EM-enforceable security
policies\,\cite{schneider2000enforceable,schneider2} specified with predicates
on runs. A run is said to be secure according to a security policy when it
satisfies the predicate specifying this policy.

In this paper, we focus on a class of security policies we call software
execution isolation policies. Such a policy prevents a set of untrusted software
components to tamper with the execution of another set of so-called trusted
software components. We consider that a software component tampers with the
execution of another when it is able to make the latter execute an instruction
of its choice.

In practice, a subset of states of the hardware architecture is dedicated to
each software component. For instance, the x86 CPU has a feature called
protection rings where each ring can be seen as an execution mode dedicated to a
software component. Hence, the ring 0 is dedicated to the operating system
whereas the userland applications are executed when the CPU is in ring 3. In
SpecCert, we take advantage of this CPU state sharing to infer which software
component is currently executed from a hardware architecture state. For the
following definitions, we assume the hardware architecture contains only one
CPU.

\begin{definition}[Hardware-Software Mapping]
  \label{def:hardsoftmap}
  A hardware-software mapping $con\-text : \set{H} \rightarrow \set{S}$ is a
  function which takes a hardware state and returns the software component
  currently executed.
\end{definition}

Dealing with multi-core architectures would require additional efforts and
notations. One possible solution could be to define an identifier per core and
to use this identifier in addition to the current hardware state to deduce the
software component currently executed by the corresponding core. However, this
is out of the scope of this article.

% okay so know we need to explain the event-software mapping. it is not that
% hard, but the issue is that it is more or less related to the memory
% ownership, so maybe i need to introduced it here.
We now introduce the concept of \textit{memory location ownership}. A memory
location within a hardware architecture is a container which is able to store
data used by a software component \emph{e.g.}\,a general-purpose register of a
CPU, a DRAM memory cell, etc. We say that a Computing Platform tracks the memory
location ownership if the hardware architecture states maps each memory location
with a software component called its \emph{owner}, and the Computing Platform
semantics updates this mapping through state-transformations. A software
component becomes the new owner of a memory location when it overrides its
content during a state-transformation. By extension, we say a software component
owns some data when it owns the memory location in which these data are stored.

With this mapping, it becomes possible to determine the owner of an instruction
fetched by the CPU in order to be decoded and executed.

\begin{definition}[Event-Software Mapping]
  \label{def:evsoft}
  An event-software mapping
  $fet\-ched: \set{H} \times \set{E} \rightarrow \mathcal{P}(\set{S})$ is a
  function which takes an initial hardware state and an event and returns the
  set of the fetched instructions owners during this state-transformation.
\end{definition}

Hence, $s \in fetched(h, ev)$ means that an instruction owned by $s$ was fetched
during a state-transformation triggered by an event $ev$ from a state $h$. With
a hardware-software mapping and an event-software mapping, we give a formal
definition of a \textit{software execution tampering}.

\begin{definition}[Software Execution Tampering]
  \label{def:codeinjection}
  Given $h$ the initial state of a state-transformation triggered by an event
  $ev$, $context$ a hardware-software mapping, $fetched$ an event-software
  mapping and $x, y \in \set{S}$ two software components, the software component
  $y$ tampers with the execution of another software component $x$ if the CPU
  fetches an instruction owned by $y$ in a state dedicated to $x$.
  \[ \begin{array}{l} software\_tampering(context, fetched, h, ev, x, y)
       \triangleq \\
       \qquad\qquad\qquad\qquad context(h) = x\,\wedge\,y \in fetched(h,ev)
     \end{array}
   \]
 \end{definition}

 Given $\set{T} \subseteq \set{S}$ a set of trusted software components, the
 software execution isolation policy prevents the untrusted components from
 tampering with the execution of the trusted components. Such a policy is
 enforced during a run if no untrusted component is able to tamper with the
 execution of a trusted component.

 \begin{definition}[Software Execution Isolation]
   \label{def:softwareisolation}
   Given $context$ a hardware-software mapping, $fetched$ an event-software
   mapping and $\rho$ a run of $\shortLTS{Ez}$,
   \[ \begin{array}{l}
        software\_execution\_isolation(context, fetched, \rho, \set{T}) \triangleq \\
        \qquad\forall \transitionLTS{Ez}{h}{ev}{h'} \in \rho, \forall t \in
        \set{T}, \forall
        u \not\in \set{T}, \\
        \qquad\qquad \neg software\_tampering(context, fetched, h, ev, t, u)
      \end{array} \]
  \end{definition}

  In this definition, $t$ is a trusted software component and $u$ is an
  untrusted ---~potentially malicious or hijacked~--- one.

  \subsection{Hardware-based Security Enforcement Mechanism}

  A HSE mechanism is a set of requirements on states to characterize safe
  hardware configurations and a set of requirements on state-transformations to
  preserve the state requirements through software execution. The software
  components which implement a HSE mechanism form the Trusted Computing Base
  (TCB).

  \begin{definition}[HSE Mechanism]
    \label{def:hse}
    Given $\set{H} $ a set of states of a hardware architecture, $\set{E}$ a set
    of events and $\shortLTS{Ez}$ a Computing Platform, we model a HSE mechanism
    $\Delta$ with a tuple $(\fun{inv}, \fun{behavior}, \set{T}, context)$ such
    as
    \begin{compactitem}
    \item $inv$ is a predicate on $\set{H}$ to distinguish between safe hardware
      configurations and potentially vulnerable ones
    \item $behavior$ is a predicate on $\set{H} \times \set{E}_{Soft}$ to
      distinguish between safe software state-transformations and potentially
      harmful ones
    \item $\set{T} \subseteq{S}$ is the set of software components which form
      the TCB of the HSE mechanism
    \item $context$ is a hardware-software mapping to determine when the TCB is
      executed
    \end{compactitem}
  \end{definition}

  For instance, in x86-based hardware architectures, the SPI Flash contents (the
  code and configuration of the firmware) is protected as follows:

  \begin{compactenum}
  \item By default, the SPI Flash is locked and its content cannot be overriden
    until it has been unlocked
  \item Some software components can unlock the SPI Flash
  \item When they do so, the CPU is forced to start the execution of a
    special-purpose software component
  \item This software component has to lock the SPI Flash before the end of its
    execution
  \end{compactenum}
  In this example, the special-purpose software component is the TCB. A safe
  hardware state (modeled with $inv$) is either a state wherein the
  special-purpose software component is executed or a state wherein the SPI
  Flash is locked. This requirement on hardware architecture states is preserved
  by preventing the special-purpose software component to end its execution
  before it has locked the SPI Flash (modeled with $behavior$).

  For a HSE mechanism to be correctly defined, it must obey a few axioms,
  together called the HSE Laws. The first law says that the state requirements
  specified by $inv$ are preserved through state-transformations if the software
  transformations which do not satisfy $behavior$ are discarded. The second law
  says that the $behavior$ predicate specifies state-transformations
  restrictions for the TCB only. The software components which are not part of
  the TCB are considered untrusted and we make no assumption on their behavior.

  \begin{definition}[HSE Laws]
    \label{def:laws}
    A HSE mechanism $\Delta = (inv, behavior, \set{T}, context)$ has to satisfy
    the following properties:
    \begin{compactenum}
    \item $behavior$ preserves $inv$: $\forall \transitionLTS{Ez}{h}{ev}{h'}$,
      \[ \begin{array}{l} inv(h) \Rightarrow (ev \in \set{E}_{Soft} \Rightarrow
          behavior(h,ev)) \Rightarrow inv(h')
         \end{array}
       \]
     \item $behavior$ only restricts the TCB:
       $\forall x \not\in \set{T}, \forall h \in \set{H}, \forall ev \in
       \set{E}_{Soft}$,
       \[
         \begin{array}{l}
           context(h) = x \Rightarrow behavior(h, ev)
         \end{array}
       \]
     \end{compactenum}
   \end{definition}

   A run complies to a HSE mechanism definition if its initial state satisfies
   the state requirements and each state-transformation of the run satisfies the
   state-transformations requirements. The set of the runs which comply with
   $\Delta$ is denoted by $\mathcal{C}(\Delta)$.

\begin{definition}[Compliant Runs]
  Given $\rho \in \pathesLTS{Ez}$,
  \[ \rho \in \mathcal{C}(\Delta) \triangleq inv(init(\rho))\,\wedge\,\forall
    \transitionLTS{Ez}{h}{ev}{h'}, ev \in \set{E}_{Soft} \Rightarrow
    behavior(h,ev) \]
\end{definition}

Eventually, we aim to prove that a HSE mechanism is sound \mbox{---it} succeeds
to enforce a security policy--- under the assumption that software components of
the TCB always behave according to the specification given in the HSE mechanism
definition.

\begin{definition}[Sound HSE Mechanism]
  \label{def:sound}
  A HSE mechanism $\Delta$ succeeds in enforcing a security policy $P$ when each
  compliant run of $\Delta$ is secure. In such a case, $\Delta$ is said to be
  sound.
  \[ sound(\Delta, P) \triangleq \forall \rho \in \set{C}(\Delta), P(\rho)
  \]
\end{definition}

\begin{table}[h]
  \begin{tabular}{clc}
    \hline
    \bf Notation  & \multicolumn{1}{c}{\bf Description} \\
    \hline
    $\set{S}$ & Set of software components \\
    \hline
    $\set{H}$ & Set of states of the hardware architecture \\
    \hline
    $\set{E}$ & Set of states of events of the hardware architecture \\
    \hline
    $\shortLTS{Ez}$ & Semantic of events as state-transformers (Computing
                      Platform) \\
    \hline
    $\transitionLTS{Ez}{h}{ev}{h'}$ & State-transformation according to
                                      $\shortLTS{Ez}$ \\
    \hline
    $\pathesLTS{Ez}$ & Set of sequences of $\shortLTS{Ez}$ state-transformations
                       (Runs) \\
    \hline
    $P$ & Predicate on run to model a EM-enforceable security policy \\
    \hline
    $\Delta$ & Requirements on states and state-transformation (HSE mechanism)
    \\
    \hline
  \end{tabular}
  \caption{SpecCert CheatSheet}
\end{table}

\section{\textsc{MinX86}} \label{sec:speccert:hardware}

The SpecCert formalism is the foundation of the SpecCert framework. It comprises
a set of high-level definitions to specify a HSE mechanism against a hardware
architecture model. In its current state, the SpecCert framework contains a
model of x86 called $\formatLTS{Minx86}$. $\formatLTS{Minx86}$ is intended to be
a minimal model for single core x86-based machines and we have used publicly
available Intel documents\,\cite{intel2013celeron,intel2009mch,intel2014manual}
to define it.

\subsection{Model Scope}

The hardware architecture we are modeling with $\formatLTS{Minx86}$ contains a
CPU, a cache, a memory controller, a DRAM controller and a VGA
controller\,\footnote{A VGA controller is a hardware device which on we can
  connect a screen. It exposes some memory to the CPU for communication
  purposes.}  which both expose some memory to the CPU.

$\formatLTS{Minx86}$ is meant to be a proof of concept of the SpecCert formalism
and thus is not exhaustive. In its current state of implementation, its scope
focuses on the System Management Mode (SMM) feature of x86 microprocessors.

\paragraph{Hardware Specifications}
We consider the CPU can be either in System Management Mode (SMM) or in an
unprivileged mode. The SMM is "a special-purpose operating mode provided for
handling system-wide functions like power management, system hardware control,
or proprietary OEM-designed code"\,\cite{intel2014manual}. It is the most
privileged execution mode of x86 processors.  When a CPU receives a special
hardware interrupt called System Management Interrupt (SMI), it halts its
current execution and reconfigures itself to a specified state from which it
executes the code stored in memory at the address $SMBASE + \texttt{0x8000}$. In
practice, the SMBASE value points to the base of a memory region called the
SMRAM. Leaving the SMM is done by executing a special purpose instruction called
\texttt{rsm} (for \emph{resume}).

The CPU relies on a cache to reduce the Input/Output (\IO, that is a read or
write access to the memory) latency. We model one level of cache which stores
both data and instructions and we consider two cache strategies: uncacheable
(UC) and writeback (WB). With the UC cache strategy, the cache is not used and
all \IOs are forwarded to the memory controller, whereas with the WB strategy,
the cache is used as much as possible\,\footnote{These cache strategies are
  explained in \cite{intel2014manual}, Volume 3A, Chapter 11, Section 11.3 (page
  2316 -- 2317)}. To determine which cache strategy to use, the CPU relies on
several configuration registers and mechanisms. One of them is a pair of
registers called the System Management Range Registers (SMRR) which can only be
configured when the CPU is in SMM. They are used to tell the CPU where the SMRAM
is and which cache strategy to use for \IO targeting the SMRAM when the CPU is
in SMM.  When it is not in SMM, the CPU always uses the UC strategy for \IO
targeting the SMRAM. SMRR have been introduced as a countermeasure of the SMRAM
cache poisoning attack\,\cite{wojtczuk2009smram,duflot2009smram} which allowed
an untrusted code to tamper with the copy of the SMRAM stored in the cache.  The
memory controller\,\cite{intel2009mch} receives all the CPU \IOs which are not
handled by the cache and dispatches them to the DRAM controller or to the VGA
controller. It exposes a unified view (the memory map) of the system memory to
the CPU. The CPU manipulates this memory map with a set of addresses called the
physical addresses. The memory controller dedicates a special range of physical
addresses to form the SMRAM. The SMRAM is dedicated to store the code intended
to be executed when the CPU is in SMM.

\paragraph{Tracking the Memory Ownership} The \formatLTS{Minx86} definition is
parameterized with an hardware-software mapping (see
Definition~\ref{def:hardsoftmap}). The memory locations of \formatLTS{Minx86}
Computing Platforms are either cache lines or memory cells exposed by the DRAM
controller or the VGA controller. The memory ownership is updated through
state-transformations according to three rules:
\begin{compactenum}
\item When a cache line gets a copy of a DRAM or VGA cell content, the owner of
  this cell becomes the new owner of this cache line.
\item When the content of this cache line is written back to a memory cell, the
  new owner of this memory cell is the owner of this cache line.
\item When a state-transformation implies the content of a memory location to be
  overriden with a new value, the software currently executed becomes its new
  owner.
\end{compactenum}

Given $\set{S}$ a set of software components, the set of states of
\formatLTS{Minx86} Computing Platform hardware architecture is denoted by
$\texttt{Archi}_{\set{S}}$ and the set of \formatLTS{Minx86} Computing Platform
events is denoted by $\texttt{Event}$. Given $context$ a hardware-software
mapping, we denote the Computing Platform $\formatLTS{Minx86}$ parameterized
with $context$\,\footnote{The related definitions and explanations are given on
  page \pageref{page:minx86def}.} such that
\[ \formatLTS{Minx86}(context) \triangleq (minx86\_pre,
  minx86\_post(context)) \]

\subsection{Hardware Architecture State}

$\texttt{Archi}_{\set{S}}$ is defined as the Cartesian product of the set of
states of the CPU, the CPU's cache, the memory controller and the hardware
memories exposed by both the DRAM controller and the VGA controller. Each of
these sets is defined in order to model the hardware features we have previously
described. We define
$\texttt{PhysAddr} \triangleq \setdef{\val{pa}_i}{i \leq \val{max\_addr}}$ the
set of physical addresses the CPU uses to perform \IO. The maximal address
offset (denoted by $\val{max\_addr}$ here) is specific to the CPU and may vary
in time according to its addressing mode (real mode, long mode, etc.), therefore
we left its value as a parameter of our model.  An in-depth definition of
$\texttt{Archi}_{\set{S}}$ is given in appendix~\ref{app:archis}.
% TODO

We model the projection of the SMRAM in the memory map such that
$\texttt{pSmram} \triangleq \setdef{\val{pa}_i}{\val{smram\_base} \leq i \le
  \val{smram\_end}}$.  The values of $\val{smram\_base}$ and $\val{smram\_end}$
are specified in the memory controller specifications. It is the software
responsability to set the SMRR accordingly. We assume
$\val{smram\_end} - \val{smram\_base} > \val{0x8000}$. This way, when the SMBASE
contains the address of the beginning of the SMRAM, the SMM entry point (that is
$SMBASE + \val{0x8000}$) is in SMRAM.

The hardware architecture states are implemented in the
\emph{SpecCert.x86.Archi\-tecture} module (about 1\,500 lines of code).

\subsection{Events as State-Transformers}

The set of events which trigger the state-transformations is denoted by
$\texttt{Event}$. As we said in Section \ref{subsec:speccert:computing}, we
distinguish hardware events denoted by $\texttt{Event}_{Hard}$ and software
events denoted by $\texttt{Event}_{Soft}$.

\begin{table}
  \bigcentering
  \begin{tabular}{lp{3cm}p{6cm}}
    \hline
    \textbf{Event} & \textbf{Paramters} & \textbf{Description} \\
    \hline
    $Write$ & $pa \in \texttt{PhysAddr}$ & A CPU \IO to write at physical address
                                           $pa$ \\
    \hline
    $Read$ & $pa \in \texttt{PhysAddr}$ & A CPU \IO to read at physical address $pa \in
                                          \texttt{PhysAddr}$ \\
    \hline
    $SetCacheStrat$ & $pa \in \texttt{PhysAddr}$ \newline $strat \in
                      \setshortdef{\val{UC}, \val{WB}}$ & Change the cache strategy for $pa$ to
                                                          $strat$ ($\val{WB}$ means write-back and $\val{UC}$ means uncacheable) \\
    \hline
    $UpdateSmrr$ & $smrr \in \texttt{Smrr}$ & Update the SMRR content with the
                                              new value $smrr$ \\
    \hline
    $Rsm$ & \centering --- & The CPU leaves SMM \\
    \hline
    $OpenBitFlip$ & \centering --- & Flip the $d\_open$ bit \\
    \hline
    $LockSmramc$ & \centering --- & Set the $d\_lock$ bit \\
    \hline
    $NextInstruction\quad$ & $pa \in \texttt{PhysAddr}$ & The program counter register
                                                          of the CPU is set to $pa$ \\
    \hline
  \end{tabular}
  \caption{List of software events}
  \label{tab:softev}
\end{table}

Table\,\ref{tab:softev} lists the software events we consider in the
$\formatLTS{Minx86}$ Computing Platforms. We model the CPU \IOs with $Read(pa)$
and $Write(pa)$, the configuration of the memory controller with $OpenBitFlip$
and $LockSmramc$, the configuration of the cache strategy with
$SetCacheStrat(pa,strat)$, the configuration of the SMRR with $UpdateSmrr(smrr)$
the exit of the SMM with $Rsm$ and the update of the CPU program counter
register with $NextInstruction(pa)$.

\begin{table}
  \bigcentering
  \begin{tabular}{lp{9cm}}
    \hline
    \textbf{Event} & \textbf{Description} \\
    \hline
    $Fetch$ & A CPU \IO to fetch the instruction stored at the physical address
              contained in the program counter register \\
    \hline
    $ReceiveSmi\quad$ & A SMI is raised and the CPU handles it \\
    \hline
  \end{tabular}
  \caption{List of hardware events}
  \label{tab:hardev}
\end{table}

The other causes of state-transformations are modeled using hardware events.
Table \ref{tab:hardev} lists the hardware events we consider in the
$\formatLTS{Minx86}$ Computing Platforms. $Fetch$ models the \IO to fetch the
instruction pointed by the program counter register. $ReceiveSmi$ models a
System Management Interrupt being risen and handled by the CPU.

We define $minx86\_fetched$ an event-software mapping for $\formatLTS{Minx86}$
Computing Platforms (see Definition \ref{def:evsoft}). The $minx86\_fetched$
function maps a state-transformation to the set of software components which own
an instruction fetched during this state-transformation. In the case of
\formatLTS{Minx86}, there is only one event which implies fetching instructions:
$Fetch$. Let $o$ be the owner of the instruction pointed by the program counter
register in the formula
\[ minx86\_fetched(h, ev) \triangleq \begin{cases}
    \setshortdef{o} & \text{if }ev = Fetch \\
    \emptyset & \text{otherwise} \\
  \end{cases} \] We can determine $o$ because $\formatLTS{Minx86}$ tracks the
memory location ownership.

\label{page:minx86def} Given $context$ a hardware-software mapping (see
Definition \ref{def:hardsoftmap}), the precondition predicate of the Computing
Platform $\formatLTS{Minx86}(context)$ is named $min\-x86\_pre$ and the
postcondition predicate is named $min\-x86\_post(con\-text)$. We give an
informal description of the $min\-x86\_pre$ and $minx\-86\_post(context)$ for
each event. These definitions have been implemented in Coq in the module
\emph{Spec\-Cert.x86.Transi\-tion}.

We first give the semantics of software events as state-transformers. A software
component can always read and write at any physical address. As a consequence,
the precondition for $Read(pa)$ and $Write(pa)$ always holds true. The
postcondition for $Read(pa)$ and $Write(pa)$ requires the memory ownership to be
updated according to the memories and cache state updates. The memory controller
enforces a simple access control to protect the SMRAM content in the DRAM memory
by forwarding the related \IO to the VGA controller when the CPU is not in SMM.
To determine the owner of the memory location which sees its content overriden
during a state transformation, the postcondition uses the hardware-software
mapping used to define the Computing Platform.

A software component can always update the cache strategy used for an \IO. The
postcondition for $SetCacheStrat(pa,strat)$ requires only the cache strategy
setting for this physical address $pa$ to change. The precondition for
$UpdateSmrr$ requires the CPU to be in SMM. The postcondition requires the SMRR
of the CPU to be updated with the correct value, the rest of the hardware
architecture state being left unchanged.

A software component can jump to any physical address, hence the postcondition
for $NextInstru\-ction(pa)$ always holds true. The postcondition for
$NextInstru\-ction(pa)$ requires the program counter register to be updated with
$pa$. The $OpenBitFlip$ precondition requires the SMRAMC register to be
unlocked. The postcondition requires the \texttt{d\_open} bit to be updated. The
$Lock\-Smramc$ precondition requires the $\texttt{d\_lock}$ bit to be unset. The
postcondition requires the $\texttt{d\_open}$ bit to be unset and the $d\_lock$
bit to be unset.

We now describe the semantics of hardware events as state-transformers. $Fetch$
models the fetching of an instruction by the CPU. As a consequence, the
definition of its precondition and postcondition are the same as $Read(pa)$ with
$pa$ being the program register value.  $ReceiveSmi$ precondition requires the
CPU not to be in SMM because SMM is non-reentrant.  The postcondition of
$ReceiveSmi$ requires the program counter to be set with the
$smbase + \val{0x8000}$ (where $smbase$ is the value of the SMBASE register of
the CPU) and the CPU is in SMM.

\section{System Management Mode HSE} \label{sec:speccert:smm}

In \cite{intel2014manual}, Intel states "the main benefit of SMM is that it
offers a distinct and easily isolated processor environment that operates
transparently to the operating system or executive and software
applications". For the SMM processor environment to be isolated, the code
executed when the CPU is in SMM needs to implement a HSE mechanism. In this
section, we formalize and verify this mechanism against the model we have
previously introduced.

\subsection{Computing Platform and Security Policy}

We consider three software components: the boot sequence code, the SMM code and
the OS code. During the boot sequence, only the boot sequence code is executed
and it loads both the OS code and the SMM code into memory. At the end of the
boot sequence, the OS kernel is executed. This OS kernel will schedule different
applications. Because applications are less privileged than the OS kernel, we
will not distinguish them from the kernel code. Thus, in the following, OS code
refers to both OS kernel and application codes.

At runtime, both the OS code and the SMM code can be executed. Our objective is
to evaluate the security provided by the hardware to isolate SMM code from OS
code. Thus, we define \[ \set{S} \triangleq \setshortdef{\val{smm}, \val{os}} \]

We assume the SMM is dedicated to the SMM code.  Let
$cpu\_in\_smm : \texttt{Archi}_{\set{S}} \rightarrow \setshortdef{\val{true},
  \val{false}}$ be the function which returns $\val{true}$ if the CPU is in SMM
and $\val{false}$ otherwise.  We define $smm\_context$ a hardware-software
mapping such that
\[ smm\_context(h) \triangleq \begin{cases}
    \val{smm} & \text{if } cpu\_in\_smm(h) = \val{true} \\
    \val{os} & \text{otherwise}
  \end{cases} \] Let $\formatLTS{Smmx86}$ be the Computing Platform such as
\[ \formatLTS{Smmx86} \triangleq \formatLTS{Minx86}(smm\_context) \]

We assume that both the OS code and the SMM code have been loaded in distinct
memory regions. In particular, all the SMM code has been loaded in SMRAM. Our
objective is to enforce a security policy which prevents the OS code to tamper
with the SMM code execution. This way, the SMM (which is the most privileged
execution mode of the CPU) cannot be used to perform an escalation privilege. We
define $smm\_security$ a predicate to model this security policy such as given
$\rho \in \formatLTS{Smmx86}$,
\[ \begin{array}{l} smm\_security(\rho) \triangleq \\ \qquad
    software\_execution\_isolation(smm\_context, minx86\_execute, \rho,
    \setshortdef{\val{smm}}) \end{array} \]

\subsection{HSE Definition}

We define $\Delta_{Smm}$ to model the HSE mechanism applied by the SMM code such
that
$\Delta_{Smm} = (inv_{Smm}, behavior_{Smm}, \setshortdef{\val{smm}},
smm\_context)$ (see Definition \ref{def:hse}).

In order to enforce the SMM security policy, we have identified six requirements
on states.
\begin{compactitem}
\item When the CPU executes the SMM code, the program counter register value
  needs to be an address in SMRAM.
\item The SMBASE register was correctly set during the boot sequence to point to
  the base of the SMRAM.
\item The SMRAM contains only SMM code.
\item For a physical address in SMRAM, in case of cache hit, the related cache
  line content must be owned by the SMM code.
\item In order to protect the content of the SMRAM inside the DRAM memory, the
  boot sequence code has locked the SMRAMC controller. This ensures that an OS
  cannot set the \texttt{d\_open} bit any longer and only a CPU in SMM can
  modify the content of the SMRAM.
\item The range of memory declared with the SMRR needs to overlap with the
  SMRAM.
\end{compactitem}

The Appendix~\ref{app:invsmm} gives the formal definitions of each requirements
and of $inv_{Smm}$.  We now define $\fun{behavior_{Smm}}$. We only define two
restrictions. First, we force the SMM code execution to remain confined within
the SMRAM. The reason is simple: the OS code can tamper with the memory outside
the SMRAM. As a consequence, jumping outside the SMRAM is the best way to fail
the security policy. Secondly, we prevent the SMM code to update the SMRR
registers as it is the responsability of the boot sequence code to correctly set
them.

\[
  \fun{behavior_{Smm}}(h, ev) \triangleq \begin{array}{l}
                                           \fun{smm\_context}(h) = \val{smm} \\
                                           \quad \Rightarrow ((e =
                                           NextInstruction(pa)
                                           \Rightarrow pa \in \texttt{pSmram}) \\
                                           \qquad \wedge\,(e \neq UpdateSmrr(smrr))) \\
                                         \end{array}
                                       \]

                                       For $\Delta_{Smm}$ to be a HSE mechanism,
                                       we need to prove the two HSE Laws (see
                                       Definition \ref{def:laws}).  The first
                                       law states the state requirements modeled
                                       with $inv_{Smm}$ are preserved through
                                       state-transformations if the
                                       transformations which do not satisfy
                                       $behavior_{Smm}$ are discarded. We prove
                                       this by enumeration of
                                       $ev \in \texttt{Event}$ and
                                       $h \in \texttt{Archi}_{Smm}$, we check
                                       that each requirement described
                                       previously is preserved by
                                       $\Delta_{Smm}$. We use those intermediary
                                       results to conclude. The second law
                                       states that the $behavior_{Smm}$
                                       predicate specifies
                                       state-trans\-formation requirements for
                                       the TCB only. In this use case, it means
                                       $behavior_{Smm}$ should always hold true
                                       when the OS code is executed by the
                                       hardware architecture.  By definition of
                                       $behavior_{Smm}$,
                                       $smm\_context(h) = \val{smm}$ is an
                                       antecedent of the conditional.

                                       Let $smm\_secure\_transformation$ be a
                                       predicate which holds true when a
                                       state-transformation does not imply the
                                       OS code to tamper with the execution of
                                       SMM code.
                                       \[ \begin{array}{l}
                                            smm\_secure\_transformation(h, ev) \triangleq \\
                                            \qquad \neg
                                            software\_tampering(smm\_context,
                                            minx86\_execute, h, ev, \val{os},
                                            \val{smm})
                                          \end{array} \]

                                        We prove that this predicate holds true
                                        for a state-transformation with respect
                                        to the HSE mechanism. With this result,
                                        we can prove the HSE mechanism is sound
                                        (see Definition \ref{def:sound}).

\begin{lemma}[Invariants Enforce Security]
  $\forall \transitionLTS{Smmx86}{h}{ev}{h'}$,
  \[ \begin{array}{l}
       inv_{Smm}(h) \\
       \qquad \Rightarrow (ev \in \texttt{Event}_{Soft} \Rightarrow
       behavior_{Smm}(h,ev) \\
       \qquad\qquad \Rightarrow smm\_secure\_transformation(h, ev)
     \end{array} \]
 \end{lemma}

\begin{proof}
  By enumeration of $ev \in \texttt{Event}$ and
  $h \in \texttt{Archi}_{\set{S}}$.
\end{proof}

\begin{theorem}[$\Delta_{Smm}$ is Sound]
  \[ sound(\Delta_{Smm}, smm\_security) \]
\end{theorem}

\begin{proof}
  The "Invariants Enforce Security" lemma applies for one transition and the
  first HSE law allows to reason by induction on runs.
\end{proof}

\section{Discussion} \label{sec:speccert:discuss}

Our effort has been originally motivated by the disclosure of several
vulnerabilities targeting multiple x86 HSE mechanisms for the past few
years\,\cite{wojtczuk2009smram,duflot2009smram,rutkowska2008remap,domas2015sinkhole,kallenberg2015racecondition}. These
attacks do not benefit from a software implementation error but rather from a
flaw in the hardware specifications themselves. The result of our work is a
three-steps methodology for formally specifying and verifying HSE mechanisms
against a hardware architecture model. We believe each aspect is important.

First, the hardware architecture model can be used as a formal specification.
The main benefit of a formal specification is to avoid any ambiguity such as the
one we have found in \cite{intel2009mch}. One can read at Section 3.8.3.8, page
102 that “the OPEN bit must be reset before the LOCK bit is set”.  At the same
page, in the description of the LOCK bit, one can also read that “when [LOCK] is
set to 1 then [OPEN] is reset to 0”. We had modeled the second statement as the
behavior of the memory controller is not specified if the first statement is
true\,\footnote{If we had to actually implement the HSE mechanism, we would have
  to assume the first was the correct one.}  \formatLTS{Minx86} as a formal
specification does not suffer from the same flaw.

Secondly, a formal specification of a HSE mechanism will help software
developers when the time comes to implement it. For instance, the Chapter 34,
Volume 3C of \cite{intel2014manual} about SMM is about 30 pages long, it gives
many details on how the SMM actually works, yet no section is actually dedicated
to security. On the contrary, our HSE mechanism definition gathers six
requirements on hardware configurations and two requirements on software
executions to enforce a well-defined security property. Even if the proofs only
apply to an abstract model, we believe it is a valuable improvement.

Lastly, the verification process of a HSE mechanism specification against a
hardware architecture model may help to highlight hidden flaws in the hardware
specifications assumptions. We take the example of the SMRAM cache poisonning
attack\,\cite{wojtczuk2009smram,duflot2009smram}, which has motivated the
introduction of the SMRR. If an attacker can set the proper cache strategy (WB)
for the SMRAM physical addresses, then the code inside the SMRAM is loaded into
the cache as soon as the CPU in SMM is executing it. From this point forward
---because the access control is enforced at the memory controller level---
nothing prevents the attacker to tamper with it. The next time the CPU enters in
SMM, it executes the code stored in the cache. With a SMRR-less version of
\formatLTS{Minx86}, we were not able to conclude our HSE mechanism was sound:
such a scenario draws attention of the SpecCert user who is forced to
investigate.

From our point of view, the clear separation between the hardware model, the
security properties and the HSE mechanisms to enforce those properties are the
main advantage of our approach, as two different use cases can be studied
against the same hardware model. However, the scalability of SpecCert remains to
be proven.


\chapter{FreeSpec}

\section{Modelling Programs with Effects}

\section{Modular Verification of Programs with Effects}

\section{Discussion}

\chapter{Conclusion}

%----------------------------------------------------------------------------------------
%	THESIS CONTENT - APPENDICES
%----------------------------------------------------------------------------------------

\appendix % Cue to tell LaTeX that the following "chapters" are Appendices

% Include the appendices of the thesis as separate files from the Appendices folder
% Uncomment the lines as you write the Appendices

\chapter{SpecCert Notations}

\chapter{FreeSpec Notations}

%\include{Appendices/AppendixA}
%\include{Appendices/AppendixB}
%\include{Appendices/AppendixC}

%----------------------------------------------------------------------------------------
%	BIBLIOGRAPHY
%----------------------------------------------------------------------------------------

\printbibliography[heading=bibintoc]

%----------------------------------------------------------------------------------------

\end{document}
